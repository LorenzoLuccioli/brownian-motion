% In this file you should put the actual content of the blueprint.
% It will be used both by the web and the print version.
% It should *not* include the \begin{document}
%
% If you want to split the blueprint content into several files then
% the current file can be a simple sequence of \input. Otherwise It
% can start with a \section or \chapter for instance.

\part{Brownian motion}

\paragraph{Overview}

This part of the blueprint is a guide to the formalization of a standard Brownian motion in Lean using Mathlib. There are two main parts to this formalization:
\begin{itemize}
  \item a development of the theory of Gaussian distributions, the construction of a projective family of Gaussian distributions and its projective limit by the Kolmogorov extension theorem,
  \item a proof of a Kolmogorov-Chentsov continuity theorem, following \cite{kratschmer2023kolmogorov}.
\end{itemize}

Putting the two sides together, we then build a stochastic process that fits the definition of a Brownian motion on the real line.

\paragraph{Status} The formalization is complete.

\paragraph{Formalization authors} Rémy Degenne, Markus Himmel, David Ledvinka, Etienne Marion, Peter Pfaffelhuber.

With additional contributions from Jonas Bayer, Lorenzo Loccioli, Pietro Monticone, Alessio Rondelli and Jérémy Scanvic.

\chapter{Characteristic function and covariance}

\section{Characteristic functions}
\label{sec:characteristic_function}


\begin{definition}[Characteristic function]\label{def:charFunDual}
  \mathlibok
  \lean{MeasureTheory.charFunDual}
The characteristic function of a measure $\mu$ on a normed space $E$ is the function $E^* \to \mathbb{C}$ defined by
\begin{align*}
  \hat{\mu}(L) = \int_E e^{i L(x)} \: d\mu(x) \: .
\end{align*}
\end{definition}


\begin{theorem}\label{thm:ext_of_charFunDual}
  \uses{def:charFunDual}
  \mathlibok
  \lean{MeasureTheory.Measure.ext_of_charFunDual}
In a separable Banach space, if two finite measures have same characteristic function, they are equal.
\end{theorem}

\begin{proof}\leanok

\end{proof}


\begin{definition}[Characteristic function]\label{def:charFun}
  \mathlibok
  \lean{MeasureTheory.charFun}
The characteristic function of a measure $\mu$ on an inner product space $E$ is the function $E \to \mathbb{C}$ defined by
\begin{align*}
  \hat{\mu}(t) = \int_E e^{i \langle t, x \rangle} \: d\mu(x) \: .
\end{align*}
This is equal to the normed space version of the characteristic function applied to the linear map $x \mapsto \langle t, x \rangle$.
\end{definition}


\begin{theorem}\label{thm:ext_of_charFun}
  \uses{def:charFun}
  \mathlibok
  \lean{MeasureTheory.Measure.ext_of_charFun}
In a separable Hilbert space, if two finite measures have same characteristic function, they are equal.
\end{theorem}

\begin{proof}\leanok

\end{proof}


\begin{lemma}\label{lem:charFun_map_eq_charFunDual_smul}
  \uses{def:charFun, def:charFunDual}
  \mathlibok
  \lean{MeasureTheory.charFun_map_eq_charFunDual_smul}
Let $\mu$ be a measure on $F$ and let $L \in F^*$. Then
\begin{align*}
  \widehat{L_*\mu}(x) &= \hat{\mu}(x \cdot L) \: .
\end{align*}
\end{lemma}

\begin{proof}\leanok

\end{proof}


\begin{lemma}\label{lem:charFunDual_map}
  \uses{def:charFunDual}
  \mathlibok
  \lean{MeasureTheory.charFunDual_map}
Let $\mu$ be a measure on a normed space $E$ and let $L$ be a continuous linear map from $E$ to $F$.
Then for all $L' \in F^*$,
\begin{align*}
  \widehat{L_*\mu}(L') = \hat{\mu}(L' \circ L) \: .
\end{align*}
\end{lemma}

\begin{proof}\leanok

\end{proof}



\section{Covariance}
\label{sec:covariance}

Let $F$ be a Banach space and $E$ be a Hilbert space.

\begin{definition}[Covariance]\label{def:covarianceBilin}
  \mathlibok
  \lean{ProbabilityTheory.covarianceBilinDual, ProbabilityTheory.covarianceBilinDual_apply, ProbabilityTheory.covarianceBilinDual_apply'}
The covariance bilinear form of a measure $\mu$ on $F$ with finite second moment is the continuous bilinear form $C_\mu : F^* \times F^* \to \mathbb{R}$ with
\begin{align*}
  C_\mu(L_1, L_2)
  &= \int_x (L_1(x) - L_1(m_\mu)) (L_2(x) - L_2(m_\mu)) \: d\mu(x)
  \\
  &= \int_x L_1(x - m_\mu) L_2(x- m_\mu) \: d\mu(x)
  \: .
\end{align*}
\end{definition}

\begin{lemma}\label{lem:covarianceBilin_same_eq_variance}
  \uses{def:covarianceBilin}
  \mathlibok
  \lean{ProbabilityTheory.covarianceBilinDual_self_eq_variance}
For $\mu$ a measure on $F$ with finite second moment and $L \in F^*$, $C_\mu(L, L) = \mathbb{V}_\mu[L]$.
\end{lemma}

\begin{proof}\leanok

\end{proof}


\begin{definition}[Covariance in a Hilbert space]\label{def:covInnerBilin}
  \leanok
  \lean{ProbabilityTheory.covInnerBilin}
The covariance bilinear form of a finite measure $\mu$ with finite second moment on a Hilbert space $E$ is the continuous bilinear form $C_\mu : E \times E \to \mathbb{R}$ with
\begin{align*}
  C'_\mu(x, y) = \int_z \langle x, z - m_\mu \rangle \langle y, z - m_\mu \rangle \: d\mu(z) \: .
\end{align*}
This is $C_\mu$ applied to the linear maps $L_x, L_y \in E^*$ defined by $L_x(z) = \langle x, z \rangle$ and $L_y(z) = \langle y, z \rangle$.
\end{definition}


\begin{lemma}\label{lem:covInnerBilin_map}
  \uses{def:covInnerBilin}
  \leanok
  \lean{ProbabilityTheory.covInnerBilin_map}
Let $E$ and $F$ be two Hilbert spaces with $F$ finite dimensional, $\mu$ a finite measure on $E$ with finite second moment, and $L : E \to F$ a continuous linear map.
Then the covariance bilinear form of the measure $L_*\mu$ is given by
\begin{align*}
  C'_{L_*\mu}(u, v)
  &= C'_\mu(L^\dagger(u), L^\dagger(v))
  \: ,
\end{align*}
in which $L^\dagger : F \to E$ is the adjoint of $L$.
\end{lemma}

\begin{proof}\leanok
\begin{align*}
  C'_{L_*\mu}(u, v)
  &= (L_*\mu)\left[\langle u, x - m_{L_*\mu}\rangle \langle x - m_{L_*\mu}, v \rangle\right]
  \\
  &= \mu\left[\langle u, L(x) - L(m_\mu)\rangle \langle L(x) - L(m_\mu), v \rangle \right]
  \\
  &= \mu\left[\langle L^\dagger(u), x - m_\mu\rangle \langle x - m_\mu, L^\dagger(v) \rangle \right]
  \\
  &= C'_\mu(L^\dagger(u), L^\dagger(v))
  \: .
\end{align*}
\end{proof}


\begin{definition}[Covariance matrix]\label{def:covMatrix}
  \uses{def:IsGaussian, lem:covarianceBilin_same_eq_variance}
  \leanok
  \lean{ProbabilityTheory.covMatrix, ProbabilityTheory.posSemidef_covMatrix}
The covariance matrix of a finite measure $\mu$ with finite second moment on a finite dimensional inner product space $E$ is the positive semidefinite matrix $\Sigma_\mu$ such that for $u, v \in E$,
\begin{align*}
  \langle u, \Sigma_\mu v\rangle = \mu[\langle u, x - m_\mu \rangle \langle x - m_\mu, v \rangle] \: .
\end{align*}
This is the covariance bilinear form $C'_\mu(u, v)$, as a matrix.
\end{definition}


\begin{lemma}\label{lem:covMatrix_map}
  \uses{def:covMatrix}
  \leanok
  \lean{ProbabilityTheory.covMatrix_map}
Let $E$ and $F$ be two finite dimensional inner product spaces, $\mu$ a measure on $E$ with finite second moment, and $L : E \to F$ a continuous linear map.
Then the covariance matrix of the measure $L_*\mu$ has entries
\begin{align*}
  \langle e_i, \Sigma_{L_*\mu} e_j\rangle
  &= \langle L^\dagger(e_i), \Sigma_\mu L^\dagger(e_j)\rangle
  \: ,
\end{align*}
in which $L^\dagger : F \to E$ is the adjoint of $L$.
\end{lemma}

\begin{proof}\leanok
  \uses{lem:covInnerBilin_map}
  On the left-hand side we have
  $$\langle e_i, \Sigma_{L_*\mu} e_j\rangle = C'_{L_*\mu}(e_i, e_j) = C'_\mu(L^\dagger(e_i), L^\dagger(e_j)),$$
  where the last equality comes from Lemma~\ref{lem:covInnerBilin_map}. On the right-hand side we have
  $$\langle L^\dagger(e_i), \Sigma_{L_*\mu} L^\dagger(e_j)\rangle = C'_\mu(L^\dagger(e_i), L^\dagger(e_j)),$$
  which concludes the proof.
\end{proof}

\chapter{Stochastic processes}
\label{chap:process}

Let $T$ be an index set and $\Omega$ a measurable space, with measure $\mathbb{P}$.
A stochastic process is a function $X : T \to \Omega \to E$, where $E$ is another measurable space, such that for all $t \in T$, $X_t : \Omega \to E$ is $\mathbb{P}$-a.e. measurable.


\begin{definition}[Law of a stochastic process]\label{def:processLaw}
  \leanok
The law of a stochastic process $X$ is the measure on the measurable space $E^T$ obtained by pushing forward the measure $\mathbb{P}$ by the map $\omega \mapsto X(\cdot, \omega)$.
\end{definition}

\textbf{Lean remark}: we don't use a Lean definition for the law, but write the map in full.

\begin{definition}[Modification]\label{def:modification}
  \leanok
We say that a stochastic process $Y$ is a \emph{modification} of another stochastic process $X$ if for all $t \in T$, $Y_t =_{\mathbb{P}\text{-a.e.}} X_t$.
\end{definition}

\textbf{Lean remark}: we don't use a Lean definition for being a modification, but write explicitly the condition $\forall t \in T,\ Y_t =_{\mathbb{P}\text{-a.e.}} X_t$~.

\begin{definition}[Indistinguishable]\label{def:indistinguishable}
  \leanok
We say that a stochastic processes $Y$ is a \emph{indistinguishable} from $X$ if $\mathbb{P}$-a.e., for all $t \in T$, $X_t = Y_t$.
\end{definition}

A summary of the next few lemmas is this:
\begin{itemize}
  \item indistinguishable $\implies$ modification $\implies$ same law,
  \item modification and continuous with $T$ separable $\implies$ indistinguishable.
\end{itemize}


\begin{lemma}\label{lem:Indistinguishable.Modification}
  \uses{def:indistinguishable, def:modification}
  \leanok
  \lean{modification_of_indistinguishable}
If $Y$ is indistinguishable from $X$, then $Y$ is a modification of $X$.
\end{lemma}

\begin{proof}\leanok
Obvious.
\end{proof}


\begin{lemma}\label{lem:map_eq_of_modification}
  \uses{def:modification}
  \mathlibok
  \lean{ProbabilityTheory.map_eq_of_forall_ae_eq}
Let $X, Y : T \to \Omega \to E$ be two stochastic processes that are modifications of each other.
Then for all $t_1, \ldots, t_n \in T$, the random vector $(X_{t_1}, \ldots, X_{t_n})$ has the same distribution as the random vector $(Y_{t_1}, \ldots, Y_{t_n})$.
That is, $X$ and $Y$ have same finite-dimensional distributions.
\end{lemma}

\begin{proof}\leanok
By the modification property, almost surely $X_{t_i} = Y_{t_i}$ for all $i \in [n]$.
Thus the function $\omega \mapsto (X_{t_1}(\omega), \ldots, X_{t_n}(\omega))$ is equal to $\omega \mapsto (Y_{t_1}(\omega), \ldots, Y_{t_n}(\omega))$ almost surely, hence the maps of $\mathbb{P}$ by these two functions are equal.
\end{proof}


\begin{lemma}\label{lem:map_eq_iff}
  \uses{def:processLaw}
  \mathlibok
  \lean{ProbabilityTheory.map_eq_iff_forall_finset_map_restrict_eq}
Let $X, Y : T \to \Omega \to E$ be two stochastic processes.
Then $X$ and $Y$ have same finite-dimensional distributions if and only if they have the same law.
\end{lemma}

\begin{proof}\leanok
TODO: consider the $\pi$-system of cylinder sets.
\end{proof}


\begin{lemma}\label{lem:indistinguishable_of_modification_of_continuous}
  \uses{def:modification, def:indistinguishable}
  \leanok
  \lean{indistinguishable_of_modification}
Let $T$ and $E$ be topological spaces and suppose that $T$ is separable Hausdorff.
Let $X, Y : T \to \Omega \to E$ be two stochastic processes that are modifications of each other and are almost surely continuous.
Then $X$ and $Y$ are indistinguishable.
\end{lemma}

\begin{proof}\leanok
Since $T$ is separable, it has a countable dense subset $D$.
Since $D$ is countable,
\begin{align*}
  (\forall t \in D, \mathbb{P}\text{-a.e.}, X_t = Y_t)
  \iff (\mathbb{P}\text{-a.e.}, \forall t \in D, X_t = Y_t)
\end{align*}
Hence by the modification property we have that almost surely, for all $t \in D$, $X_t = Y_t$.
Then almost surely $X$ and $Y$ are continuous functions which are equal on a dense subset of $T$: those two functions are equal everywhere.
\end{proof}

\chapter{Gaussian distributions}
\label{chap:gaussian}

\section{Gaussian measures}
\label{sec:gaussian_measures}

\subsection{Real Gaussian measures}

\begin{definition}[Real Gaussian measure]\label{def:gaussianReal}
  \mathlibok
  \lean{ProbabilityTheory.gaussianReal}
  The real Gaussian measure with mean $\mu \in \mathbb{R}$ and variance $\sigma^2 > 0$ is the measure on $\mathbb{R}$ with density $\frac{1}{\sqrt{2 \pi \sigma^2}} \exp\left(-\frac{(x - \mu)^2}{2 \sigma^2}\right)$ with respect to the Lebesgue measure.
  The real Gaussian measure with mean $\mu \in \mathbb{R}$ and variance $0$ is the Dirac measure $\delta_\mu$.
  We denote this measure by $\mathcal{N}(\mu, \sigma^2)$.
\end{definition}


\begin{lemma}\label{lem:charFun_gaussianReal}
  \uses{def:gaussianReal, def:charFun}
  \mathlibok
  \lean{ProbabilityTheory.charFun_gaussianReal}
The characteristic function of a real Gaussian measure with mean $\mu$ and variance $\sigma^2$ is given by
$x \mapsto \exp\left(i \mu x - \frac{\sigma^2 x^2}{2}\right)$.
\end{lemma}

\begin{proof}\leanok

\end{proof}


\begin{lemma}\label{lem:centralMoment_two_mul_gaussianReal}
  \uses{def:gaussianReal}
  \leanok
  \lean{ProbabilityTheory.centralMoment_two_mul_gaussianReal}
The central moment of order $2n$ of a real Gaussian measure $\mathcal{N}(\mu, \sigma^2)$ is given by
\begin{align*}
  \mathbb{E}[(X - \mu)^{2n}] = \sigma^{2n} (2n - 1)!! \: ,
\end{align*}
in which $(2n - 1)!! = (2n - 1)(2n - 3) \cdots 3 \cdot 1$ is the double factorial of $2n - 1$.
\end{lemma}

\begin{proof}\leanok
\begin{align*}
	\mathbb{E}[(X - \mu)^{2n}] &= \int_{-\infty}^\infty (x - \mu)^{2n} \frac{1}{\sqrt{2 \pi \sigma^2}} e^{-\frac{(x - \mu)^2}{2 \sigma^2}} \mathrm dx \\
	&= \int_{-\infty}^\infty x^{2n} \frac{1}{\sqrt{2 \pi \sigma^2}} e^{-\frac{x^2}{2 \sigma^2}} \mathrm dx \\
	&= 2 \int_{0}^\infty x^{2n} \frac{1}{\sqrt{2 \pi \sigma^2}} e^{-\frac{x^2}{2 \sigma^2}} \mathrm dx \\
	&= 2 \int_{0}^\infty {\sqrt{2 \sigma^2 x}}^{2n} \frac{1}{\sqrt{2 \pi \sigma^2}} e^{-x)} \frac{\sigma^2}{\sqrt{2 \sigma^2 x'}} \mathrm dx \\
	&= \frac{\sigma^{2n} 2^n}{\sqrt{\pi}} \int_{0}^\infty x^{n - 1/2} e{-x} \mathrm dx \\
	&= \frac{\sigma^{2n} 2^n}{\sqrt{\pi}} \Gamma(n + 1/2) \\
	&= \frac{\sigma^{2n} 2^n}{\Gamma(1/2)} \left( \prod_{k=0}^{n-1} (k + 1/2) \right) \Gamma(1/2) \\
	&= \sigma^{2n} \prod_{k=0}^{n-1} (2k + 1) \\
	&= \sigma^{2n} (2n - 1)!!
\end{align*}
\end{proof}


\subsection{Gaussian measures on a Banach space}

That kind of generality is not needed for this project, but we happen to have results about Gaussian measures on a Banach space in Mathlib, so we will use them.
The main reference for this section is \cite{hairer2009introduction}.

Let $F$ be a separable Banach space.

\begin{definition}[Gaussian measure]\label{def:IsGaussian}
  \uses{def:gaussianReal}
  \mathlibok
  \lean{ProbabilityTheory.IsGaussian}
A measure $\mu$ on $F$ is Gaussian if for every continuous linear form $L \in F^*$, the pushforward measure $L_* \mu$ is a Gaussian measure on $\mathbb{R}$.
\end{definition}


\begin{lemma}\label{lem:IsGaussian.IsProbabilityMeasure}
  \uses{def:IsGaussian}
  \mathlibok
A Gaussian measure is a probability measure.
\end{lemma}

\begin{proof}\leanok

\end{proof}


\begin{theorem}\label{thm:isGaussian_iff_charFunDual_eq}
  \uses{def:IsGaussian, def:charFunDual}
  \mathlibok
  \lean{ProbabilityTheory.isGaussian_iff_charFunDual_eq}
A finite measure $\mu$ on $F$ is Gaussian if and only if for every continuous linear form $L \in F^*$, the characteristic function of $\mu$ at $L$ is
\begin{align*}
  \hat{\mu}(L) = \exp\left(i \mu[L] - \mathbb{V}_\mu[L] / 2\right) \: ,
\end{align*}
in which $\mathbb{V}_\mu[L]$ is the variance of $L$ with respect to $\mu$.
\end{theorem}

\begin{proof}\uses{thm:ext_of_charFunDual, lem:charFun_gaussianReal}\leanok

\end{proof}



\paragraph{Transformations of Gaussian measures}

\begin{lemma}\label{lem:isGaussian_map}
  \uses{def:IsGaussian}
  \mathlibok
  \lean{ProbabilityTheory.isGaussian_map}
Let $F, G$ be two Banach spaces, let $\mu$ be a Gaussian measure on $F$ and let $T : F \to G$ be a continuous linear map.
Then $T_*\mu$ is a Gaussian measure on $G$.
\end{lemma}

\begin{proof}\leanok

\end{proof}


\begin{lemma}\label{lem:isGaussian_add_const}
  \uses{def:IsGaussian}
  \leanok
  % This is an instance without name in the code, hence we don't give a \lean{...}.
Let $\mu$ be a Gaussian measure on $F$ and let $c \in F$.
Then the measure $\mu$ translated by $c$ (the map of $\mu$ by $x \mapsto x + c$) is a Gaussian measure on $F$.
\end{lemma}

\begin{proof}\leanok

\end{proof}


\begin{lemma}\label{lem:isGaussian_conv}
  \uses{def:IsGaussian}
  \mathlibok
  \lean{ProbabilityTheory.isGaussian_conv}
The convolution of two Gaussian measures is a Gaussian measure.
\end{lemma}

\begin{proof}\leanok

\end{proof}



\paragraph{Fernique's theorem}


\begin{theorem}\label{thm:exists_integrable_exp_sq_of_map_rotation_eq_self}
  \leanok
  % In Mathlib PR #26291
Let $\mu$ be a finite measure on $F$ such that $\mu \times \mu$ is invariant under the rotation of angle $-\frac{\pi}{4}$.
Then there exists $C > 0$ such that the function $x \mapsto \exp (C \Vert x \Vert ^ 2)$ is integrable with respect to $\mu$.
\end{theorem}

\begin{proof}\leanok

\end{proof}


\begin{lemma}\label{lem:IsGaussian.map_rotation_eq_self}
  \uses{def:IsGaussian}
  \leanok
  % In Mathlib PR #26291
For a Gaussian measure $\mu$, $\mu \times \mu$ is invariant by rotation.
\end{lemma}

\begin{proof}\leanok
  \uses{lem:isGaussian_conv}

\end{proof}


\begin{theorem}[Fernique's theorem]\label{thm:IsGaussian.exists_integrable_exp_sq}
  \uses{def:IsGaussian}
  \leanok
  % In Mathlib PR #26291
For a Gaussian measure, there exists $C > 0$ such that the function $x \mapsto \exp (C \Vert x \Vert ^ 2)$ is integrable.
\end{theorem}

\begin{proof}\leanok
  \uses{thm:isGaussian_iff_charFunDual_eq, lem:IsGaussian.IsProbabilityMeasure, thm:exists_integrable_exp_sq_of_map_rotation_eq_self, lem:IsGaussian.map_rotation_eq_self}

\end{proof}


\begin{lemma}\label{lem:IsGaussian.memLp_id}
  \uses{def:IsGaussian}
  \leanok
  \lean{ProbabilityTheory.IsGaussian.memLp_id}
A Gaussian measure $\mu$ has finite moments of all orders.
In particular, there is a well defined mean $m_\mu := \mu[\mathrm{id}]$, and for all $L \in F^*$, $\mu[L] = L(m_\mu)$.
\end{lemma}

\begin{proof}\leanok
  \uses{thm:IsGaussian.exists_integrable_exp_sq}

\end{proof}

A Gaussian measure has finite second moment by Lemma~\ref{lem:IsGaussian.memLp_id}, hence its covariance bilinear form is well defined.


\subsection{Gaussian measures on a finite dimensional Hilbert space}

We specialize directly from Banach space to finite dimensional Hilbert space since that's what we need in this project, although there are results for Gaussian measures on infinite dimensional Hilbert spaces that would worth stating.

\begin{lemma}\label{lem:isGaussian_iff_charFun_eq}
  \uses{def:IsGaussian, def:charFunDual, def:charFun}
  \leanok
  \lean{ProbabilityTheory.isGaussian_iff_charFun_eq}
A finite measure $\mu$ on a Hilbert space $E$ is Gaussian if and only if for every $t \in E$, the characteristic function of $\mu$ at $t$ is
\begin{align*}
  \hat{\mu}(t) =  \exp\left(i \mu[\langle t, \cdot \rangle] - \mathbb{V}_\mu[\langle t, \cdot \rangle] / 2\right) \: .
\end{align*}
\end{lemma}

\begin{proof}\leanok
  \uses{thm:isGaussian_iff_charFunDual_eq}
By Theorem~\ref{thm:isGaussian_iff_charFunDual_eq}, $\mu$ is Gaussian iff for every continuous linear form $L \in E^*$, the characteristic function of $\mu$ at $L$ is
\begin{align*}
  \hat{\mu}(L) = \exp\left(i \mu[L] - \mathbb{V}_\mu[L] / 2\right) \: .
\end{align*}
Every continuous linear form $L \in E^*$ can be written as $L(x) = \langle t, x \rangle$ for some $t \in E$, hence we have that $\mu$ is Gaussian iff for every $t \in E$,
\begin{align*}
  \hat{\mu}(t) = \exp\left(i \mu[\langle t, \cdot \rangle] - \mathbb{V}_\mu[\langle t, \cdot \rangle] / 2\right) \: .
\end{align*}
\end{proof}

Let $E$ be a separable Hilbert space. We denote by $\langle \cdot, \cdot \rangle$ the inner product on $E$ and by $\Vert \cdot \Vert$ the associated norm.

\begin{lemma}\label{lem:IsGaussian.charFun_eq}
  \uses{def:IsGaussian, def:charFun, def:covInnerBilin}
  \leanok
  \lean{ProbabilityTheory.IsGaussian.charFun_eq}
The characteristic function of a Gaussian measure $\mu$ on $E$ is given by
\begin{align*}
  \hat{\mu}(t) = \exp\left(i \langle t, m_\mu \rangle - \frac{1}{2} C'_\mu(t, t)\right) \: .
\end{align*}
\end{lemma}

\begin{proof}\leanok
  \uses{lem:isGaussian_iff_charFun_eq, lem:IsGaussian.memLp_id, lem:covarianceBilin_same_eq_variance}
By Lemma~\ref{lem:isGaussian_iff_charFun_eq}, for every $t \in E$,
\begin{align*}
  \hat{\mu}(t) = \exp\left(i \mu[\langle t, \cdot \rangle] - \mathbb{V}_\mu[\langle t, \cdot \rangle] / 2\right) \: .
\end{align*}
By Lemma~\ref{lem:IsGaussian.memLp_id}, $\mu$ has finite first moment and $\mu[\langle t, \cdot \rangle] = \langle t, m_\mu \rangle$. By the same lemma, $\mu$ has finite second moment and for any $t$ we have $\mathbb{V}_\mu[\langle t, \cdot\rangle] = C'_\mu(t, t)$.
\end{proof}

\begin{lemma}\label{lem:isGaussian_iff_gaussian_charFun}
  \uses{def:IsGaussian, def:charFun, def:covMatrix}
  \leanok
  \lean{ProbabilityTheory.isGaussian_iff_gaussian_charFun, ProbabilityTheory.gaussian_charFun_congr}
A finite measure $\mu$ on $E$ is Gaussian if and only if there exists $m \in E$ and $C$ positive semidefinite such that for all $t \in E$, the characteristic function of $\mu$ at $t$ is
\begin{align*}
  \hat{\mu}(t) = \exp\left(i \langle t, m \rangle - \frac{1}{2} C(t, t)\right) \: ,
\end{align*}
If that's the case, then $m = m_\mu$ and $C = C'_\mu$.
\end{lemma}

Note that this lemma does not say that there exists a Gaussian measure for any such $m$ and $C$.
We will prove that later.

\begin{proof}\leanok
  \uses{lem:IsGaussian.charFun_eq, lem:charFun_map_eq_charFunDual_smul, thm:ext_of_charFun}
Lemma~\ref{lem:IsGaussian.charFun_eq} states that the characteristic function of a Gaussian measure has the wanted form.

Suppose now that there exists $m \in E$ and $C$ positive semidefinite such that for all $t \in E$, $\hat{\mu}(t) = \exp\left(i \langle t, m \rangle - \frac{1}{2} C(t, t)\right)$.

We need to show that for all $L \in E^*$, $L_*\mu$ is a Gaussian measure on $\mathbb{R}$.
Such an $L$ can be written as $\langle u, \cdot \rangle$ for some $u \in E$.
Let then $u \in E$. We compute the characteristic function of $\langle u, \cdot\rangle_*\mu$ at $x \in \mathbb{R}$ with Lemma~\ref{lem:charFun_map_eq_charFunDual_smul}:
\begin{align*}
  \widehat{\langle u, \cdot\rangle_*\mu}(x)
  &= \hat{\mu}(x \cdot u)
  \\
  &= \exp\left(i x \langle u, m \rangle - \frac{1}{2} x^2 C(u, u)\right)
  \: .
\end{align*}
This is the characteristic function of a Gaussian measure on $\mathbb{R}$ with mean $\langle u, m \rangle$ and variance $C(u, u)$.
By Theorem~\ref{thm:ext_of_charFun}, $\langle u, \cdot\rangle_*\mu$ is Gaussian, hence $\mu$ is Gaussian.

By Lemma~\ref{lem:IsGaussian.charFun_eq}, we deduce that for any $t \in E$ we have
$$\exp\left(i\langle t, m \rangle - \frac{1}{2} C(t, t)\right) = \exp\left(i\langle t, m_\mu \rangle - \frac{1}{2} C'_\mu(t, t)\right).$$
In particular, for any $t$ there exists $n_t \in \mathbb{Z}$ such that
$$i\langle t, m \rangle - \frac{1}{2} C(t, t) = i\langle t, m_\mu \rangle - \frac{1}{2} C'_\mu(t, t) + 2i\pi n_t.$$
We deduce that $n$ is a continuous map from $E$ to $\mathbb{Z}$, and thus must be constant because $E$ is connected. By looking at the value at $t = 0$, we deduce that for any $t$, $n_t = 0$. Looking at real and imaginary parts we obtain that for any $t$,
$$\langle t, m \rangle = \langle t, m_\mu \rangle \quad \text{and} \quad C(t, t) = C'_\mu(t, t).$$
We immediately deduce that $m = m_\mu$. Moreover, because $C$ and $C'_\mu$ are symmetric, they are characterized by their values on the diagonal. Indeed, for any $x, y$,
$$C(x, y) = \frac{1}{2} (C(x + y, x + y) - C(x, x) - C(y, y)).$$
We deduce that $C = C'_\mu$.
\end{proof}

\begin{lemma}\label{lem:IsGaussian.ext_iff}
  \uses{def:IsGaussian, def:covInnerBilin}
  \leanok
  \lean{ProbabilityTheory.IsGaussian.ext, ProbabilityTheory.IsGaussian.ext_iff}
Two Gaussian measures $\mu$ and $\nu$ on a separable Hilbert space are equal if and only if they have same mean and same covariance.
\end{lemma}

\begin{proof}\leanok
  \uses{thm:ext_of_charFun, lem:IsGaussian.charFun_eq}
The forward direction is immediate.

For the converse direction, it is enough to show that $\mu$ and $\nu$ have the same characteristic function by Theorem~\ref{thm:ext_of_charFun}. As they are both Gaussian, their characteristic functions only depend on their mean and covariance by Lemma~\ref{lem:IsGaussian.charFun_eq}. Thus they are equal.
\end{proof}


\begin{definition}[Standard Gaussian measure]\label{def:stdGaussian}
  \uses{def:gaussianReal}
  \leanok
  \lean{ProbabilityTheory.stdGaussian}
Let $(e_1, \ldots, e_d)$ be an orthonormal basis of $E$ and let $\mu$ be the standard Gaussian measure on $\mathbb{R}$.
The standard Gaussian measure on $E$ is the pushforward measure of the product measure $\mu \times \ldots \times \mu$ by the map $x \mapsto \sum_{i=1}^d x_i \cdot e_i$.
\end{definition}

The fact that this definition does not depend on the choice of basis will be a consequence of the fact that its characteristic function does not depend on the basis.


\begin{lemma}\label{lem:integral_eval_pi}
  \mathlibok
  \lean{MeasureTheory.integral_comp_eval}
For $\mu_1, \ldots, \mu_d$ probability measures on $\mathbb{R}$ and $f : \mathbb{R} \to \mathbb{R}$ integrable with respect to $\mu_i$, we have
\begin{align*}
  \int_x f(x_i) \, d(\mu_1 \times \ldots \times \mu_d)(x)
  = \int_x f(x) \, d\mu_i
  \: .
\end{align*}
\end{lemma}

\begin{proof}\leanok
As $f$ is integrable, we can use Fubini theorem to obtain that
$$\int f(x_i) \, d(\mu_1 \times \ldots \times \mu_d)(x) = \int f(x) \, d\mu_i(x) \times \prod_{j \ne i} \int 1 \, d\mu_j(x) = \int f(x) \, d\mu_i(x)$$
because the $\mu_j$s are probability measures.
\end{proof}


\begin{lemma}\label{lem:isCentered_stdGaussian}
  \uses{def:stdGaussian}
  \leanok
  \lean{ProbabilityTheory.isCentered_stdGaussian}
The standard Gaussian measure on $E$ is centered, i.e., $\mu[L] = 0$ for every $L \in E^*$.
\end{lemma}

\begin{proof}\leanok
  \uses{lem:integral_eval_pi}

\end{proof}


\begin{lemma}\label{lem:isProbabilityMeasure_stdGaussian}
  \uses{def:stdGaussian}
  \leanok
  \lean{ProbabilityTheory.isProbabilityMeasure_stdGaussian}
The standard Gaussian measure is a probability measure.
\end{lemma}

\begin{proof}\leanok

\end{proof}


\begin{lemma}\label{lem:charFun_stdGaussian}
  \uses{def:stdGaussian, def:charFun}
  \leanok
  \lean{ProbabilityTheory.charFun_stdGaussian}
The characteristic function of the standard Gaussian measure on $E$ is given by
\begin{align*}
  \hat{\mu}(t) = \exp\left(-\frac{1}{2} \Vert t \Vert^2 \right) \: .
\end{align*}
\end{lemma}

\begin{proof}\leanok
  \uses{lem:charFun_gaussianReal}
Denote by $\nu$ the standard Gaussian measure on $\mathbb{R}$. This is a straightforward computation:
\begin{align*}
  \hat{\mu}(t) = \int \exp\left(i\langle t, \sum_{j=1}^d x_j \cdot e_j \rangle\right) d(\nu \times \ldots \times \nu)(dx) &= \int \exp\left(\sum_{j=1}^d ix_j\langle t, e_j \rangle\right) d(\nu \times \ldots \times \nu)(dx) \\
  &= \int \prod_{j=1}^d \exp\left(ix_j\langle t, e_j \rangle\right) d(\nu \times \ldots \times \nu)(dx) \\
  &= \prod_{j=1}^d \int \exp\left(ix\langle t, e_j \rangle\right) d\nu(x) \\
  &= \prod_{j=1}^d \exp\left(-\frac{\langle t, e_j \rangle^2}{2}\right) \\
  &= \exp\left(-\frac{1}{2} \Vert t \Vert^2 \right).
\end{align*}
\end{proof}


\begin{lemma}\label{lem:isGaussian_stdGaussian}
  \uses{def:stdGaussian, def:IsGaussian}
  \leanok
  \lean{ProbabilityTheory.isGaussian_stdGaussian}
The standard Gaussian measure on $E$ is a Gaussian measure.
\end{lemma}

\begin{proof}\leanok
  \uses{lem:isGaussian_iff_gaussian_charFun, lem:charFun_stdGaussian, lem:isProbabilityMeasure_stdGaussian}
Since the standard Gaussian is a probability measure (hence finite), we can apply Lemma~\ref{lem:isGaussian_iff_gaussian_charFun} that states that it suffices to show that the characteristic function has a particular form.
That form is given by Lemma~\ref{lem:charFun_stdGaussian}, taking $m=0$ and $C = \langle\cdot, \cdot\rangle$.
\end{proof}


\begin{lemma}\label{lem:integral_id_stdGaussian}
  \uses{def:stdGaussian}
  \leanok
  \lean{ProbabilityTheory.integral_id_stdGaussian}
The mean of the standard Gaussian measure is $0$.
\end{lemma}

\begin{proof}\leanok
  \uses{lem:integral_eval_pi}

\end{proof}


\begin{lemma}\label{lem:covMatrix_stdGaussian}
  \uses{def:stdGaussian, def:covMatrix}
  \leanok
  \lean{ProbabilityTheory.covMatrix_stdGaussian}
The covariance matrix of the standard Gaussian measure is the identity matrix.
\end{lemma}

\begin{proof}\leanok
  \uses{lem:isGaussian_iff_gaussian_charFun, lem:charFun_stdGaussian}
From Lemma~\ref{lem:charFun_stdGaussian}, we know that for all $t \in \mathbb{R}$,
$$\hat{\mu}(t) = \exp\left(-\frac{\|t\|^2}{2}\right) = \exp\left(-\frac{\langle t, \mathrm{I}t\rangle}{2}\right).$$
As the identity is positive semidefinite, we deduce from Lemma~\ref{lem:isGaussian_iff_gaussian_charFun} that $\Sigma_\mu$ is the identity matrix.
\end{proof}


\begin{definition}[Multivariate Gaussian]\label{def:multivariateGaussian}
  \uses{def:stdGaussian}
  \leanok
  \lean{ProbabilityTheory.multivariateGaussian}
The multivariate Gaussian measure on $\mathbb{R}^d$ with mean $m \in \mathbb{R}^d$ and covariance matrix $\Sigma \in \mathbb{R}^{d \times d}$, with $\Sigma$ positive semidefinite, is the pushforward measure of the standard Gaussian measure on $\mathbb{R}^d$ by the map $x \mapsto m + \Sigma^{1/2} x$.
We denote this measure by $\mathcal{N}(m, \Sigma)$.
\end{definition}


\begin{lemma}\label{lem:integral_id_multivariateGaussian}
  \uses{def:multivariateGaussian}
  \leanok
  \lean{ProbabilityTheory.integral_id_multivariateGaussian}
The mean of the multivariate Gaussian measure $\mathcal{N}(m, \Sigma)$ is $m$.
\end{lemma}

\begin{proof}\leanok
  \uses{lem:integral_id_stdGaussian}

\end{proof}


\begin{lemma}\label{lem:covMatrix_multivariateGaussian}
  \uses{def:multivariateGaussian}
  \leanok
  \lean{ProbabilityTheory.covInnerBilin_multivariateGaussian}
The covariance matrix of the multivariate Gaussian measure $\mathcal{N}(m, \Sigma)$ is $\Sigma$.
\end{lemma}

\begin{proof}\leanok
  \uses{lem:covMatrix_stdGaussian}

\end{proof}


\begin{lemma}\label{lem:isGaussian_multivariateGaussian}
  \uses{def:multivariateGaussian, def:IsGaussian}
  \leanok
  \lean{ProbabilityTheory.isGaussian_multivariateGaussian}
A multivariate Gaussian measure is a Gaussian measure.
\end{lemma}

\begin{proof}\leanok
  \uses{lem:isGaussian_stdGaussian, lem:isGaussian_add_const, lem:isGaussian_map}
The multivariate Gaussian measure is the pushforward of the standard Gaussian measure by an affine map, and is thus Gaussian by Lemma~\ref{lem:isGaussian_add_const} and Lemma~\ref{lem:isGaussian_map}.
\end{proof}


\begin{theorem}\label{thm:charFun_multivariateGaussian}
  \uses{def:multivariateGaussian, def:charFun}
  \leanok
  \lean{ProbabilityTheory.charFun_multivariateGaussian}
The characteristic function of a multivariate Gaussian measure $\mathcal{N}(m, \Sigma)$ is given by
\begin{align*}
  \hat{\mu}(t) = \exp\left(i \langle m, t \rangle - \frac{1}{2} \langle t, \Sigma t \rangle\right)
  \: .
\end{align*}
\end{theorem}

\begin{proof}\leanok
  \uses{lem:isGaussian_multivariateGaussian, lem:IsGaussian.charFun_eq, lem:integral_id_multivariateGaussian, lem:covMatrix_multivariateGaussian}
Since the multivariate Gaussian measure is a Gaussian measure, we can apply Lemma~\ref{lem:IsGaussian.charFun_eq} to it.
It suffices then to show that the mean and the covariance matrix of the multivariate Gaussian measure are equal to $m$ and $\Sigma$, respectively.
This is given by Lemma~\ref{lem:integral_id_multivariateGaussian} and Lemma~\ref{lem:covMatrix_multivariateGaussian}.
\end{proof}


\section{Gaussian processes}
\label{sec:gaussian_processes}

\begin{definition}[Gaussian process]\label{def:IsGaussianProcess}
  \uses{def:IsGaussian}
  \leanok
  \lean{ProbabilityTheory.IsGaussianProcess}
A process $X : T \to \Omega \to E$ is Gaussian if for every finite subset $t_1, \ldots, t_n \in T$, the random vector $(X_{t_1}, \ldots, X_{t_n})$ has a Gaussian distribution.
\end{definition}


\begin{lemma}\label{lem:isGaussianProcess_of_modification}
  \uses{def:IsGaussianProcess}
  \leanok
  \lean{ProbabilityTheory.IsGaussianProcess.modification}
Let $X, Y : T \to \Omega \to E$ be two stochastic processes that are modifications of each other (that is, for all $t \in T$, $X_t =_{a.e.} Y_t$).
If $X$ is a Gaussian process, then $Y$ is a Gaussian process as well.
\end{lemma}

\begin{proof}\leanok
  \uses{lem:map_eq_of_modification}
Being a Gaussian process is defined in terms of the distribution of finite-dimensional random vectors.
By Lemma~\ref{lem:map_eq_of_modification}, the random vector $(Y_{t_1}, \ldots, Y_{t_n})$ has the same distribution as the random vector $(X_{t_1}, \ldots, X_{t_n})$ for all $t_1, \ldots, t_n \in T$.
\end{proof}

\chapter{Projective family of the Brownian motion}
\label{chap:projective_family}


\section{Kolmogorov extension theorem}

This theorem has been formalized by Rémy Degenne and Peter Pfaffelhuber in the repository \href{https://github.com/RemyDegenne/kolmogorov_extension4}{kolmogorov\_extension4}.

\begin{definition}[Projective family]\label{def:IsProjectiveMeasureFamily}
  \mathlibok
  \lean{MeasureTheory.IsProjectiveMeasureFamily}
A family of measures $P$ indexed by finite sets of $T$ is projective if, for finite sets $J \subseteq I$, the projection from $E^I$ to $E^J$ maps $P_I$ to $P_J$.
\end{definition}


\begin{definition}[Projective limit]\label{def:IsProjectiveLimit}
  \uses{def:IsProjectiveMeasureFamily}
  \mathlibok
  \lean{MeasureTheory.IsProjectiveLimit}
A measure $\mu$ on $E^T$ is the projective limit of a projective family of measures $P$ indexed by finite sets of $T$ if, for every finite set $I \subseteq T$, the projection from $E^T$ to $E^I$ maps $\mu$ to $P_I$.
\end{definition}


\begin{theorem}[Kolmogorov extension theorem]\label{thm:kolmogorovExtension}
  \uses{def:IsProjectiveLimit, def:IsProjectiveMeasureFamily}
  \leanok
  \lean{MeasureTheory.projectiveLimit, MeasureTheory.IsProjectiveLimit.unique, MeasureTheory.isProjectiveLimit_projectiveLimit, MeasureTheory.isFiniteMeasure_projectiveLimit, MeasureTheory.isProbabilityMeasure_projectiveLimit}
Let $\mathcal{X}$ be a Polish space, equipped with the Borel $\sigma$-algebra, and let $T$ be an index set.
Let $P$ be a projective family of finite measures on $\mathcal{X}$.
Then the projective limit $\mu$ of $P$ exists, is unique, and is a finite measure on $\mathcal{X}^T$.
Moreover, if $P_I$ is a probability measure for every finite set $I \subseteq T$, then $\mu$ is a probability measure.
\end{theorem}

\begin{proof}\leanok

\end{proof}


\section{Projective family of Gaussian measures}

We build a projective family of Gaussian measures indexed by $\mathbb{R}_+$.
In order to do so, we need to define specific Gaussian measures on finite index sets $\{t_1, \ldots, t_n\}$.
We want to build a multivariate Gaussian measure on $\mathbb{R}^n$ with mean $0$ and covariance matrix $C_{ij} = \min(t_i, t_j)$ for $1 \leq i,j \leq n$.

We prove that the matrix $C_{ij} = \min(t_i, t_j)$ is positive semidefinite, which means that there exists a Gaussian distribution with mean 0 and covariance matrix $C$.

\begin{definition}[Gram matrix]\label{def:gramMatrix}
  \mathlibok
  \lean{Matrix.gram}
Let $v_1, \ldots, v_n$ be vectors in an inner product space $E$.
The Gram matrix of $v_1, \ldots, v_n$ is the matrix in $\mathbb{R}^{n \times n}$ with entries $G_{ij} = \langle v_i, v_j \rangle$ for $1 \leq i,j \leq n$.
\end{definition}


\begin{lemma}\label{lem:posSemidef_gramMatrix}
  \uses{def:gramMatrix}
  \mathlibok
  \lean{Matrix.posSemidef_gram}
A gram matrix is positive semidefinite.
\end{lemma}

\begin{proof}\leanok
Symmetry is obvious from the definition.
Let $x \in E$. Then
\begin{align*}
  \langle x, G x \rangle
  &= \sum_{i,j} x_i x_j \langle v_i, v_j \rangle
  \\
  &= \langle \sum_i x_i v_i, \sum_j x_j v_j \rangle
  \\
  &= \left\Vert \sum_i x_i v_i \right\Vert^2
  \\
  &\ge 0
  \: .
\end{align*}
\end{proof}


\begin{lemma}\label{lem:C_eq_gramMatrix}
  \uses{def:gramMatrix}
  \leanok
Let $I = \{t_1, \ldots, t_n\}$ be a finite subset of $\mathbb{R}_+$.
For $i \le n$, let $v_i = \mathbb{I}_{[0, t_i]}$ be the indicator function of the interval $[0, t_i]$, as an element of $L^2(\mathbb{R})$.
Then the Gram matrix of $v_1, \ldots, v_n$ is equal to the matrix $C_{ij} = \min(t_i, t_j)$ for $1 \leq i,j \leq n$.
\end{lemma}

\begin{proof}\leanok
By definition of the inner product in $L^2(\mathbb{R})$,
\begin{align*}
  \langle v_i, v_j \rangle
  &= \int_{\mathbb{R}} \mathbb{I}_{[0, t_i]}(x) \mathbb{I}_{[0, t_j]}(x) \: dx
  = \min(t_i, t_j)
  \: .
\end{align*}
\end{proof}


\begin{lemma}\label{lem:posSemidef_brownianCov}
  \leanok
  \lean{ProbabilityTheory.posSemidef_brownianCovMatrix}
For $I = \{t_1, \ldots, t_n\}$ a finite subset of $\mathbb{R}_+$, let $C \in \mathbb{R}^{n \times n}$ be the matrix $C_{ij} = \min(t_i, t_j)$ for $1 \leq i,j \leq n$.
Then $C$ is positive semidefinite.
\end{lemma}

\begin{proof}\leanok
  \uses{lem:C_eq_gramMatrix, lem:posSemidef_gramMatrix}
$C$ is a Gram matrix by Lemma~\ref{lem:C_eq_gramMatrix}.
By Lemma~\ref{lem:posSemidef_gramMatrix}, it is positive semidefinite.
\end{proof}


\paragraph{Definition of the projective family and extension}

\begin{definition}[Projective family of the Brownian motion]\label{def:gaussianProjectiveFamily}
  \uses{def:multivariateGaussian, lem:posSemidef_brownianCov}
  \leanok
  \lean{ProbabilityTheory.gaussianProjectiveFamily}
For $I = \{t_1, \ldots, t_n\}$ a finite subset of $\mathbb{R}_+$, let $P^B_I$ be the multivariate Gaussian measure on $\mathbb{R}^n$ with mean $0$ and covariance matrix $C_{ij} = \min(t_i, t_j)$ for $1 \leq i,j \leq n$.
We call the family of measures $P^B_I$ the \emph{projective family of the Brownian motion}.
\end{definition}


\begin{lemma}\label{lem:isProjectiveMeasureFamily_gaussianProjectiveFamily}
  \uses{def:gaussianProjectiveFamily, def:IsProjectiveMeasureFamily}
  \leanok
  \lean{ProbabilityTheory.isProjectiveMeasureFamily_gaussianProjectiveFamily}
The projective family of the Brownian motion is a projective family of measures.
\end{lemma}

\begin{proof}\leanok
  \uses{lem:isGaussian_multivariateGaussian, lem:covMatrix_map,
  lem:integral_id_multivariateGaussian, lem:covMatrix_multivariateGaussian, lem:IsGaussian.ext_iff}
Let $J \subseteq I$ be finite subsets of $\mathbb{R}_+$.
We need to show that the restriction from $\mathbb{R}^I$ to $\mathbb{R}^J$ (denote it by $\pi_{IJ}$) maps $P^B_I$ to $P^B_J$.

The restriction is a continuous linear map from $\mathbb{R}^I$ to $\mathbb{R}^J$.
The map of a Gaussian measure by a continuous linear map is Gaussian (Lemma~\ref{lem:isGaussian_map}).
It thus suffices to show that the mean and covariance matrix of the map are equal to the ones of $P^B_J$ by Lemma~\ref{lem:IsGaussian.ext_iff}.

The mean of the map is $0$, since the mean of $P^B_I$ is $0$ and the map is linear.

Let us turn to the covariance matrix. For any $i \in J$, the map $x : \mathbb{R}^I \mapsto \pi_{IJ}(x) i$ is equal to $x : \mathbb{R}^I \mapsto x i$. Let $i, j \in J$. The covariance of $x : \mathbb{R}^J \mapsto x i$ and $x : \mathbb{R}^J \mapsto x j$ with respect to ${\pi_{IJ}}_*P^B_J$ is equal to the covariance of $x : \mathbb{R}^I \mapsto \pi_{IJ}(x) i$ and $x : \mathbb{R}^I \mapsto \pi_{IJ}(x) j$ with respect to $P^B_I$, which is equal to the covariance of $x : \mathbb{R}^I \mapsto x i$ and $x : \mathbb{R}^I \mapsto x i$ with respect to $P^B_I$, which is equal to $t_i \land t_j$. But this is also the covariance of $x : \mathbb{R}^J \mapsto x i$ and $x : \mathbb{R}^J \mapsto x j$ with respect to $P^B_J$, so we are done.
\end{proof}


\begin{definition}\label{def:gaussianLimit}
  \uses{thm:kolmogorovExtension, lem:isProjectiveMeasureFamily_gaussianProjectiveFamily}
  \leanok
  \lean{ProbabilityTheory.gaussianLimit}
We denote by $P_B$ the projective limit of the projective family of the Brownian motion given by Theorem~\ref{thm:kolmogorovExtension}.
This is a probability measure on $\mathbb{R}^{\mathbb{R}_+}$.
\end{definition}

\chapter{Kolmogorov-Chentsov Theorem}
\label{chap:kolmogorov_chentsov}

We follow the proof of the Kolmogorov-Chentsov theorem from \cite{kratschmer2023kolmogorov}.
That proof notably uses the chaining technique developed by Talagrand \cite{talagrand2022upper}.

That theorem is about stochastic processes $X : T \to \Omega \to E$, where $\Omega$ is a measurable space with a probability measure $\mathbb{P}$, the index set $T$ is a metric space with distance $d_T$, and $E$ is also a metric space with distance $d_E$, on which we put the Borel $\sigma$-algebra.

The main result is Theorem~\ref{thm:countable_set_bound}.
Under an assumption on the covering number of $T$, for a process $X$ that satisfies the Kolmogorov condition $\mathbb{E}[d_E(X_s, X_t)^p] \le M d_T(s, t)^q$ (see Definition~\ref{def:IsKolmogorovProcess}),the theorem gives a finite bound on the expectation of the supremum of the ratio
\begin{align*}
  \mathbb{E}\left[ \sup_{s, t \in T'} \frac{d_E(X_s, X_t)^p}{d_T(s, t)^{\beta p}} \right]
  \: ,
\end{align*}
for $T'$ a countable subset of $T$.
As a corollary, we obtain that there exists a modification of $X$ with Hölder continuous paths.

In Lean, we will use the typeclass \texttt{PseudoEMetricSpace} for both $T$ and $E$ as long as possible, and then specialize to \texttt{EMetricSpace} (or perhaps even \texttt{MetricSpace}) when we need the stronger properties of a metric space.
For example, to prove the existence of a modification of a stochastic process, we will eventually use the fact that $d_E(x, y) = 0$ implies $x = y$, which does not hold in a pseudo-metric space.
All distances will be expressed with \texttt{edist}, which takes values in \texttt{ENNReal}, and the integrals refer to Lebesgue integrals.

\section{Covers and covering numbers}

Let $(E, d_E)$ be a pseudo-metric space.

\begin{definition}[$\varepsilon$-cover]\label{def:IsCover}
  \leanok
  \lean{IsCover}
  A set $C \subseteq E$ is an $\varepsilon$-cover of a set $A \subseteq E$ if for every $x \in A$, there exists $y \in C$ such that $d_E(x, y) \le \varepsilon$.
\end{definition}


\begin{definition}[External covering number]\label{def:externalCoveringNumber}
  \uses{def:IsCover}
  \leanok
  \lean{externalCoveringNumber}
  The external covering number of a set $A \subseteq E$ for $\varepsilon \ge 0$ is the smallest cardinality of an $\varepsilon$-cover of $A$.
  Denote it by $N^{ext}_\varepsilon(A)$.
\end{definition}


\begin{definition}[Internal covering number]\label{def:internalCoveringNumber}
  \uses{def:IsCover}
  \leanok
  \lean{internalCoveringNumber}
  The internal covering number of a set $A \subseteq E$ for $\varepsilon \ge 0$ is the smallest cardinality of an $\varepsilon$-cover of $A$ which is a subset of $A$.
  Denote it by $N^{int}_\varepsilon(A)$.
\end{definition}


\begin{definition}[Separated set]\label{def:IsSeparated}
  \mathlibok
  \lean{Metric.IsSeparated}
A set $c \subseteq E$ is $\varepsilon$-separated if for all $x, y \in c$, $d_E(x, y) > \varepsilon$.
\end{definition}


\begin{definition}[Packing number]\label{def:packingNumber}
  \uses{def:IsSeparated}
  \leanok
  \lean{packingNumber}
The packing number of a set $A \subseteq E$ for $\varepsilon > 0$ is the largest cardinality of an $\varepsilon$-separated subset of $A$.
Denote it by $P_\varepsilon(A)$.
\end{definition}


\begin{lemma}\label{lem:externalCoveringNumber_le_internalCoveringNumber}
  \uses{def:externalCoveringNumber, def:internalCoveringNumber}
  \leanok
  \lean{externalCoveringNumber_le_internalCoveringNumber}
$N^{ext}_\varepsilon(A) \le N^{int}_\varepsilon(A)$.
\end{lemma}

\begin{proof}\leanok

\end{proof}


\begin{lemma}\label{lem:internalCoveringNumber_le_packingNumber}
  \uses{def:internalCoveringNumber, def:packingNumber}
  \leanok
  \lean{internalCoveringNumber_le_packingNumber}
$N^{int}_\varepsilon(A) \le P_\varepsilon(A)$.
\end{lemma}

\begin{proof}\leanok

\end{proof}


\begin{lemma}\label{lem:packingNumber_two_le_externalCoveringNumber}
  \uses{def:packingNumber, def:externalCoveringNumber}
  \leanok
  \lean{packingNumber_two_le_externalCoveringNumber}
$P_{2\varepsilon}(A) \le N^{ext}_\varepsilon(A)$.
\end{lemma}

\begin{proof}\leanok

\end{proof}


\begin{lemma}\label{lem:internalCoveringNumber_eq_one_of_diam_le}
  \uses{def:internalCoveringNumber}
  \leanok
  \lean{internalCoveringNumber_eq_one_of_diam_le}
If $\mathrm{diam}(A) \le \varepsilon$ and $A$ is nonempty, then $N^{int}_\varepsilon(A) = 1$.
\end{lemma}

\begin{proof}\leanok

\end{proof}


\begin{lemma}\label{lem:externalCoveringNumber_mono}
  \uses{def:externalCoveringNumber}
  \leanok
  \lean{externalCoveringNumber_mono_set}
For $B \subseteq A$, $N^{ext}_\varepsilon(B) \le N^{ext}_{\varepsilon}(A)$.
\end{lemma}

\begin{proof}\leanok

\end{proof}


\begin{lemma}\label{lem:internalCoveringNumber_subset_le}
  \uses{def:internalCoveringNumber}
  \leanok
  \lean{internalCoveringNumber_subset_le}
For $B \subseteq A$, $N^{int}_\varepsilon(B) \le N^{int}_{\varepsilon/2}(A)$.
\end{lemma}

\begin{proof}\leanok
  \uses{lem:internalCoveringNumber_le_packingNumber, lem:packingNumber_two_le_externalCoveringNumber, lem:externalCoveringNumber_mono, lem:externalCoveringNumber_le_internalCoveringNumber}
\begin{align*}
  N^{int}_\varepsilon(B)
  &\le P_{\varepsilon}(B)
  \le N^{ext}_{\varepsilon/2}(B)
  \le N^{ext}_{\varepsilon/2}(A)
  \le N^{int}_{\varepsilon/2}(A)
  \: .
\end{align*}
\end{proof}


\subsection{Covering number and volume}

In this section $E$ is a finite dimensional inner product space, with dimension $d$.

\begin{lemma}\label{lem:volume_le_of_isCover}
  \uses{def:IsCover}
  \leanok
  \lean{volume_le_of_isCover}
Let $A \subseteq E$ and $C \subseteq E$ be a finite $\varepsilon$-cover of $A$. Denote by $V(A)$ the volume of $A$.
Then $V(A) \le \vert C \vert V(B_\varepsilon)$, in which $B_\varepsilon$ is the closed ball of radius $\varepsilon$ in $E$.
\end{lemma}

\begin{proof}\leanok
Since $C$ is a cover of $A$, $A$ is a subset of the union of the closed balls $B_\varepsilon(c)$ for $c \in C$. Then
\begin{align*}
  V(A) \le V(\bigcup_{c \in C} B_\varepsilon(c))
  &\le \sum_{c \in C} V(B_\varepsilon(c))
  = \vert C \vert V(B_\varepsilon)
  \: .
\end{align*}
\end{proof}

The volume of $B_\varepsilon$ is given by the following formula (see \href{https://leanprover-community.github.io/mathlib4_docs/Mathlib/MeasureTheory/Measure/Lebesgue/VolumeOfBalls.html#InnerProductSpace.volume_closedBall}{InnerProductSpace.volume\_closedBall}).
\begin{align*}
  V(B_\varepsilon) = \frac{\pi^{d/2}}{\Gamma(d/2 + 1)} \varepsilon^d
  \: .
\end{align*}


\begin{lemma}\label{lem:volume_le_externalCoveringNumber_mul}
  \uses{def:externalCoveringNumber}
  \leanok
  \lean{volume_le_externalCoveringNumber_mul}
If $0 < \varepsilon$ then $V(A) \le N^{ext}_\varepsilon(A) V(B_\varepsilon)$.
\end{lemma}

\begin{proof}\leanok
  \uses{lem:volume_le_of_isCover}
If $A$ has no $\varepsilon$-cover, then $N^{ext}_\varepsilon(A) = \infty$, and because $0 < \varepsilon$, we have that $0 < V(B_\varepsilon)$, so the right-hand side is infinite and the inequality follows.

Otherwise there exists an $\varepsilon$-cover which realizes $N^{ext}_\varepsilon(A)$. We can conclude by Lemma~\ref{lem:volume_le_of_isCover}.
\end{proof}


\begin{lemma}\label{lem:le_volume_of_isSeparated}
  \uses{def:IsSeparated}
  \leanok
  \lean{le_volume_of_isSeparated}
Let $A \subseteq E$ and let $S \subseteq A$ be an $\varepsilon$-separated set.
Then $\vert S \vert V(B_{\varepsilon/2}) \le V(A + B_{\varepsilon/2})$.
\end{lemma}

\begin{proof}\leanok
Since $S$ is $\varepsilon$-separated, the closed balls $B_{\varepsilon/2}(s)$ for $s \in S$ are pairwise disjoint.
Furthermore, these balls are contained in $A + B_{\varepsilon/2}$.
Thus, we have
\begin{align*}
  \vert S \vert V(B_{\varepsilon/2})
  &= \sum_{s \in S} V(B_{\varepsilon/2}(s))
  = V(\bigcup_{s \in S} B_{\varepsilon/2}(s))
  \le V(A + B_{\varepsilon/2})
  \: .
\end{align*}
\end{proof}


\begin{lemma}\label{lem:packingNumber_mul_le_volume}
  \uses{def:packingNumber}
  \leanok
  \lean{packingNumber_mul_le_volume}
$P_\varepsilon(A) V(B_{\varepsilon/2}) \le V(A + B_{\varepsilon/2})$.
\end{lemma}

\begin{proof}\leanok
  \uses{lem:le_volume_of_isSeparated}
Use Lemma~\ref{lem:le_volume_of_isSeparated} with $S$ an $\varepsilon$-separated set of maximal cardinality.
\end{proof}


\begin{lemma}\label{lem:volume_div_le_internalCoveringNumber}
  \uses{def:internalCoveringNumber}
  \leanok
  \lean{volume_div_le_internalCoveringNumber}
If $0 < \varepsilon$ then $\frac{V(A)}{V(B_\varepsilon)} \le N^{int}_\varepsilon(A)$.
\end{lemma}

\begin{proof}\leanok
  \uses{lem:volume_le_externalCoveringNumber_mul, lem:externalCoveringNumber_le_internalCoveringNumber}
We have $\frac{V(A)}{V(B_\varepsilon)} \le N^{ext}_\varepsilon(A)$ by Lemma~\ref{lem:volume_le_externalCoveringNumber_mul} and $N^{ext}_\varepsilon(A) \le N^{int}_\varepsilon(A)$ by Lemma~\ref{lem:externalCoveringNumber_le_internalCoveringNumber}.
\end{proof}


\begin{lemma}\label{lem:internalCoveringNumber_le_volume_div}
  \uses{def:internalCoveringNumber}
  \leanok
  \lean{internalCoveringNumber_le_volume_div}
If $0 < \varepsilon < \infty$ then $N^{int}_\varepsilon(A) \le \frac{V(A + B_{\varepsilon/2})}{V(B_{\varepsilon/2})}$.
\end{lemma}

\begin{proof}\leanok
  \uses{lem:packingNumber_mul_le_volume, lem:internalCoveringNumber_le_packingNumber}
We have $N^{int}_\varepsilon(A) \le P_\varepsilon(A)$ by Lemma~\ref{lem:internalCoveringNumber_le_packingNumber} and $P_\varepsilon(A) \le \frac{V(A + B_{\varepsilon/2})}{V(B_{\varepsilon/2})}$ by Lemma~\ref{lem:packingNumber_mul_le_volume}.
\end{proof}


\begin{lemma}\label{lem:internalCoveringNumber_closedBall_ge}
  \uses{def:internalCoveringNumber}
  \leanok
  \lean{internalCoveringNumber_closedBall_ge}
$N_\varepsilon^{int}(B_1) \ge \frac{1}{\varepsilon^d}$.
\end{lemma}

\begin{proof}\leanok
  \uses{lem:volume_div_le_internalCoveringNumber}
By Lemma~\ref{lem:volume_div_le_internalCoveringNumber},
\begin{align*}
  N^{int}_\varepsilon(B_1)
  &\ge \frac{V(B_1)}{V(B_\varepsilon)}
  = \frac{1}{\varepsilon^d}
  \: .
\end{align*}

\end{proof}


\begin{lemma}\label{lem:internalCoveringNumber_closedBall_le}
  \uses{def:internalCoveringNumber}
  \leanok
  \lean{internalCoveringNumber_closedBall_le}
$N_\varepsilon^{int}(B_1) \le \left(\frac{2}{\varepsilon} + 1\right)^d$.
\end{lemma}

\begin{proof}\leanok
  \uses{lem:internalCoveringNumber_le_volume_div}
By Lemma~\ref{lem:internalCoveringNumber_le_volume_div},
\begin{align*}
  N^{int}_\varepsilon(B_1)
  &\le \frac{V(B_1 + B_{\varepsilon/2})}{V(B_{\varepsilon/2})}
  = \frac{V(B_{1 + \varepsilon/2})}{V(B_{\varepsilon/2})}
  = \frac{(1 + \varepsilon/2)^d}{(\varepsilon/2)^d}
  \: .
\end{align*}
\end{proof}


\subsection{Bounded internal covering number}

\begin{definition}[Bounded internal covering number]\label{def:HasBoundedInternalCoveringNumber}
  \uses{def:internalCoveringNumber}
  \leanok
  \lean{HasBoundedInternalCoveringNumber}
  Let $\mathrm{diam}(A)$ be the diameter of $A \subseteq E$, i.e. $\mathrm{diam}(A) = \sup_{x,y \in A} d_E(x, y)$.
  A set $A \subseteq E$ has bounded internal covering number with constant $c>0$ and exponent $t>0$ if for all $\varepsilon \in (0, \mathrm{diam}(A)]$, $N^{int}_\varepsilon(A) \le c \varepsilon^{-t}$.
\end{definition}


\begin{lemma}\label{lem:hasBoundedInternalCoveringNumber_unitInterval}
  \uses{def:HasBoundedInternalCoveringNumber}
  \leanok
  \lean{internalCoveringNumber_Icc_zero_one_le_one_div}
The unit interval $I = [0, 1] \subseteq \mathbb{R}$ has bounded internal covering number with constant $1$ and exponent $1$: for $\varepsilon \le 1$, $N^{int}_\varepsilon(I) \le 1/\varepsilon$.
\end{lemma}

\begin{proof}\leanok

\end{proof}


\begin{lemma}\label{lem:hasBoundedInternalCoveringNumber_subset}
  \uses{def:HasBoundedInternalCoveringNumber}
  \leanok
  \lean{HasBoundedInternalCoveringNumber.subset}
If $A$ has bounded internal covering number with constant $c>0$ and exponent $d>0$, then for all $B \subseteq A$, $B$ has bounded internal covering number with constant $2^d c$ and exponent $d$.
\end{lemma}

\begin{proof}\leanok
  \uses{lem:internalCoveringNumber_subset_le}
\begin{align*}
  N^{int}_\varepsilon(B)
  &\le N^{int}_{\varepsilon/2}(A)
  \le c (\varepsilon/2)^{-d}
  = 2^d c \varepsilon^{-d}
  \: .
\end{align*}
\end{proof}


\section{Chaining}

\subsection{Chaining sequence}


\begin{definition}\label{def:nearestPt}
  \leanok
  \lean{nearestPt}
Let $S$ be a finite set of $E$ and $x \in E$.
We denote by $\pi(x, S)$ the point in $S$ which is closest to $x$, i.e. a point such that $d_E(x, S) = \min_{y \in S} d_E(x, y)$ (chosen arbitrarily among the minima if there are several).
\end{definition}


\begin{lemma}\label{lem:dist_nearestPt_le}
  \uses{def:nearestPt}
  \leanok
  \lean{edist_nearestPt_le}
Let $S$ be a finite set of $E$ and $x \in E$.
Then for all $y \in S$, $d_E(x, \pi(x, S)) \le d_E(x, y)$.
\end{lemma}

\begin{proof}\leanok
By definition.
\end{proof}


\begin{lemma}\label{lem:dist_nearestPt_of_isCover}
  \uses{def:nearestPt, def:IsCover}
  \leanok
  \lean{edist_nearestPt_of_isCover}
Let $C_\varepsilon$ be a finite $\varepsilon$-cover of $A \subseteq E$ (assuming such a finite cover exists).
Then for all $x \in A$, $d_E(x, \pi(x, C_\varepsilon)) \le \varepsilon$.
\end{lemma}

\begin{proof}\leanok

\end{proof}


\begin{definition}[Chaining sequence]\label{def:chainingSequence}
  \uses{def:nearestPt, def:IsCover}
  \leanok
  \lean{chainingSequence}
Let $(\varepsilon_n)_{n \in \mathbb{N}}$ be a sequence of positive numbers, $C_n$ a finite $\varepsilon_n$-cover of $A \subseteq E$ with $C_n \subseteq A$ and $x \in C_k$ for some $k \in \mathbb{N}$.
We define the chaining sequence of $x$, denoted $(\bar{x}_i)_{i \le k}$, recursively as follows: $\bar{x}_k = x$ and for $i < k$, $\bar{x}_i = \pi(\bar{x}_{i+1}, C_i)$.
\end{definition}


\begin{lemma}\label{lem:chainingSequence_mem}
  \uses{def:chainingSequence}
  \leanok
  \lean{chainingSequence_mem}
Let $(\varepsilon_n)_{n \in \mathbb{N}}$ be a sequence of positive numbers, $C_n$ a finite $\varepsilon_n$-cover of $A \subseteq E$ with $C_n \subseteq A$ and $x \in C_k$ for some $k \in \mathbb{N}$.
Then for all $i \le k$, $\bar{x}_i\in C_i$.
\end{lemma}

\begin{proof}\leanok
By definition.
\end{proof}


\begin{lemma}\label{lem:dist_chainingSequence_add_one}
  \uses{def:chainingSequence}
  \leanok
  \lean{edist_chainingSequence_add_one}
Let $(\varepsilon_n)_{n \in \mathbb{N}}$ be a sequence of positive numbers, $C_n$ a finite $\varepsilon_n$-cover of $A \subseteq E$ with $C_n \subseteq A$ and $x \in C_k$ for some $k \in \mathbb{N}$.
Then for all $i < k$, $d_E(\bar{x}_i, \bar{x}_{i+1}) \le \varepsilon_i$.
\end{lemma}

\begin{proof}\leanok
  \uses{lem:dist_nearestPt_of_isCover, lem:chainingSequence_mem}
Apply Lemma~\ref{lem:dist_nearestPt_of_isCover} with $S = C_i$ and $x = \bar{x}_{i+1}$.
\end{proof}


\begin{lemma}\label{lem:dist_chainingSequence_le_sum}
  \uses{def:chainingSequence}
  \leanok
  \lean{edist_chainingSequence_le_sum}
Let $(\varepsilon_n)_{n \in \mathbb{N}}$ be a sequence of positive numbers, $C_n$ a finite $\varepsilon_n$-cover of $A \subseteq E$ with $C_n \subseteq A$ and $x \in C_k$ for some $k \in \mathbb{N}$.
Then for $m \le k$, $d_E(\bar{x}_m, x) \le \sum_{i=m}^{k-1} \varepsilon_i$.
\end{lemma}

\begin{proof}\leanok
  \uses{lem:dist_chainingSequence_add_one}
By the triangle inequality and Lemma~\ref{lem:dist_chainingSequence_add_one},
\begin{align*}
  d_E(\bar{x}_m, x)
  \le \sum_{i=m}^{k-1} d_E(\bar{x}_i, \bar{x}_{i+1})
  \le \sum_{i=m}^{k-1} \varepsilon_i
  \: .
\end{align*}
\end{proof}


\begin{lemma}\label{lem:dist_chainingSequence_le}
  \uses{def:chainingSequence}
  \leanok
  \lean{edist_chainingSequence_le}
Let $(\varepsilon_n)_{n \in \mathbb{N}}$ be a sequence of positive numbers, $C_n$ a finite $\varepsilon_n$-cover of $A \subseteq E$ with $C_n \subseteq A$.
Let $m, k, \ell \in \mathbb{N}$ with $m \le k$ and $m \le \ell$ and let $x \in C_k$ and $y \in C_\ell$.
Then
\begin{align*}
  d_E(\bar{x}_m, \bar{y}_m)
  &\le d_E(x, y) + \sum_{i=m}^{k-1} \varepsilon_i + \sum_{j=m}^{\ell-1} \varepsilon_j
\end{align*}
\end{lemma}

\begin{proof}\leanok
  \uses{lem:dist_chainingSequence_le_sum}
Triangle inequality and Lemma~\ref{lem:dist_chainingSequence_le_sum}.
\end{proof}


\begin{corollary}\label{cor:dist_chainingSequence_pow_two_le}
  \uses{def:chainingSequence}
  \leanok
  \lean{edist_chainingSequence_pow_two_le}
For $\varepsilon_n = \varepsilon_0 2^{-n}$, with the hypothesis of Lemma~\ref{lem:dist_chainingSequence_le}, we have
\begin{align*}
  d_E(\bar{x}_m, \bar{y}_m)
  &\le d_E(x, y) + \varepsilon_0 2^{-m+2}
  \: .
\end{align*}
\end{corollary}

\begin{proof}\leanok
  \uses{lem:dist_chainingSequence_le}

\end{proof}


\subsection{A subset of pairs}

We will be interested in bounding expressions of the form $\sup_{s,t\in J, d_T(s,t) \le c} d_E(f(s), f(t))$ for a finite set $J$ and some function $f : T \to E$.
This is a supremum over pairs in $J$ and there could be $\vert J \vert^2$ such pairs.
We will build a subset $K$ of $J^2$ which is much smaller, of size linear in $\vert J \vert$, such that its points are not too far apart and
\begin{align*}
  \sup_{s,t\in J, d_T(s,t) \le c} d_E(f(s), f(t))
  & \le 2 \sup_{(s,t) \in K} d_E(f(s), f(t))
  \: .
\end{align*}
The pairs $(s, t) \in K$ will still be close together, in the sense that $d_T(s, t) \le c n$ for some $n$ that is logarithmic in the size of $J$.

For $t \in V \subseteq T$ and $u\ge 0$, we denote by $B_V(t, u)$ the closed ball with center $t$ and radius $u$ in $V$.
That is, $B_V(t, u) = \{s \in V \mid d_T(s, t) \le u\}$.

We want to cover $J$ with balls that have radius logarithmic in the number of points of the ball.

\begin{definition}\label{def:logSizeRadius}
  \mathlibok
  \lean{PairReduction.logSizeRadius}
Let $V$ be a finite subset of a metric space and let $t \in V$ and $a > 1$, $c > 0$.
Let the \emph{log-size radius} of $t$ in $V$, denoted by $r_{V,t}$, be the smallest positive integer $r$ such that $\vert B_V(t, r c) \vert \le a^{r}$.
\end{definition}


\begin{lemma}\label{lem:card_logSizeRadius_ge}
  \uses{def:logSizeRadius}
  \mathlibok
  \lean{PairReduction.pow_logSizeRadius_le_card_le_logSizeRadius}
$a^{r_{V,t}-1} \le \vert B_V(t, (r_{V,t}-1)c) \vert$~.
\end{lemma}

\begin{proof}\leanok

\end{proof}


\begin{lemma}\label{lem:card_logSizeRadius_le}
  \uses{def:logSizeRadius}
  \mathlibok
  \lean{PairReduction.card_le_logSizeRadius_le_pow_logSizeRadius}
$\vert B_V(t, r_{V,t}c) \vert \le a^{r_{V,t}}$~.
\end{lemma}

\begin{proof}\leanok

\end{proof}


\begin{definition}[Log-size ball sequence]\label{def:logSizeBallSequence}
  \uses{def:logSizeRadius}
  \mathlibok
  \lean{PairReduction.logSizeBallSeq}
Let $(T,d_T)$ be a metric space and let $J \subseteq T$ be finite, $a,c \in \mathbb R_+$ with $a \ge 1$ and $n \in \{1, 2, ...\}$ such that $|J| \le a^n$.
An log-size ball sequence for $(J, a, c, n)$ is a sequence of $(V_i, t_i, r_i)_{i \in \mathbb{N}}$ such that
\begin{itemize}
  \item $V_0 = J$, $t_0$ is an arbitrary point in $J$,
  \item for all $i$, $r_i$ is the log-size radius of $t_i$ in $V_i$,
  \item $V_{i+1} = V_i \setminus B_{V_i}(t_i, (r_i - 1)c)$, $t_{i+1}$ is arbitrarily chosen in $V_{i+1}$.
\end{itemize}
\end{definition}

A log-size ball sequence gives a partition of $J$ into sets which are contained in balls of radius $(r_i - 1)c$ around $t_i$, and satisfy cardinality constraints.


\begin{lemma}\label{lem:logSizeRadius_logSizeBallSequence_le}
  \uses{def:logSizeBallSequence}
  \mathlibok
  \lean{PairReduction.radius_logSizeBallSeq_le}
The radius of a log-size ball sequence $(V_i, t_i, r_i)_{i \in \mathbb{N}}$ for $(J, a, c, n)$ satisfies $r_i \le n$ for all $i \in \mathbb{N}$.
\end{lemma}

\begin{proof}
\leanok
Since $|J| \le a^n$, we have $\vert B_{V_i}(t_i, n c) \vert \le \vert J \vert \le a^{n}$.
\end{proof}


\begin{lemma}\label{lem:logSizeBallSequence_V_anti}
  \uses{def:logSizeBallSequence}
  \mathlibok
  \lean{PairReduction.finset_logSizeBallSeq_add_one_subset}
The sets $V_i$ of a log-size ball sequence $(V_i, t_i, r_i)_{i \in \mathbb{N}}$ are a decreasing sequence of sets. That is, $V_{i+1} \subseteq V_i$ for all $i \in \mathbb{N}$.
\end{lemma}

\begin{proof}
\leanok
$V_{i+1} = V_i \setminus B_{V_i}(t_i, (r_i - 1)c)$ hence $V_{i+1} \subseteq V_i$.
\end{proof}


\begin{lemma}\label{lem:logSizeBallSequence_eq_zero}
  \uses{def:logSizeBallSequence}
  \mathlibok
  \lean{PairReduction.card_finset_logSizeBallSeq_card_eq_zero}
For any log-size ball sequence $(V_i, t_i, r_i)_{i \in \mathbb{N}}$ for $(J, a, c, n)$, for all $k \ge \vert J \vert$, $V_k = \emptyset$.
\end{lemma}

\begin{proof}
  \leanok
  \uses{lem:logSizeBallSequence_V_anti}
$V_{i+1} = V_i \setminus B_{V_i}(t_i, (r_i - 1)c)$ and since $t_i \in B_{V_i}(t_i, (r_i - 1)c)$, we have $\vert V_{i+1} \vert < \vert V_i \vert$ and the cardinal eventually reaches $0$, in at most $\vert J \vert$ steps.
\end{proof}


\begin{lemma}\label{lem:logSizeBallSequence_disjoint_B}
  \uses{def:logSizeBallSequence}
  \mathlibok
  \lean{PairReduction.disjoint_smallBall_logSizeBallSeq}
For $i \ne j$, the balls $B_{V_i}(t, (r_i-1)c)$ and $B_{V_j}(t_j, (r_j-1)c)$ of a log-size ball sequence $(V_i, t_i, r_i)_{i \in \mathbb{N}}$ are disjoint.
\end{lemma}

\begin{proof}
  \leanok
  \uses{lem:logSizeBallSequence_V_anti}
Assume w.l.o.g. that $i < j$.
Then $B_{V_j}(t_j, (r_j-1)c) \subseteq V_j \subseteq V_{i+1}$.
It suffices to show that $B_{V_i}(t_i, (r_i-1)c)$ and $V_{i+1}$ are disjoint.
This follows from the definition of $V_{i+1} = V_i \setminus B_{V_i}(t_i, (r_i-1)c)$.
\end{proof}


\begin{definition}\label{def:pairSet}
  \uses{def:logSizeBallSequence}
  \mathlibok
  \lean{PairReduction.pairSet}
Let $(V_i, t_i, r_i)_{i \in \mathbb{N}}$ be a log-size ball sequence for $(J, a, c, n)$.
For $i \in \mathbb{N}$, let $K_i = \{t_i\} \times B_{V_i}(t_i, r_i c)$ be the set of pairs $(t_i, s)$ for $s$ in the ball $B_{V_i}(t_i, r_i c)$.
We define $K = \bigcup_{i=0}^{\vert J \vert-1} K_i$, set of all pairs from the log-size ball sequence.
\end{definition}


\begin{lemma}\label{lem:card_pairSet_le}
  \uses{def:pairSet}
  \mathlibok
  \lean{PairReduction.card_pairSet_le}
The cardinal of the pair set $K$ of a log-size ball sequence for $(J, a, c, n)$ satisfies $|K| \le a |J|$.
\end{lemma}

\begin{proof}
  \leanok
  \uses{lem:card_logSizeRadius_ge, lem:card_logSizeRadius_le, lem:logSizeBallSequence_disjoint_B}
Using Lemma~\ref{lem:card_logSizeRadius_le}, the cardinal of $K$ is bounded by
\begin{align*}
  \vert K \vert
  &\le \sum_{i=0}^{m-1} \vert K_i \vert
  \le \sum_{i=0}^{m-1} a^{r_i}
  \: .
\end{align*}
Since the sets $B_{V_i}(t_i, (r_i-1)c)$ are disjoint by Lemma~\ref{lem:logSizeBallSequence_disjoint_B}, we can use Lemma~\ref{lem:card_logSizeRadius_ge} to get
\begin{align*}
  \sum_{i=0}^{m-1} a^{r_i - 1}
  \le \sum_{i=0}^{m-1} \vert B_{V_i}(t_i, (r_i-1)c) \vert
  = \left\vert \bigcup_{i=0}^{m-1} B_{V_i}(t_i, (r_i-1)c) \right\vert
  \le \vert J \vert
  \: .
\end{align*}
We obtained the inequality $\vert K \vert \le a \vert J \vert$
\end{proof}


\begin{lemma}\label{lem:dist_le_of_mem_pairSet}
  \uses{def:pairSet}
  \mathlibok
  \lean{PairReduction.edist_le_of_mem_pairSet}
Let $(s, t)$ be a pair in the pair set $K$ of a log-size ball sequence for $(J, a, c, n)$.
Then $d_T(s, t) \le c n$.
\end{lemma}

\begin{proof}
  \leanok
  \uses{lem:logSizeRadius_logSizeBallSequence_le}
A pair $(t, s) \in K$ is of the form $(t_i, s)$ for $s \in B_V(t_i, r_i c)$ and satisfies
\begin{align*}
  d_T(t_i, s) \le c r_i \le c n \: .
\end{align*}
The last inequality is from Lemma~\ref{lem:logSizeRadius_logSizeBallSequence_le}.
\end{proof}


\begin{lemma}\label{lem:sup_dist_le_two_mul_sup_dist_pairSet}
  \uses{def:pairSet}
  \mathlibok
  \lean{PairReduction.iSup_edist_pairSet}
Let $K$ be the pair set of a log-size ball sequence $(V_i, t_i, r_i)_{i \in \mathbb{N}}$ for $(J, a, c, n)$.
Then for any function $f : T \to E$ with $(E,d_E)$ a metric space,
\begin{align*}
  \sup_{s,t\in J, d_T(s,t) \le c} d_E(f(s), f(t))
  & \le 2 \sup_{(s,t) \in K} d_E(f(s), f(t))
  \: .
\end{align*}
\end{lemma}

\begin{proof}
\leanok
Let $(s, t) \in J^2$ such that $d_T(s, t) \le c$.
Then there exists a largest $\ell \in \mathbb{N}$ such that $s, t \in V_\ell$.
Assume w.l.o.g. that $s \notin V_{\ell + 1}$. Then $s \in B_{V_\ell}(t_\ell, (r_\ell-1)c)$ (since $V_{\ell + 1} = V_\ell \setminus B_{V_\ell}(t_\ell, (r_\ell-1)c)$), which implies $d_T(s, t_\ell) \le (r_\ell - 1)c$.

Since $d_T(s, t) \le c$, $d_T(t, t_\ell) \le d_T(t, s) + d_T(s, t_\ell) \le r_\ell c$, hence $t \in B_{V_\ell}(t_\ell, r_\ell c)$ and we have that both $s$ and $t$ are in $B_{V_\ell}(t_\ell, r_\ell c)$.
Thus both $(t_\ell, s)$ and $(t_\ell, t)$ are in $K_\ell \subseteq K$.
Finally
\begin{align*}
  d_E(f(s), f(t))
  &\le d_E(f(s), f(t_\ell)) + d_E(f(t_\ell), f(t))
  \\
  &\le 2\sup_{(s',t') \in K} d_E(f(s'), f(t'))
  \: .
\end{align*}
\end{proof}


\begin{lemma}\label{lem:pair_reduction}
  \uses{def:pairSet}
  \mathlibok
  \lean{EMetric.pair_reduction}
Let $(T,d_T)$ be a metric space.
Let $J \subseteq T$ be finite, $a > 1$, $c>0$ and $n \in \{1, 2, ...\}$ such that $|J| \le a^n$.
Then, there is $K \subseteq J^2$ such that for any function $f : T \to E$ with $(E,d_E)$ a metric space,
\begin{align}
  |K|
  & \le a |J|
  \:, \label{eq:chain1} \\
  \forall (s,t) \in K,
  &\:  d_T(s,t) \le c n
  \:, \label{eq:chain2} \\
  \sup_{s,t\in J, d_T(s,t) \le c} d_E(f(s), f(t))
  & \le 2 \sup_{(s,t) \in K} d_E(f(s), f(t))
  \: . \label{eq:chain3}
\end{align}
\end{lemma}

\begin{proof}\leanok
  \uses{lem:card_pairSet_le, lem:dist_le_of_mem_pairSet, lem:sup_dist_le_two_mul_sup_dist_pairSet}
Let $(V_i, t_i, r_i)_{i \in \mathbb{N}}$ be a log-size ball sequence for $(J, a, c, n)$. We show that its pair set satisfies the conditions of the lemma.

Equation~\eqref{eq:chain1} is given by Lemma~\ref{lem:card_pairSet_le}.
The second property~\eqref{eq:chain2} is Lemma~\ref{lem:dist_le_of_mem_pairSet}.
Equation~\eqref{eq:chain3} was proved in Lemma~\ref{lem:sup_dist_le_two_mul_sup_dist_pairSet}.
\end{proof}





\section{Chaining for stochastic processes under Kolmogorov conditions}

\subsection{Kolmogorov condition}

\begin{definition}[Kolmogorov condition]\label{def:IsKolmogorovProcess}
  \mathlibok
  \lean{ProbabilityTheory.IsAEKolmogorovProcess}
Let $X : T \to \Omega \to E$ be a stochastic process, where $(T, d_T)$ and $(E, d_E)$ are pseudo-metric spaces and $(\Omega, \mathbb{P})$ is a measure space.
Let $p, q > 0$.
We say that $X$ satisfies the Kolmogorov condition for exponents $(p,q)$ with constant $M$ if for all $s, t \in T$, $(X_s, X_t)$ is $\mathbb{P}$-a.e. measurable for the Borel $\sigma$-algebra on $E^2$ and
\begin{align*}
  \mathbb{E}[d_E(X_s, X_t)^p] \le M d_T(s, t)^q
  \: .
\end{align*}
\end{definition}

Remark: the measurability condition on the pair would be implied by the measurability of $X_t$ for all $t \in T$ if we assumed that $E$ is separable (\texttt{SecondCountableTopology} in Lean), which implies that the Borel $\sigma$-algebra on the product is equal to the product of the Borel $\sigma$-algebras.
We follow \cite{kratschmer2023kolmogorov} and do not require separability.


\begin{lemma}\label{lem:IsKolmogorovProcess.edist_eq_zero}
  \uses{def:IsKolmogorovProcess}
  \mathlibok
  \lean{ProbabilityTheory.IsAEKolmogorovProcess.edist_eq_zero}
If $X : T \to \Omega \to E$ is a process that satisfies the Kolmogorov condition for exponents $(p,q)$ with constant $M$ and $s, t \in T$ are such that $d_T(s, t) = 0$, then $\mathbb{P}$-a.e. $d_E(X_s, X_t) = 0$.
\end{lemma}

\begin{proof}\leanok
It suffices to show that $d_E(X_s, X_t)^p = 0$ almost everywhere, which is in turn implied by $\mathbb{E}[d_E(X_s, X_t)^p] \le M d_t(s, t)^q = 0$.
\end{proof}


\begin{lemma}\label{lem:IsKolmogorovProcess.lintegral_sup_rpow_edist_eq_zero}
  \uses{def:IsKolmogorovProcess}
  \leanok
  \lean{ProbabilityTheory.IsAEKolmogorovProcess.lintegral_sup_rpow_edist_eq_zero}
Let $X : T \to \Omega \to E$ be a process that satisfies the Kolmogorov condition for exponents $(p,q)$ with constant $M$.
Let $T'$ be a countable subset of $T$ such that for all $s, t \in T'$, $d_T(s, t) = 0$.
Then
\begin{align*}
  \mathbb{E}\left[ \sup_{s, t \in T'} d_E(X_s, X_t)^p \right]
  &= 0
  \: .
\end{align*}
\end{lemma}

\begin{proof}\leanok
  \uses{lem:IsKolmogorovProcess.edist_eq_zero}
Since $T'$ is countable, we get from Lemma~\ref{lem:IsKolmogorovProcess.edist_eq_zero} that almost surely, for all $s, t \in T'$, $d_E(X_s, X_t)^p = 0$.
In particular the expectation of the supremum is $0$.
\end{proof}


\paragraph{Measurability}

\begin{lemma}\label{lem:IsKolmogorovProcess.aemeasurable}
  \uses{def:IsKolmogorovProcess}
  \mathlibok
  \lean{ProbabilityTheory.IsAEKolmogorovProcess.aemeasurable}
If $X : T \to \Omega \to E$ is a function that satisfies the Kolmogorov condition, then for all $t \in T$, $X_t$ is $\mathbb{P}$-a.e. measurable.
\end{lemma}

\begin{proof}\leanok

\end{proof}


\begin{lemma}\label{lem:aemeasurable_pair_of_aemeasurable}
  \leanok
  \lean{ProbabilityTheory.aemeasurable_pair_of_aemeasurable}
If $E$ is separable and $X : T \to \Omega \to E$ is a process such that $X_t$ is $\mathbb{P}$-a.e. measurable for all $t \in T$, then for all $s, t \in T$, the pair $(X_s, X_t)$ is $\mathbb{P}$-a.e. measurable for the Borel $\sigma$-algebra on $E^2$.
\end{lemma}

\begin{proof}\leanok

\end{proof}


\begin{lemma}\label{lem:IsKolmogorovProcess.aemeasurable_edist}
  \uses{def:IsKolmogorovProcess}
  \mathlibok
  \lean{ProbabilityTheory.IsAEKolmogorovProcess.aemeasurable_edist}
If $X : T \to \Omega \to E$ is a process that satisfies the Kolmogorov condition, then for all $s,t \in T$ the function $\omega \mapsto d_E(X_s(\omega), X_t(\omega))$ is $\mathbb{P}$-a.e. measurable.
\end{lemma}

\begin{proof}\leanok

\end{proof}

\paragraph{Distance bounds}

\begin{lemma}\label{lem:integral_sup_rpow_dist_le_card_mul_rpow}
  \uses{def:IsKolmogorovProcess}
  \leanok
  \lean{ProbabilityTheory.lintegral_sup_rpow_edist_le_card_mul_rpow}
Let $X : T \to \Omega \to E$ be a process that satisfies the Kolmogorov condition for exponents $(p,q)$ with constant $M$.
Let $\varepsilon > 0$ and $C \subseteq T^2$ be a finite set such that for all $(s, t) \in C$, $d_T(s, t) \le \varepsilon$.
Then
\begin{align*}
  \mathbb{E}\left[\sup_{(s,t) \in C} d_E(X_s, X_t)^p \right]
  &\le \vert C \vert M \varepsilon^q
  \: .
\end{align*}
\end{lemma}

\begin{proof}\leanok
  \uses{def:IsKolmogorovProcess}
\begin{align*}
  \mathbb{E}\left[\sup_{(s,t) \in C} d_E(X_s, X_t)^p \right]
  &\le \mathbb{E}\left[\sum_{(s,t) \in C} d_E(X_s, X_t)^p \right]
  \\
  &\le M \sum_{(s,t) \in C} d_T(s, t)^q
  \\
  &\le \vert C \vert M \varepsilon^q
  \: .
\end{align*}
\end{proof}


\begin{lemma}\label{lem:integral_sup_rpow_dist_of_dist_le}
  \uses{def:IsKolmogorovProcess}
  \leanok
  \lean{ProbabilityTheory.lintegral_sup_rpow_edist_le_card_mul_rpow_of_dist_le}
Let $X : T \to \Omega \to E$ be a process that satisfies the Kolmogorov condition for exponents $(p,q)$ with constant $M$.
Let $J \subseteq T$ be finite, $a, c \in \mathbb R_+$ with $a \ge 1$ and $n \in \{1, 2, ...\}$ such that $|J| \le a^n$.
Then
\begin{align*}
  \mathbb{E} \left[ \sup_{s, t \in J; d_T(s, t) \le c} d_E(X_s, X_t)^p \right]
  &\le 2^p a |J| M (cn)^q
  \: .
\end{align*}
\end{lemma}

\begin{proof}\leanok
  \uses{lem:pair_reduction, lem:integral_sup_rpow_dist_le_card_mul_rpow}
By Lemma~\ref{lem:pair_reduction}, there exists $K \subseteq J^2$ such that
\begin{align*}
  |K|
  & \le a |J|
  \:, \\
  \forall (s,t) \in K,
  & \ d_T(s,t) \le c n
  \:, \\
  \sup_{s,t\in J, d_T(s,t) \le c} d_E(X_s, X_t)
  & \le 2 \sup_{(s,t) \in K} d_E(X_s, X_t)
  \: .
\end{align*}
Hence for such a set $K$,
\begin{align*}
  \mathbb{E} \left[ \sup_{s, t \in J; d_T(s, t) \le c} d_E(X_s, X_t)^p \right]
  &\le 2^p \mathbb{E} \left[ \sup_{(s, t) \in K} d_E(X_s, X_t)^p \right]
  \: .
\end{align*}
Then by Lemma~\ref{lem:integral_sup_rpow_dist_le_card_mul_rpow},
\begin{align*}
  \mathbb{E} \left[ \sup_{(s, t) \in K} d_E(X_s, X_t)^p \right]
  &\le |K| M (cn)^q
  \le a |J| M (cn)^q
  \: .
\end{align*}
\end{proof}


\subsection{Bound for a set of points that are close together}

For a finite index set $T$, we want to obtain a bound on
\begin{align*}
  \mathbb{E}\left[ \sup_{s, t \in T; d_T(s, t) \le \delta} d_E(X_s, X_t)^p \right] \: .
\end{align*}
Note the condition that the supremum is taken over pairs $(s, t)$ such that $d_T(s, t) \le \delta$.

We consider covers of $T$ at different scales. $C_n$ is a finite $\varepsilon_n$-cover of $T$ with $\varepsilon_n = \varepsilon_0 2^{-n}$.
$T$ is equal to $C_k$ for some $k$ large enough, so the supremum over $T$ is a supremum at that scale $k$.
We will change scale to some $m \le k$ that depends on the distance bound $\delta$ ($m$ is of order $\log_2 \delta$) and consider the supremum over $C_m$ (plus a term due to the scale change).
Then for the supremum over a set in $C_m$, we use the reduction in the number of pairs of Lemma~\ref{lem:pair_reduction}.

\begin{lemma}\label{lem:scale_change}
  \uses{def:chainingSequence}
  \leanok
  \lean{scale_change}
Let $X : T \to E$.
Let $(\varepsilon_n)_{n \in \mathbb{N}}$ be a sequence of positive numbers, $C_n$ a finite $\varepsilon_n$-cover of $J \subseteq T$ with $C_n \subseteq J$.
For $m \le k$,
\begin{align*}
  \sup_{s, t \in C_k; d_T(s, t) \le \delta} d_E(X_s, X_t)
  &\le \sup_{s, t \in C_k; d_T(s, t) \le \delta} d_E(X_{\bar{s}_m}, X_{\bar{t}_m})
    + 2 \sup_{s \in C_k} d_E(X_s, X_{\bar{s}_m})
  \: .
\end{align*}
\end{lemma}

\begin{proof}\leanok
By the triangle inequality,
\begin{align*}
  d_E(X_s, X_t)
  &\le d_E(X_s, X_{\bar{s}_m}) + d(X_{\bar{s}_m}, X_{\bar{t}_m}) + d_E(X_{\bar{t}_m}, X_t)
  \: .
\end{align*}
\end{proof}


\begin{corollary}\label{cor:scale_change_rpow}
  \uses{def:chainingSequence}
  \leanok
  \lean{scale_change_rpow}
Let $X : T \to E$.
Let $(\varepsilon_n)_{n \in \mathbb{N}}$ be a sequence of positive numbers, $C_n$ a finite $\varepsilon_n$-cover of $J \subseteq T$ with $C_n \subseteq J$.
For $m \le k$,
\begin{align*}
  \sup_{s, t \in C_k; d_T(s, t) \le \delta} d_E(X_s, X_t)^p
  &\le 2^p \sup_{s, t \in C_k; d_T(s, t) \le \delta} d_E(X_{\bar{s}_m}, X_{\bar{t}_m})^p
    + 4^p \sup_{s \in C_k} d_E(X_s, X_{\bar{s}_m})^p
  \: .
\end{align*}
\end{corollary}

\begin{proof}\leanok
  \uses{lem:scale_change}
This is Lemma~\ref{lem:scale_change}, together with the fact that for $a, b \ge 0$,
\begin{align*}
  (a + b)^p \le (2\max(a,b))^p = 2^p \max(a^p,b^p) \le 2^p (a^p + b^p)
  \: .
\end{align*}
\end{proof}



\subsubsection{First term}


\begin{lemma}\label{lem:integral_sup_rpow_dist_cover_of_dist_le}
  \uses{def:IsKolmogorovProcess}
  \leanok
  \lean{ProbabilityTheory.lintegral_sup_rpow_edist_cover_of_dist_le}
Let $X : T \to \Omega \to E$ be a process that satisfies the Kolmogorov condition for exponents $(p,q)$ with constant $M$.
Let $C$ be a finite $\varepsilon$-cover of $J \subseteq T$ with $C \subseteq J$, with minimal cardinal.
Then for $c \ge 0$,
\begin{align*}
  \mathbb{E} \left[ \sup_{s, t \in C; d_T(s, t) \le c} d_E(X_s, X_t)^p \right]
  &\le 2^{p+1} M \left(2 c \log_2 N^{int}_{\varepsilon}(J) \right)^q  N^{int}_{\varepsilon}(J)
  \: .
\end{align*}
Note the logarithm has base $2$.
\end{lemma}

\begin{proof}\leanok
  \uses{lem:integral_sup_rpow_dist_of_dist_le}
Let $\bar{r} = 1 + \log_2 N^{int}_{\varepsilon}(J)$. Then
\begin{align*}
  \vert C \vert
  = N^{int}_{\varepsilon}(J)
  \le 2^{\bar{r}}
  \: .
\end{align*}
By Lemma~\ref{lem:integral_sup_rpow_dist_of_dist_le} with $J = C$, $a = 2$, $c = c$, $n = \bar{r}$,
\begin{align*}
  \mathbb{E} \left[ \sup_{s, t \in C; d_T(s, t) \le c} d_E(X_s, X_t)^p \right]
  &\le 2^{p+1} |C| M (c \bar{r})^q
  = 2^{p+1} M (c \bar{r})^q N^{int}_{\varepsilon}(J)
  \: .
\end{align*}

Suppose $N^{int}_{\varepsilon}(J) \ge 2$ (if it equals one the result is trivial).
Then $\bar{r} \le 2 \log_2 N^{int}_{\varepsilon}(J)$.
\begin{align*}
  \mathbb{E} \left[ \sup_{s, t \in C; d_T(s, t) \le c} d_E(X_s, X_t)^p \right]
  &\le 2^{p+1} M \left(2 c \log_2 N^{int}_{\varepsilon}(J) \right)^q  N^{int}_{\varepsilon}(J)
  \: .
\end{align*}
\end{proof}

\begin{lemma}\label{lem:integral_sup_rpow_dist_cover_rescale}
  \uses{def:IsKolmogorovProcess, def:chainingSequence}
  \leanok
  \lean{ProbabilityTheory.lintegral_sup_rpow_edist_cover_rescale}
Let $X : T \to \Omega \to E$ be a process that satisfies the Kolmogorov condition for exponents $(p,q)$ with constant $M$.
For all $n \in \mathbb{N}$, let $C_n$ a finite $\varepsilon_n$-cover of $J \subseteq T$ with $C_n \subseteq J$ for $\varepsilon_n = \varepsilon_0 2^{-n}$, with minimal cardinal.
Suppose $\varepsilon_0 < \infty$, let $\delta \in (0, 4 \varepsilon_0]$ and let $m$ be a natural number such that $\varepsilon_0 2^{-m} \le \delta$ and $\delta \le \varepsilon_0 2^{-m+2}$.
Then for $k \ge m$,
\begin{align*}
  \mathbb{E} \left[ \sup_{s, t \in C_k; d_T(s, t) \le \delta} d_E(X_{\bar{s}_m}, X_{\bar{t}_m})^p \right]
  &\le 2^{p+1} M \left(16 \delta \log_2 N^{int}_{\delta/4}(J) \right)^q  N^{int}_{\delta/4}(J)
  \: .
\end{align*}
\end{lemma}

\begin{proof}\leanok
  \uses{lem:integral_sup_rpow_dist_cover_of_dist_le, cor:dist_chainingSequence_pow_two_le}
By definition of $m$, $\delta \le \varepsilon_0 2^{-m+2}$.
For $s, t \in C_k$ with $d_T(s, t) \le \delta$, $d_T(\bar{s}_m, \bar{t}_m) \le \delta + \varepsilon_0 2^{-m+2} \le \varepsilon_0 2^{-m+3}$ (Corollary~\ref{cor:dist_chainingSequence_pow_two_le}).
It thus suffices to get a bound on $\mathbb{E} \left[ \sup_{s, t \in C_m; d_T(s, t) \le \varepsilon_0 2^{-m+3}} d_E(X_s, X_t)^p \right]$.

We can apply Lemma~\ref{lem:integral_sup_rpow_dist_cover_of_dist_le} with $\varepsilon = \varepsilon_m$, $c = \varepsilon_0 2^{-m+3}$. We obtain
\begin{align*}
  \mathbb{E} \left[ \sup_{s, t \in C_m; d_T(s, t) \le \varepsilon_0 2^{-m+3}} d_E(X_s, X_t)^p \right]
  &\le 2^{p+1} M \left(16 \varepsilon_0 2^{-m} \log_2 N^{int}_{\varepsilon_m}(J) \right)^q  N^{int}_{\varepsilon_m}(J)
  \: .
\end{align*}
By definition of $m$, $\varepsilon_m = \varepsilon_0 2^{-m} \ge \delta/4$,
hence $N^{int}_{\varepsilon_m}(J) \le N^{int}_{\delta / 4}(J)$.

Finally, by definition of $m$ we have $\varepsilon_0 2^{-m} \le \delta$.
\end{proof}



\subsubsection{Second term}


\begin{lemma}\label{lem:integral_sup_rpow_dist_succ}
  \uses{def:IsKolmogorovProcess}
  \leanok
  \lean{ProbabilityTheory.lintegral_sup_rpow_edist_succ}
Let $X : T \to \Omega \to E$ be a process that satisfies the Kolmogorov condition for exponents $(p,q)$ with constant $M$.
Let $(\varepsilon_n)_{n \in \mathbb{N}}$ be a sequence of positive numbers and $C_n$ a finite $\varepsilon_n$-cover of $T$ with $C_n \subseteq T$.
Then for $j < k$,
\begin{align*}
  \mathbb{E}\left[\sup_{t \in C_k} d_E(X_{\bar{t}_j}, X_{\bar{t}_{j+1}})^p \right]
  &\le \vert C_{j+1} \vert M \varepsilon_j^q
  \: .
\end{align*}
\end{lemma}

\begin{proof}\leanok
  \uses{lem:dist_chainingSequence_add_one, lem:integral_sup_rpow_dist_le_card_mul_rpow}
\begin{align*}
  \mathbb{E}\left[\sup_{t \in C_k} d_E(X_{\bar{t}_j}, X_{\bar{t}_{j+1}})^p \right]
  &\le \mathbb{E}\left[\sup_{u \in C_{j+1}} d_E(X_{\bar{u}_j}, X_{u})^p \right]
  \: .
\end{align*}
We then apply Lemma~\ref{lem:integral_sup_rpow_dist_le_card_mul_rpow} to the set $C = \{(\bar{u}_j, u) \mid u \in C_{j+1}\}$, which satisfies the condition $d_T(\bar{u}_j, u) \le \varepsilon_j$ and has cardinal $\vert C_{j+1} \vert$.
\end{proof}



\paragraph{Case $p \ge 1$}


\begin{lemma}\label{lem:integral_sup_dist_le_sum_rpow}
  \uses{def:chainingSequence}
  \leanok
  \lean{ProbabilityTheory.lintegral_sup_rpow_edist_le_sum_rpow}
Let $X : T \to \Omega \to E$ be a stochastic process.
Let $(\varepsilon_n)_{n \in \mathbb{N}}$ be a sequence of positive numbers and $C_n$ a finite $\varepsilon_n$-cover of $T$ with $C_n \subseteq T$.
For $p \ge 1$ and $m \le k$,
\begin{align*}
  \mathbb{E}\left[\sup_{t \in C_k} d_E(X_t, X_{\bar{t}_m})^p \right]
  &\le \left(\sum_{i=m}^{k-1} \left( \mathbb{E}\left[\sup_{t \in C_k} d_E(X_{\bar{t}_i}, X_{\bar{t}_{i+1}})^p\right] \right)^{1/p}\right)^p
  \: .
\end{align*}
\end{lemma}

\begin{proof}\leanok
By the triangle inequality,
\begin{align*}
  \sup_{t \in C_k} d_E(X_t, X_{\bar{t}_m})^p
  &\le \sup_{t \in C_k} \left( \sum_{i=m}^{k-1} d_E(X_{\bar{t}_i}, X_{\bar{t}_{i+1}}) \right)^p
  \\
  &\le \left( \sum_{i=m}^{k-1} \sup_{t \in C_k} d_E(X_{\bar{t}_i}, X_{\bar{t}_{i+1}}) \right)^p
  \: .
\end{align*}
We thus have
\begin{align*}
  \left(\mathbb{E} \left[\sup_{t \in C_k} d_E(X_t, X_{\bar{t}_m})^p \right]\right)^{1/p}
  &\le \left(\mathbb{E} \left[\left( \sum_{i=m}^{k-1} \sup_{t \in C_k} d_E(X_{\bar{t}_i}, X_{\bar{t}_{i+1}}) \right)^p\right]\right)^{1/p}
  \: .
\end{align*}
And then, by Minkowski's inequality, since $p \ge 1$,
\begin{align*}
  \left(\mathbb{E} \left[\sup_{t \in C_k} d_E(X_t, X_{\bar{t}_m})^p \right]\right)^{1/p}
  &\le \sum_{i=m}^{k-1} \left( \mathbb{E}\left[\sup_{t \in C_k} d_E(X_{\bar{t}_i}, X_{\bar{t}_{i+1}})^p \right] \right)^{1/p}
  \: .
\end{align*}
Finally, we raise to the $p$-th power to obtain the result.
\end{proof}


\begin{lemma}\label{lem:integral_sup_rpow_dist_le_sum}
  \uses{def:IsKolmogorovProcess}
  \leanok
  \lean{ProbabilityTheory.lintegral_sup_rpow_edist_le_sum}
Let $X : T \to \Omega \to E$ be a process that satisfies the Kolmogorov condition for exponents $(p,q)$ with constant $M$.
Let $(\varepsilon_n)_{n \in \mathbb{N}}$ be a sequence of positive numbers and $C_n$ a finite $\varepsilon_n$-cover of $T$ with $C_n \subseteq T$.
Then for $p \ge 1$ and $m \le k$,
\begin{align*}
  \mathbb{E} \left[\sup_{t \in C_k} d_E(X_t, X_{\bar{t}_m})^p \right]
  &\le M \left( \sum_{j=m}^{k-1} \vert C_{j+1} \vert^{1/p} \varepsilon_j^{q/p} \right)^p
  \: .
\end{align*}
\end{lemma}

\begin{proof}\leanok
  \uses{lem:integral_sup_rpow_dist_succ, lem:integral_sup_dist_le_sum_rpow}
Put together Lemma~\ref{lem:integral_sup_rpow_dist_succ} and Lemma~\ref{lem:integral_sup_dist_le_sum_rpow}.
\end{proof}


\begin{lemma}\label{lem:integral_sup_rpow_dist_le_of_minimal_cover}
  \uses{def:IsKolmogorovProcess, def:HasBoundedInternalCoveringNumber}
  \leanok
  \lean{ProbabilityTheory.lintegral_sup_rpow_edist_le_of_minimal_cover}
Let $X : T \to \Omega \to E$ be a process that satisfies the Kolmogorov condition for exponents $(p,q)$ with constant $M$.
Let $(\varepsilon_n)_{n \in \mathbb{N}}$ be a sequence of positive numbers in $(0, \mathrm{diam}(T))$ and $C_n$ a finite $\varepsilon_n$-cover of $T$ with $C_n \subseteq T$, and with minimal cardinality.
Suppose that $T$ has bounded internal covering number with constant $c_1>0$ and exponent $d > 0$.
Then for $p \ge 1$ and $m \le k$,
\begin{align*}
  \mathbb{E} \left[\sup_{t \in C_k} d_E(X_t, X_{\bar{t}_m})^p \right]
  &\le M c_1 \left( \sum_{j=m}^{k-1} \varepsilon_{j+1}^{-d/p} \varepsilon_j^{q/p} \right)^p
  \: .
\end{align*}
\end{lemma}

\begin{proof}\leanok
  \uses{lem:integral_sup_rpow_dist_le_sum, def:HasBoundedInternalCoveringNumber}
By Lemma~\ref{lem:integral_sup_rpow_dist_le_sum}, we have
\begin{align*}
  \mathbb{E} \left[\sup_{t \in C_k} d_E(X_t, X_{\bar{t}_m})^p \right]
  &\le M \left( \sum_{j=m}^{k-1} \vert C_{j+1} \vert^{1/p} \varepsilon_j^{q/p} \right)^p
  \: .
\end{align*}
Then by the minimality of the cardinality of $C_n$ and the bounded internal covering number hypothesis, we have
\begin{align*}
  \vert C_{j+1} \vert
  &\le N^{int}_{\varepsilon_{j+1}}(T)
  \le c_1 \varepsilon_{j+1}^{-d}
  \: .
\end{align*}
\end{proof}


\begin{corollary}\label{cor:integral_sup_rpow_dist_le_of_minimal_cover_two}
  \uses{def:IsKolmogorovProcess, def:HasBoundedInternalCoveringNumber}
  \leanok
  \lean{ProbabilityTheory.lintegral_sup_rpow_edist_le_of_minimal_cover_two}
Under the assumptions of Lemma~\ref{lem:integral_sup_rpow_dist_le_of_minimal_cover}, for $\varepsilon_n = \varepsilon_0 2^{-n}$, then for $m \le k$,
\begin{align*}
  \mathbb{E} \left[\sup_{t \in C_k} d_E(X_t, X_{\bar{t}_m})^p \right]
  &\le 2^d M c_1 (\varepsilon_0 2^{-m + 1})^{q - d} \frac{1}{\left( 2^{(q -d)/p} - 1\right)^p}
  \: .
\end{align*}
\end{corollary}

\begin{proof}\leanok
  \uses{lem:integral_sup_rpow_dist_le_of_minimal_cover}
Applying first Lemma~\ref{lem:integral_sup_rpow_dist_le_of_minimal_cover}, we get
\begin{align*}
  \mathbb{E} \left[\sup_{t \in C_k} d_E(X_t, X_{\bar{t}_m})^p \right]
  &\le 2^d M c_1 \varepsilon_0^{q - d} \left( \sum_{j=m}^{k-1} 2^{- j(q - d)/p} \right)^p
  \\
  &= 2^d M c_1 (\varepsilon_0 2^{-m})^{q - d} \left( \sum_{j=0}^{k-m-1} 2^{- j(q - d)/p} \right)^p
  \\
  &\le 2^d M c_1 (\varepsilon_0 2^{-m})^{q - d} \left( \sum_{j=0}^{\infty} 2^{- j(q - d)/p} \right)^p
  \\
  &= 2^d M c_1 (\varepsilon_0 2^{-m})^{q - d} \frac{1}{(1 - 2^{-(q-d)/p})^p}
  \\
  &= 2^d M c_1 (\varepsilon_0 2^{-m+1})^{q - d} \frac{1}{(2^{(q-d)/p} - 1)^p}
  \: .
\end{align*}
\end{proof}



\paragraph{Case $p \le 1$}


\begin{lemma}\label{lem:integral_sup_dist_le_sum_rpow_of_le_one}
  \uses{def:chainingSequence}
  \leanok
  \lean{ProbabilityTheory.lintegral_sup_rpow_edist_le_sum_rpow_of_le_one}
Let $X : T \to \Omega \to E$ be a stochastic process.
Let $(\varepsilon_n)_{n \in \mathbb{N}}$ be a sequence of positive numbers and $C_n$ a finite $\varepsilon_n$-cover of $T$ with $C_n \subseteq T$.
For $0 < p \le 1$ and $m \le k$,
\begin{align*}
  \mathbb{E}\left[\sup_{t \in C_k} d_E(X_t, X_{\bar{t}_m})^p \right]
  &\le \sum_{i=m}^{k-1} \mathbb{E}\left[\sup_{t \in C_k} d_E(X_{\bar{t}_i}, X_{\bar{t}_{i+1}})^p\right]
  \: .
\end{align*}
\end{lemma}

\begin{proof}\leanok
For $0 < p \le 1$, the power function is sub-additive, i.e. for $a, b \ge 0$,
\begin{align*}
  (a + b)^p \le a^p + b^p
  \: .
\end{align*}
We can thus apply the triangle inequality to obtain
\begin{align*}
  \sup_{t \in C_k} d_E(X_t, X_{\bar{t}_m})^p
  &\le \sup_{t \in C_k} \left(\sum_{i=m}^{k-1} d_E(X_{\bar{t}_i}, X_{\bar{t}_{i+1}})\right)^p
  \\
  &\le \sup_{t \in C_k} \sum_{i=m}^{k-1} d_E(X_{\bar{t}_i}, X_{\bar{t}_{i+1}})^p
  \\
  &\le \sum_{i=m}^{k-1} \sup_{t \in C_k} d_E(X_{\bar{t}_i}, X_{\bar{t}_{i+1}})^p
  \: .
\end{align*}
\end{proof}


\begin{lemma}\label{lem:integral_sup_rpow_dist_le_sum_of_le_one}
  \uses{def:chainingSequence}
  \leanok
  \lean{ProbabilityTheory.lintegral_sup_rpow_edist_le_sum_of_le_one}
Let $X : T \to \Omega \to E$ be a process that satisfies the Kolmogorov condition for exponents $(p,q)$ with constant $M$.
Let $(\varepsilon_n)_{n \in \mathbb{N}}$ be a sequence of positive numbers and $C_n$ a finite $\varepsilon_n$-cover of $T$ with $C_n \subseteq T$.
For $0 < p \le 1$ and $m \le k$,
\begin{align*}
  \mathbb{E}\left[\sup_{t \in C_k} d_E(X_t, X_{\bar{t}_m})^p \right]
  &\le M \sum_{i=m}^{k-1} \vert C_{j+1} \vert \varepsilon_j^{q}
  \: .
\end{align*}
\end{lemma}

\begin{proof}\leanok
  \uses{lem:integral_sup_rpow_dist_succ, lem:integral_sup_dist_le_sum_rpow_of_le_one}
Put together Lemma~\ref{lem:integral_sup_rpow_dist_succ} and Lemma~\ref{lem:integral_sup_dist_le_sum_rpow_of_le_one}.
\end{proof}


\begin{lemma}\label{lem:integral_sup_rpow_dist_le_of_minimal_cover_of_le_one}
  \uses{def:IsKolmogorovProcess, def:HasBoundedInternalCoveringNumber}
  \leanok
  \lean{ProbabilityTheory.lintegral_sup_rpow_edist_le_of_minimal_cover_of_le_one}
Let $X : T \to \Omega \to E$ be a process that satisfies the Kolmogorov condition for exponents $(p,q)$ with constant $M$.
Let $(\varepsilon_n)_{n \in \mathbb{N}}$ be a sequence of positive numbers in $(0, \mathrm{diam}(T)]$ and $C_n$ a finite $\varepsilon_n$-cover of $T$ with $C_n \subseteq T$, and with minimal cardinality.
Suppose that $T$ has bounded internal covering number with constant $c_1>0$ and exponent $d > 0$.
Then for $p \le 1$ and $m \le k$,
\begin{align*}
  \mathbb{E} \left[\sup_{t \in C_k} d_E(X_t, X_{\bar{t}_m})^p \right]
  &\le M c_1 \sum_{j=m}^{k-1} \varepsilon_{j+1}^{-d} \varepsilon_j^{q}
  \: .
\end{align*}
\end{lemma}

\begin{proof}\leanok
  \uses{lem:integral_sup_rpow_dist_le_sum_of_le_one, def:HasBoundedInternalCoveringNumber}
By Lemma~\ref{lem:integral_sup_rpow_dist_le_sum_of_le_one}, we have
\begin{align*}
  \mathbb{E}\left[\sup_{t \in C_k} d_E(X_t, X_{\bar{t}_m})^p \right]
  &\le M \sum_{i=m}^{k-1} \vert C_{j+1} \vert \varepsilon_j^{q}
  \: .
\end{align*}
Then by the minimality of the cardinality of $C_n$ and the bounded internal covering number hypothesis, we have
\begin{align*}
  \vert C_{j+1} \vert
  &= N^{int}_{\varepsilon_{j+1}}(T)
  \le c_1 \varepsilon_{j+1}^{-d}
  \: .
\end{align*}
\end{proof}


\begin{corollary}\label{cor:integral_sup_rpow_dist_le_of_minimal_cover_two_of_le_one}
  \uses{def:IsKolmogorovProcess, def:HasBoundedInternalCoveringNumber}
  \leanok
  \lean{ProbabilityTheory.lintegral_sup_rpow_edist_le_of_minimal_cover_two_of_le_one}
Under the assumptions of Lemma~\ref{lem:integral_sup_rpow_dist_le_of_minimal_cover_of_le_one}, for $\varepsilon_n = \varepsilon_0 2^{-n}$, then for $m \le k$,
\begin{align*}
  \mathbb{E} \left[\sup_{t \in C_k} d_E(X_t, X_{\bar{t}_m})^p \right]
  &\le 2^d M c_1 (\varepsilon_0 2^{-m + 1})^{q - d} \frac{1}{\left( 2^{(q -d)} - 1\right)}
  \: .
\end{align*}
\end{corollary}

\begin{proof}\leanok
  \uses{lem:integral_sup_rpow_dist_le_of_minimal_cover_of_le_one}
Applying first Lemma~\ref{lem:integral_sup_rpow_dist_le_of_minimal_cover_of_le_one}, we get
\begin{align*}
  \mathbb{E} \left[\sup_{t \in C_k} d_E(X_t, X_{\bar{t}_m})^p \right]
  &\le 2^d M c_1 (\varepsilon_0 2^{-m})^{q-d}\sum_{j=0}^{k-m-1} 2^{- j (q - d)}
  \\
  &\le 2^d M c_1 (\varepsilon_0 2^{-m})^{q-d}\sum_{j=0}^{+\infty} 2^{- j (q - d)}
  \\
  &= 2^d M c_1 (\varepsilon_0 2^{-m})^{q-d} \frac{1}{1 - 2^{-(q - d)}}
  \\
  &= 2^d M c_1 (\varepsilon_0 2^{-m+1})^{q-d} \frac{1}{2^{(q - d)} - 1}
  \: .
\end{align*}
\end{proof}


\paragraph{Any $p>0$}


\begin{definition}\label{def:Cp}
  \leanok
  \lean{ProbabilityTheory.Cp}
\begin{align*}
  C_p = \max\left\{\frac{1}{\left( 2^{(q -d)/p} - 1\right)^p}, \frac{1}{\left( 2^{(q -d)} - 1\right)} \right\}
  \: .
\end{align*}
\end{definition}


\begin{lemma}\label{lem:second_term_bound}
  \uses{def:IsKolmogorovProcess, def:HasBoundedInternalCoveringNumber, def:Cp}
  \leanok
  \lean{ProbabilityTheory.second_term_bound}
Let $X : T \to \Omega \to E$ be a process that satisfies the Kolmogorov condition for exponents $(p,q)$ with constant $M$.
Let $C_n$ a finite $(\varepsilon_0 2^{-n})$-cover of $T$ for $\varepsilon_0 \le \mathrm{diam}(T)$ with $C_n \subseteq T$, and with minimal cardinality.
Suppose that $T$ has bounded internal covering number with constant $c_1>0$ and exponent $d > 0$.
Then for $m \le k$,
\begin{align*}
  \mathbb{E} \left[\sup_{t \in C_k} d_E(X_t, X_{\bar{t}_m})^p \right]
  &\le 2^d M c_1 (\varepsilon_0 2^{-m + 1})^{q - d} C_p
  \: .
\end{align*}
\end{lemma}

\begin{proof}\leanok
  \uses{cor:integral_sup_rpow_dist_le_of_minimal_cover_two_of_le_one,cor:integral_sup_rpow_dist_le_of_minimal_cover_two}
This is the max of the two bounds obtained $p \ge 1$ and $p \le 1$.
\end{proof}



\subsubsection{Putting it all together}


\begin{lemma}\label{lem:lintegral_sup_cover_eq_of_lt_iInf_dist}
  \uses{def:IsKolmogorovProcess, def:IsCover}
  \leanok
  \lean{ProbabilityTheory.lintegral_sup_cover_eq_of_lt_iInf_dist}
Let $X : T \to \Omega \to E$ be a process that satisfies the Kolmogorov condition for exponents $(p,q)$ with constant $M$ and let $J$ be a finite subset of $T$.
Let $C$ be an $\varepsilon$-cover of $J$ with $C \subseteq J$.
If $\varepsilon < \inf_{s, t \in J; d_T(s, t)>0} d_T(s, t)$ then
\begin{align*}
  \mathbb{E}\left[ \sup_{s, t \in C; d_T(s, t) \le \delta} d_E(X_s, X_t)^p \right]
  &= \mathbb{E}\left[ \sup_{s, t \in J; d_T(s, t) \le \delta} d_E(X_s, X_t)^p \right]
\end{align*}
\end{lemma}

\begin{proof}\leanok
  \uses{lem:IsKolmogorovProcess.edist_eq_zero}
First, remark that $C$ is actually a $0$-cover of $J$.
For $s, t \in J$, let $s', t' \in C$ be such that $d_T(s, s') = 0$ and $d_T(t, t') = 0$.
Then by the triangle inequality,
\begin{align*}
  d_E(X_s, X_t)
  &\le d_E(X_s, X_{s'}) + d_E(X_{s'}, X_{t'}) + d_E(X_t, X_{t'})
\end{align*}
and by Lemma~\ref{lem:IsKolmogorovProcess.edist_eq_zero}, we have $d_E(X_s, X_{s'}) = 0$ and $d_E(X_t, X_{t'}) = 0$ almost surely, hence $d_E(X_s, X_t) \le d_E(X_{s'}, X_{t'})$.
Since $J$ is finite, almost surely we have that inequality for all pairs $(s, t) \in J$ and their corresponding $(s', t') \in C$.
Note that $d_T(s', t') = d_T(s, t)$, hence $d_T(s, t) \le \delta$ is equivalent to $d_T(s', t') \le \delta$.
We obtain
\begin{align*}
  \mathbb{E}\left[ \sup_{s, t \in J; d_T(s, t) \le \delta} d_E(X_s, X_t)^p \right]
  &\le \mathbb{E}\left[ \sup_{s, t \in J; d_T(s, t) \le \delta} d_E(X_{s'}, X_{t'})^p \right]
  \\
  &= \mathbb{E}\left[ \sup_{s, t \in J; d_T(s', t') \le \delta} d_E(X_{s'}, X_{t'})^p \right]
  \\
  &\le \mathbb{E}\left[ \sup_{s, t \in C; d_T(s, t) \le \delta} d_E(X_s, X_t)^p \right]
  \: .
\end{align*}
The reverse inequality holds because $C$ is a subset of $J$.
\end{proof}


\begin{theorem}\label{thm:finite_set_bound_of_dist_le_of_diam_le}
  \uses{def:IsKolmogorovProcess, def:HasBoundedInternalCoveringNumber, def:Cp}
  \leanok
  \lean{ProbabilityTheory.finite_set_bound_of_edist_le_of_diam_le}
Suppose that $T$ is a finite set with bounded internal covering number with constant $c_1>0$ and exponent $d > 0$.
Let $X : T \to \Omega \to E$ be a process that satisfies the Kolmogorov condition for exponents $(p,q)$ with constant $M$, with $q > d$ and $p > 0$.
For all $\delta \ge 4\mathrm{diam}(T)$,
\begin{align*}
  \mathbb{E}\left[ \sup_{s, t \in T; d_T(s, t) \le \delta} d_E(X_s, X_t)^p \right]
  \le 4^p 2^q M c_1 \delta^{q - d} C_p
  \: .
\end{align*}
\end{theorem}

\begin{proof}\leanok
  \uses{lem:second_term_bound, cor:scale_change_rpow, lem:lintegral_sup_cover_eq_of_lt_iInf_dist}
Let $\varepsilon_0 = \mathrm{diam}(T)$.
For all $n \in \mathbb{N}$, let $C_n$ a finite $\varepsilon_n$-cover of $T$ with $C_n \subseteq T$ for $\varepsilon_n = \varepsilon_0 2^{-n}$, with minimal cardinal.

Let $k$ be a natural number such that $\varepsilon_0 2^{-k} < \inf_{s, t \in T; d_T(s,t)>0} d_T(s, t)$, which exists since $T$ is finite.
By Lemma~\ref{lem:lintegral_sup_cover_eq_of_lt_iInf_dist}, the supremum over $T$ can be replaced by a supremum over $C_k$.

By Corollary~\ref{cor:scale_change_rpow},
\begin{align*}
  &\mathbb{E}\left[ \sup_{s, t \in C_k; d_T(s, t) \le \delta} d_E(X_s, X_t)^p \right]
  \\
  &\le 2^p \mathbb{E}\left[ \sup_{s, t \in C_k; d_T(s, t) \le \delta} d_E(X_{\bar{s}_0}, X_{\bar{t}_0})^p \right]
    + 4^p \mathbb{E}\left[ \sup_{s \in C_k} d_E(X_s, X_{\bar{s}_0})^p \right]
  \: .
\end{align*}

Since $\varepsilon_0 = \mathrm{diam}(T)$, $C_0$ is a singleton and $d_E(X_{\bar{s}_0}, X_{\bar{t}_0}) = 0$ for all $s, t$.
We thus have
\begin{align*}
  \mathbb{E} \left[ \sup_{s, t \in C_k; d_T(s, t) \le \delta} d_E(X_{\bar{s}_0}, X_{\bar{t}_0})^p \right]
  &= 0
  \: .
\end{align*}

By Lemma~\ref{lem:second_term_bound},
\begin{align*}
  \mathbb{E} \left[\sup_{t \in C_k} d_E(X_t, X_{\bar{t}_0})^p \right]
  &\le 2^q M c_1 \varepsilon_0^{q - d} C_p
  \le 2^q M c_1 \delta^{q - d} C_p
  \: .
\end{align*}
\end{proof}


\begin{theorem}\label{thm:finite_set_bound_of_dist_le_of_le_diam}
  \uses{def:IsKolmogorovProcess, def:HasBoundedInternalCoveringNumber, def:Cp}
  \leanok
  \lean{ProbabilityTheory.finite_set_bound_of_edist_le_of_le_diam}
Suppose that $T$ is a finite set with bounded internal covering number with constant $c_1>0$ and exponent $d > 0$.
Let $X : T \to \Omega \to E$ be a process that satisfies the Kolmogorov condition for exponents $(p,q)$ with constant $M$, with $q > d$ and $p > 0$.
For all $\delta \in (0, 4\mathrm{diam}(T)]$,
\begin{align*}
  &\mathbb{E}\left[ \sup_{s, t \in T; d_T(s, t) \le \delta} d_E(X_s, X_t)^p \right]
  \\
  &\le 2^{2p+4q+1} M \delta^{q-d} \left(\delta^d \left(\log_2 N^{int}_{\delta/4}(T) \right)^q  N^{int}_{\delta/4}(T)
    + c_1 C_p\right)
  \: .
\end{align*}
\end{theorem}

\begin{proof}\leanok
  \uses{lem:second_term_bound, lem:integral_sup_rpow_dist_cover_rescale, cor:scale_change_rpow, lem:lintegral_sup_cover_eq_of_lt_iInf_dist, lem:IsKolmogorovProcess.lintegral_sup_rpow_edist_eq_zero}
Let $\varepsilon_0 = \mathrm{diam}(T)$.
For all $n \in \mathbb{N}$, let $C_n$ a finite $\varepsilon_n$-cover of $T$ with $C_n \subseteq T$ for $\varepsilon_n = \varepsilon_0 2^{-n}$, with minimal cardinal.

Let $k$ be a natural number such that $\varepsilon_0 2^{-k} < \inf_{s, t \in T; d_T(s,t)>0} d_T(s, t)$, which exists since $T$ is finite.
If $\delta \le \varepsilon_0 2^{-k}$, then $\{(s, t) \in C_k; d_T(s, t) \le \delta\} = \{(s, t) \mid s,t \in C_k, d_T(s,t) = 0\}$ and the inequality holds trivially (by Lemma~\ref{lem:IsKolmogorovProcess.lintegral_sup_rpow_edist_eq_zero}).
We can thus assume $\delta > \varepsilon_0 2^{-k}$.

By Lemma~\ref{lem:lintegral_sup_cover_eq_of_lt_iInf_dist}, the supremum over $T$ can be replaced by a supremum over $C_k$.

By Corollary~\ref{cor:scale_change_rpow}, for any $m \le k$,
\begin{align*}
  &\mathbb{E}\left[ \sup_{s, t \in C_k; d_T(s, t) \le \delta} d_E(X_s, X_t)^p \right]
  \\
  &\le 2^p \mathbb{E}\left[ \sup_{s, t \in C_k; d_T(s, t) \le \delta} d_E(X_{\bar{s}_m}, X_{\bar{t}_m})^p \right]
    + 4^p \mathbb{E}\left[ \sup_{s \in C_k} d_E(X_s, X_{\bar{s}_m})^p \right]
  \: .
\end{align*}

\emph{First term}

We have $\delta \le 4\varepsilon_0$ by assumption.
Let $n_2 = \lfloor \log_2(4\varepsilon_0/\delta) \rfloor$ and $m = \min\{n_2, k\}$.
If $m = n_2$ then $\varepsilon_0 2^{-m} = \varepsilon_0 2^{-n_2} < \delta/2$.
Otherwise, $m = k$ and $\varepsilon_0 2^{-m} = \varepsilon_0 2^{-k} < \delta$ as argued at the start of the proof.
We thus get $\varepsilon_0 2^{-m} \le \delta$.
We can also verify that $\delta \le \varepsilon_0 2^{-n_2+2} \le \varepsilon_0 2^{-m+2}$.
By Lemma~\ref{lem:integral_sup_rpow_dist_cover_rescale},
\begin{align*}
  \mathbb{E} \left[ \sup_{s, t \in C_k; d_T(s, t) \le \delta} d_E(X_{\bar{s}_m}, X_{\bar{t}_m})^p \right]
  &\le 2^{p+1} M \left(16 \delta \log_2 N^{int}_{\delta/4}(T) \right)^q  N^{int}_{\delta/4}(T)
  \: .
\end{align*}

\emph{Second term}

By Lemma~\ref{lem:second_term_bound} and then the inequality $\varepsilon_0 2^{-m} \le \delta$,
\begin{align*}
  \mathbb{E} \left[\sup_{t \in C_k} d_E(X_t, X_{\bar{t}_m})^p \right]
  &\le 2^d M c_1 (\varepsilon_0 2^{-m+1})^{q - d} C_p
  \\
  &\le 2^q M c_1 \delta^{q - d} C_p
  \: .
\end{align*}

Putting the two terms together, we obtain
\begin{align*}
  &\mathbb{E}\left[ \sup_{s, t \in C_k; d_T(s, t) \le \delta} d_E(X_s, X_t)^p \right]
  \\
  &\le 4^p M \left(4\left(16 \delta \log_2 N^{int}_{\delta/4}(T) \right)^q  N^{int}_{\delta/4}(T)
    + 2^q c_1 \delta^{q - d} C_p\right)
  \\
  &\le 2^{2p+4q+1} M \delta^{q-d} \left(\delta^d \left(\log_2 N^{int}_{\delta/4}(T) \right)^q  N^{int}_{\delta/4}(T)
    + c_1 C_p\right)
  \: .
\end{align*}
\end{proof}


\begin{corollary}\label{cor:finite_set_bound_of_dist_le_of_le_diam_bis}
  \uses{def:IsKolmogorovProcess, def:HasBoundedInternalCoveringNumber}
  \leanok
  \lean{ProbabilityTheory.finite_set_bound_of_edist_le_of_le_diam'}
With the same assumptions and notations as in Theorem~\ref{thm:finite_set_bound_of_dist_le_of_le_diam}, for all $\delta \in (0, 4\mathrm{diam}(T)]$,
\begin{align*}
  \mathbb{E}\left[ \sup_{s, t \in T; d_T(s, t) \le \delta} d_E(X_s, X_t)^p \right]
  &\le 2^{2p+4q+1} M c_1 \delta^{q-d} \left(4^d \left(\log_2 \left(c_1 \delta^{-d} 4^d \right) \right)^q
    + C_p\right)
  \: .
\end{align*}
\end{corollary}

\begin{proof}\leanok
  \uses{thm:finite_set_bound_of_dist_le_of_le_diam}
We apply Theorem~\ref{thm:finite_set_bound_of_dist_le_of_le_diam} and then remark that for $\delta \le 4\mathrm{diam}(T)$, we can use the bounded internal covering number hypothesis to bound $N^{int}_{\delta/4}(T)$~:
\begin{align*}
  N^{int}_{\delta/4}(T) \le c_1 \left(\frac{\delta}{4}\right)^{-d} \: .
\end{align*}
\end{proof}


\begin{corollary}\label{cor:finite_set_bound_of_dist_le}
  \uses{def:IsKolmogorovProcess, def:HasBoundedInternalCoveringNumber}
  \leanok
  \lean{ProbabilityTheory.finite_set_bound_of_edist_le}
Suppose that $T$ is a finite set with bounded internal covering number with constant $c_1>0$ and exponent $d > 0$.
Let $X : T \to \Omega \to E$ be a process that satisfies the Kolmogorov condition for exponents $(p,q)$ with constant $M$, with $q > d$ and $p > 0$.
For all $\delta > 0$,
\begin{align*}
  \mathbb{E}\left[ \sup_{s, t \in T; d_T(s, t) \le \delta} d_E(X_s, X_t)^p \right]
  &\le 2^{2p+4q+1} M c_1 \delta^{q-d} \left(4^d \left(\max\left\{0, \log_2 \left(c_1 \delta^{-d} 4^d\right) \right\} \right)^q
    + C_p\right)
  \: .
\end{align*}
\end{corollary}


\begin{proof}\leanok
  \uses{cor:finite_set_bound_of_dist_le_of_le_diam_bis, thm:finite_set_bound_of_dist_le_of_diam_le}
We combine Corollary~\ref{cor:finite_set_bound_of_dist_le_of_le_diam_bis} and Theorem~\ref{thm:finite_set_bound_of_dist_le_of_diam_le}.
\end{proof}




\section{Kolmogorov-Chentsov Theorem}


\subsection{Sets with bounded internal covering number}

TODO: change the proofs here to avoid $s \ne t$ and use instead properties of processes satisfying the Kolmogorov condition for exponents $(p,q)$.

\begin{lemma}\label{lem:integral_div_dist_le_sum_integral_dist_le}
  \leanok
  \lean{ProbabilityTheory.lintegral_div_edist_le_sum_integral_edist_le}
Let $J \subseteq T$ be a finite set and suppose that $T$ has finite diameter.
For $k \in \mathbb{N}$, let $\eta_k = 2^{-k}(\mathrm{diam}(T) + 1)$.
For $X : T \to \Omega \to E$ a stochastic process and $\beta \in(0, (q - d)/p)$,
\begin{align*}
  \mathbb{E}\left[ \sup_{s, t \in J;\: s \ne t} \frac{d_E(X_s, X_t)^p}{d_T(s, t)^{\beta p}} \right]
  &\le \sum_{k=0}^\infty 2^{k \beta p} \mathbb{E}\left[ \sup_{s, t \in J;\: s \ne t, \: d_T(s, t) \le 2 \eta_k} d_E(X_s, X_t)^p \right]
  \: .
\end{align*}
\end{lemma}

\begin{proof}\leanok
We introduce for each $k \in \mathbb{N}$ the set of pairs $(s, t)$ such that $\eta_k < d_T(s, t) \le 2 \eta_k$.
Note that $\eta_k \ge 2^{-k}$.
\begin{align*}
  \mathbb{E}\left[ \sup_{s, t \in J;\: s \ne t} \frac{d_E(X_s, X_t)^p}{d_T(s, t)^{\beta p}} \right]
  &\le \sum_{k=0}^\infty \mathbb{E}\left[ \sup_{s, t \in J;\: s \ne t, \: \eta_k < d_T(s, t) \le 2 \eta_k} \frac{d_E(X_s, X_t)^p}{d_T(s, t)^{\beta p}} \right]
  \\
  &\le \sum_{k=0}^\infty \eta_k^{-\beta p} \mathbb{E}\left[ \sup_{s, t \in J;\: s \ne t, \: d_T(s, t) \le 2 \eta_k} d_E(X_s, X_t)^p \right]
  \\
  &\le \sum_{k=0}^\infty 2^{k \beta p} \mathbb{E}\left[ \sup_{s, t \in J;\: s \ne t, \: d_T(s, t) \le 2 \eta_k} d_E(X_s, X_t)^p \right]
  \: .
\end{align*}
\end{proof}


\begin{definition}\label{def:L}
  \uses{def:Cp}
  \leanok
  \lean{ProbabilityTheory.constL}
We introduce the constant
\begin{align*}
  L(T, c_1, d, p, q, \beta)
  &= 2^{2p+5q+1} c_1 (\mathrm{diam}(T)+1)^{q-d}
  \\&\quad \times \sum_{k=0}^\infty 2^{k (\beta p - (q-d))}\left(4^d \left(\max\left\{0, \log_2(c_1) + (k + 2)d \right\}\right)^q
    + C_p\right)
  \: .
\end{align*}
\end{definition}


\begin{lemma}\label{lem:L_lt_top}
  \uses{def:L}
  \leanok
  \lean{ProbabilityTheory.constL_lt_top}
For $\mathrm{diam}(T) < \infty$, $p> 0$, $q > d > 0$ and $\beta \in (0, (q-d)/p)$, the constant $L(T, c_1, d, p, q, \beta)$ is finite.
\end{lemma}

\begin{proof}
\leanok
Let $a_k = 2^{2p+5q+1} M c_1 (\mathrm{diam}(T)+1)^{q-d} 2^{k (\beta p - (q-d))} \left(4^d \left(\max\left\{0, \log_2(c_1) + (k + 2)d \right\}\right)^q
    + C_p\right)$.
Then $L(T, c_1, d, p, q, \beta) = \sum_{k=0}^\infty a_k$.
To show that the sum is finite, we can use the ratio test.
\begin{align*}
  \frac{\vert a_{k+1} \vert}{\vert a_k \vert}
  &= 2^{\beta p - (q - d)}
    \frac{\left(4^d \left(\max\left\{0, \log_2(c_1) + (k + 3)d \right\}\right)^q + C_p\right)}
    {\left(4^d \left(\max\left\{0, \log_2(c_1) + (k + 2)d \right\}\right)^q + C_p\right)}
\end{align*}
The limit at infinity of that ratio is $2^{\beta p - (q - d)} < 1$, hence the series $\sum_{k=0}^\infty a_k$ converges.
\end{proof}


\begin{lemma}\label{lem:finite_set_bound}
  \uses{def:IsKolmogorovProcess, def:HasBoundedInternalCoveringNumber, def:L}
  \leanok
  \lean{ProbabilityTheory.finite_kolmogorov_chentsov}
Suppose that $J \subseteq T$ is a finite set and that $T$ has bounded internal covering number with constant $c_1>0$ and exponent $d > 0$.
Let $X : T \to \Omega \to E$ be a process that satisfies the Kolmogorov condition for exponents $(p,q)$ with constant $M$, with $q > d$ and $p > 0$.
Let $\beta \in(0, (q - d)/p)$.
Then
\begin{align*}
  \mathbb{E}\left[ \sup_{s, t \in J;\: s \ne t} \frac{d_E(X_s, X_t)^p}{d_T(s, t)^{\beta p}} \right]
  \le M L(T, c_1, d, p, q, \beta)
  \: .
\end{align*}
\end{lemma}

\begin{proof}
  \leanok
  \uses{cor:finite_set_bound_of_dist_le, lem:integral_div_dist_le_sum_integral_dist_le, lem:hasBoundedInternalCoveringNumber_subset}

Since $J \subseteq T$, $J$ has bounded internal covering number with constant $2^d c_1$ and exponent $d$ (Lemma~\ref{lem:hasBoundedInternalCoveringNumber_subset}).

Let $\eta_k = 2^{-k}(\mathrm{diam}(T) + 1)$ for $k \in \mathbb{N}$.
By Lemma~\ref{lem:integral_div_dist_le_sum_integral_dist_le}, we have
\begin{align*}
  \mathbb{E}\left[ \sup_{s, t \in J;\: s \ne t} \frac{d_E(X_s, X_t)^p}{d_T(s, t)^{\beta p}} \right]
  &\le \sum_{k=0}^\infty 2^{k \beta p} \mathbb{E}\left[ \sup_{s, t \in J;\: s \ne t, \: d_T(s, t) \le 2 \eta_k} d_E(X_s, X_t)^p \right]
  \: .
\end{align*}
We apply Corollary~\ref{cor:finite_set_bound_of_dist_le} to bound each expectation in the sum.
\begin{align*}
  &\mathbb{E}\left[ \sup_{s, t \in J;\: s \ne t, \: d_T(s, t) \le 2 \eta_k} d_E(X_s, X_t)^p \right]
  \\
  &\le 2^{2p+4q+1} M 2^d c_1 (2 \eta_k)^{q-d} \left(4^d \left(\max\left\{0, \log_2 \left(2^d c_1 (2 \eta_k)^{-d} 4^d \right) \right\} \right)^q
    + C_p\right)
  \\
  &\le 2^{2p+5q+1} M c_1 (\mathrm{diam}(T)+1)^{q-d} 2^{-k(q-d)} \left(4^d \left(\max\left\{0, \log_2 \left(c_1 2^{(k + 2)d} \right) \right\} \right)^q
    + C_p\right)
  \\
  &= 2^{2p+5q+1} M c_1 (\mathrm{diam}(T)+1)^{q-d} 2^{-k(q-d)} \left(4^d \left(\max\left\{0, \log_2(c_1) + (k + 2)d \right\} \right)^q
    + C_p\right)
  \: .
\end{align*}
The sum is then less than $M$ times $L(T, c_1, d, p, q, \beta)$.
\end{proof}


\begin{theorem}\label{thm:countable_set_bound}
  \uses{def:IsKolmogorovProcess, def:HasBoundedInternalCoveringNumber}
  \leanok
  \lean{ProbabilityTheory.countable_kolmogorov_chentsov}
Suppose that $T$ has bounded internal covering number with constant $c_1>0$ and exponent $d > 0$.
Let $X : T \to \Omega \to E$ be a process that satisfies the Kolmogorov condition for exponents $(p,q)$ with constant $M$, with $q > d$ and $p > 0$.
Let $\beta \in(0, (q - d)/p)$.
Then for every countable subset $T' \subseteq T$ with positive diameter,
\begin{align*}
  \mathbb{E}\left[ \sup_{s, t \in T';\: s \ne t} \frac{d_E(X_s, X_t)^p}{d_T(s, t)^{\beta p}} \right]
  \le M L(T, c_1, d, p, q, \beta)
  \: .
\end{align*}
\end{theorem}

\begin{proof}\leanok
  \uses{lem:finite_set_bound}
Build a monotone sequence of finite sets $T_n \subseteq T'$, use Lemma~\ref{lem:finite_set_bound} to obtain a bound for each $T_n$ that does not depend on $T_n$, and then use monotone convergence.
\end{proof}


\begin{corollary}\label{cor:countable_set_bound_of_le}
Under the same assumptions as in Theorem~\ref{thm:countable_set_bound}, for every countable subset $T' \subseteq T$ with positive diameter, for $L(T, c_1, d, p, q, \beta) < \infty$ the same constant,
\begin{align*}
  \mathbb{E}\left[ \sup_{s, t \in T';\: d_T(s, t) \le \delta} d_E(X_s, X_t)^p \right]
  \le M L(T, c_1, d, p, q, \beta) \delta^{\beta p}
  \: .
\end{align*}
\end{corollary}

\begin{proof}
  \uses{thm:countable_set_bound}
Immediately follows from Theorem~\ref{thm:countable_set_bound}.
\end{proof}


\begin{lemma}\label{lem:holder_modification_single}
  \uses{def:IsKolmogorovProcess, def:HasBoundedInternalCoveringNumber}
  \leanok
  \lean{ProbabilityTheory.exists_modification_holder_aux}
Under the assumptions of Theorem~\ref{thm:countable_set_bound}, for $E$ a complete space and $\beta \in (0, (q - d)/p)$, there exists a modification $Y$ of $X$ (i.e., a process $Y$ with $\mathbb{P}(Y_t \ne X_t) = 0$ for all $t$) such that the paths of $Y$ are Hölder continuous of order $\beta$.
\end{lemma}

\begin{proof}\leanok
  \uses{thm:countable_set_bound, lem:L_lt_top}
Let $T'$ be a countable dense subset of $T$.
Let $A$ be the event
\begin{align*}
  \left\{\sup_{s, t \in T';\: s \ne t} \frac{d_E(X_s, X_t)^p}{d_T(s, t)^{\beta p}} < \infty \right\}
  \: .
\end{align*}
As a consequence of Theorem~\ref{thm:countable_set_bound}, we have $\mathbb{P}(A) = 1$.

On the event $A$, $(X_t)_{t \in T'}$ has Hölder continuous paths of order $\beta$.
Let $x_0 \in E$ be arbitrary and let $Y: T \to \Omega \to E$ be the process defined by
\begin{align*}
  Y_t(\omega)
  &= \begin{cases}
    \lim_{s \to t, s \in T'} X_s(\omega) & \text{if } \omega \in A \\
    x_0 & \text{otherwise}
  \end{cases}
  \: .
\end{align*}
Then $Y$ has Hölder continuous paths of order $\beta$ almost surely.

We can show that $(Y_s, Y_t)$ is $\mathbb{P}$-a.e. measurable for all $s, t \in T$.

It remains to show that $Y$ is a modification of $X$.
Let then $t \in T$ and let $(t_n)_{n \in \mathbb{N}}$ be a sequence in $T'$ that converges to $t$.
We want to show that $\mathbb{P}(Y_t \ne X_t) = 0$.
It suffices to show that $\mathbb{P}(d_E(Y_t, X_t) > 0) = 0$, which itself would follow from $\mathbb{P}(d_E(Y_t, X_t) > \varepsilon) = 0$ for all $\varepsilon > 0$.

\begin{align*}
  \mathbb{P}(d_E(Y_t, X_t) > \varepsilon)
  &\le \mathbb{P}(d_E(Y_t, X_{t_n}) + d_E(X_{t_n}, X_t) > \varepsilon)
  \\
  &\le \mathbb{P}(d_E(Y_t, X_{t_n}) > \varepsilon/2) + \mathbb{P}(d_E(X_{t_n}, X_t) > \varepsilon/2)
  \: .
\end{align*}

TODO
\end{proof}


\begin{theorem}\label{thm:holder_modification}
  \uses{def:IsKolmogorovProcess, def:HasBoundedInternalCoveringNumber}
  \leanok
  \lean{ProbabilityTheory.exists_modification_holder}
Under the assumptions of Theorem~\ref{thm:countable_set_bound}, for $E$ a complete space, there exists a modification $Y$ of $X$ (i.e., a process $Y$ with $\mathbb{P}(Y_t \ne X_t) = 0$ for all $t$) such that the paths of $Y$ are Hölder continuous of all orders $\gamma \in (0, (q - d)/p)$.
\end{theorem}

\begin{proof}\leanok
  \uses{lem:holder_modification_single, lem:indistinguishable_of_modification_of_continuous}
Let $(\beta_n)$ be an increasing sequence of numbers in $(0, (q - d)/p)$ such that $\beta_n \to (q - d)/p$.
For each $n$, let $Y^n$ be the modification of $X$ given by Lemma~\ref{lem:holder_modification_single} for $\beta = \beta_n$.
Then by Lemma~\ref{lem:indistinguishable_of_modification_of_continuous}, the processes $Y^0$ and $Y^n$ are indistinguishable for all $n$.
That is, there exists an event $A_n$ such that $\mathbb{P}(A_n) = 1$ and such that for all $\omega \in A_n$, $Y^0_t(\omega) = Y^n_t(\omega)$ for all $t \in T$.

Let $A = \bigcap_{n \in \mathbb{N}} A_n$ and let $x_0 \in E$ be arbitrary.
Then $\mathbb{P}(A) = 1$ and the process $Y(\omega) = Y^0(\omega)$ for $\omega \in A$ and $Y(\omega) = x_0$ for $\omega \notin A$ has paths that are Hölder continuous of all orders $\gamma \in (0, (q - d)/p)$.
\end{proof}



\subsection{Localized Kolmogorov-Chentsov theorem}

\begin{definition}[Cover with bounded covering numbers]\label{def:HasBoundedCoveringNumberCover}
  \uses{def:HasBoundedInternalCoveringNumber}
  \leanok
  \lean{IsCoverWithBoundedCoveringNumber}
A set $T$ is said to have a cover with bounded covering numbers if there exists a monotone sequence of totally bounded subsets $(T_n)_{n \in \mathbb{N}}$ of $T$ such that for all $n$, $T_n$ has bounded internal covering number with constant $c_n$ and exponent $d_n > 0$, and such that $T \subseteq \bigcup_{n \in \mathbb{N}} T_n$.
\end{definition}


\begin{lemma}\label{lem:hasBoundedCoveringNumberCover_nnreal}
  \uses{def:HasBoundedCoveringNumberCover}
  \leanok
  \lean{isCoverWithBoundedCoveringNumber_Ico_nnreal}
$\mathbb{R}_+$ has a cover with bounded covering numbers for the sets $T_n = [0,n)$, constants $c_n = n$ and exponents $d_n = 1$.
\end{lemma}

Note: in the Lean code, we proved this result for weaker constants: $c_n = 3n$.

\begin{proof}\leanok
  \uses{lem:hasBoundedInternalCoveringNumber_unitInterval}

\end{proof}

We say that a function is \emph{locally} Hölder continuous of order $\gamma$ if for any point $x$ in its domain there is a neighborhood of $x$ on which the function is Hölder continuous of order $\gamma$.

\begin{theorem}\label{thm:localized_holder_modification}
  \uses{def:IsKolmogorovProcess, def:HasBoundedCoveringNumberCover}
  \leanok
  \lean{ProbabilityTheory.exists_modification_holder'}
Let $T$ be a metric space with a cover $(T_n)$ with bounded covering numbers with constants $c_n$ and the same exponent $d$.
Let $X : T \to \Omega \to E$ be a process that satisfies the Kolmogorov condition with exponents $(p, q)$ with $q > d$.
Then $X$ has a modification $Y$ such that almost surely the paths of $Y$ are locally Hölder continuous of all orders $\gamma \in (0, (q - d)/p)$.
\end{theorem}

\begin{proof}\leanok
  \uses{thm:holder_modification, lem:indistinguishable_of_modification_of_continuous}
For each $n$, by Theorem~\ref{thm:holder_modification} there is a modification $Y_n$ of $X$ seen as a process on $T_n$ such that the paths of $Y_n$ are Hölder continuous of all orders $\gamma \in (0, (q - d)/p)$.
By Lemma~\ref{lem:indistinguishable_of_modification_of_continuous}, $Y_n$ and $Y_{n+1}$ are indistinguishable on $T_n$.
That is, almost surely $Y_n = Y_{n+1}$ on $T_n$.
Since there are countably many such almost sure equalities, we get that almost surely there is equality for all $n$.
Let $A$ be the event that this happens, let $x_0 \in E$ be arbitrary and define a process $Y : T \to \Omega \to E$ by
\begin{align*}
  Y(t, \omega)
  &= \begin{cases}
    Y_n(t, \omega) & \text{if } \omega \in A \: , \: t \in T_n \setminus T_{n-1} \: ,
    \\
    x_0 & \text{if } \omega \notin A \: .
  \end{cases}
\end{align*}
Then $Y$ is a modification of $X$ and has paths that are locally Hölder continuous of all orders $\gamma \in (0, (q - d)/p)$ almost surely.
\end{proof}


\begin{theorem}\label{thm:localized_holder_modification_sup}
  \uses{def:IsKolmogorovProcess, def:HasBoundedCoveringNumberCover}
  \leanok
  \lean{ProbabilityTheory.exists_modification_holder_iSup}
Let $T$ be a metric space with a cover $(T_n)$ with bounded covering numbers with constants $c_n$ and the same exponent $d$.
Let $(p_n, q_n)_{n \in \mathbb{N}}$ be a sequence of pairs of positive numbers such that $q_n > d$ for all $n \in \mathbb{N}$.
Let $X : T \to \Omega \to E$ be a process that satisfies the Kolmogorov condition with exponents $(p_n, q_n)$ for all $n \in \mathbb{N}$.
Then $X$ has a modification $Y$ such that almost surely the paths of $Y$ are locally Hölder continuous of all orders $\gamma \in (0, \sup_n (q_n - d)/p_n)$.
\end{theorem}

\begin{proof}\leanok
  \uses{thm:localized_holder_modification}
For each $n$, by Theorem~\ref{thm:localized_holder_modification} there is a modification $Y_n$ of $X$ such that the paths of $Y_n$ are locally Hölder continuous of all orders $\gamma \in (0, (q_n - d)/p_n)$.
By Lemma~\ref{lem:indistinguishable_of_modification_of_continuous}, any two processes $Y_n, Y_m$ are indistinguishable.
That is, almost surely $Y_n = Y_m$.
Since there are countably many such almost sure equalities, we get that almost surely there is equality for all $n, m$.
Let then $Y$ be the process equal to $Y_0$ on the event that the equalities hold, and equal to an arbitrary point $x_0 \in E$ otherwise.
Then first, $Y$ is a modification of $X$.
Then for any $\gamma < \sup_n (q_n - d)/p_n$ there is $n$ such that $\gamma < (q_n - d)/p_n$ and thus since $Y = Y_n$ the paths of $Y$ are locally Hölder continuous of order $\gamma$ almost surely.
\end{proof}

\input{chapters/brownian.tex}


\part{Stochastic integral}

\paragraph{Overview}

We describe the construction of a stochastic integral.

\paragraph{Status} The formalization is ongoing.

\paragraph{Formalization authors} Anyone is welcome to contribute!

\chapter{Debùt Theorem}
\label{chap:debut_theorem}

\section{Monotone class theorem}

% This section may be used also in other chapters of the blueprint. Therefore, depending on where it is used and also how big the proof becomes, we may want to move it somewhere else.
TODO: find the right generality (and some reference) to state the monotone class theorem and write the informal proof. It may be possible to adapt the following theorem: https://leanprover-community.github.io/mathlib4_docs/Mathlib/MeasureTheory/PiSystem.html#MeasurableSpace.DynkinSystem.generateFrom_eq.

\begin{definition}[Monotone class]\label{def:monotone_class}
  Let $\mathcal{M}$ be a collection of subsets of a set $X$. We say that $\mathcal{M}$ is a monotone class if it is closed under countable monotone unions and countable monotone intersections, i.e.:
  \begin{enumerate}
    \item if \( A_1, A_2, \ldots \in M \) and \( A_1 \subseteq A_2 \subseteq \cdots \), then
    \( \bigcup_{i=1}^\infty A_i \in M \),
    \item if \( B_1, B_2, \ldots \in M \) and \( B_1 \supseteq B_2 \supseteq \cdots \), then
    \( \bigcap_{i=1}^\infty B_i \in M \).
  \end{enumerate}
  Given a collection $\mathcal{F}$ of subsets of $X$, we call the smallest monotone class containing $\mathcal{F}$ the monotone class generated by $\mathcal{F}$.
\end{definition}

\begin{theorem}[Monotone class theorem]\label{thm:monotone_class}
  Let \(G\) be an algebra of subsets of a set \(X\). Then the monotone class generated by \(G\) coincides with the $\sigma$-algebra generated by \(G\).
\end{theorem}

\begin{proof}
  TODO
\end{proof}

\section{Debut}
The following proof is based on "R.F. Bass, The measurability of hitting times, Electron. Commun. Probab. {\bf 15} (2010), 99--105; MR2606507"
and the successive "R.F. Bass. "Correction to "The measurability of hitting times"." Electron. Commun. Probab. 16 189 - 191, 2011. \url{https://doi.org/10.1214/ECP.v16-1627}"
which is a rather clever and short proof evading the classical proof which uses more complex structures. Note that there exists also an Arxiv version of the paper with the corrections applied (\url{https://arxiv.org/pdf/1001.3619}), we will mostly reference this unified version.


Standard notation in this chapter:
$(\Omega, \mathcal{F} , P)$ is a probability space;
$\mathcal{S}$ is a topological space;
$\pi:\mathbb{R}_{\geq 0}\times\Omega\rightarrow \Omega$ is the projection; $P^*$ is the outer measure associated with $P$.

\begin{definition}[Progressively measurable set]\label{def:progr_meas_set}
  \leanok
  \lean{MeasureTheory.ProgMeasurableSet}
A subset of $[0, \infty) \times \Omega$ is progressively measurable if its indicator is a progressively measurable process.
\end{definition}

\begin{definition}[Debut of a set]\label{def:debut_set}
  \leanok
  \lean{MeasureTheory.Debut}
Let $E \subseteq{} [0, \infty) \times \Omega $, define $D_E = \inf\left\lbrace t \geq 0\ :\ (t, \omega) \in E\right\rbrace$, the debut of $E$.
\end{definition}

\begin{definition}[$\mathcal{K}^0$]\label{def:subsets_compact_RNN_times_measurable}
  \leanok
  \lean{MeasureTheory.𝓚₀}
Let $t>0$. Let $\mathcal{K}^0(t)$ be the collection of subsets of $[0, t] \times \Omega$ of the form $K \times C$, where $K$ is a compact
subset of $[0, t]$ and $C \in \mathcal{F}_t$.
\end{definition}

\begin{definition}[$\mathcal{K}$]\label{def:fin_union_RNN_times_measurable}
  \leanok
  \lean{MeasureTheory.𝓚}
  \uses{def:subsets_compact_RNN_times_measurable}
Let $t>0$. Let $\mathcal{K}(t)$ be the collection of finite unions of elements of $\mathcal{K}^0(t)$.
\end{definition}

\begin{definition}[$\mathcal{K}_\delta$]\label{def:count_inter_of_fin_union_RNN_times_measurable}
  \leanok
  \lean{MeasureTheory.𝓚δ}
  \uses{def:fin_union_RNN_times_measurable}
Let $t>0$. Let $\mathcal{K}_\delta(t)$ be the collection of countable intersections of elements of $\mathcal{K}(t)$.
\end{definition}

\begin{definition}[$t$-approximable set]\label{def:t_approx_set}
  \leanok
  \lean{MeasureTheory.Approximation}
  \uses{def:count_inter_of_fin_union_RNN_times_measurable}
Let $t>0$.
We say $A \in \mathcal{B}[0, t] \times \mathcal{F}_t$ is
$t$-approximable if given $\epsilon > 0$, there exists $B \in \mathcal{K}_\delta (t)$ with $B \subseteq{} A$ and
$$P^∗ (\pi(A)) \leq P^∗ (\pi(B)) + \epsilon,$$
where $\pi$ is the projection over $\Omega$.
\end{definition}

\begin{lemma}\label{lem:iInf_snd_eq_snd_iInf}
  \leanok
  \lean{MeasureTheory.iInf_snd_eq_snd_iInf_of_mem_𝓚δ}
  \uses{def:count_inter_of_fin_union_RNN_times_measurable}
If $B \in \mathcal{K}_\delta (t)$, $\forall n\in \mathbb{N}$, $B_n \in \mathcal{K}^\delta(t)$ and $B_n \searrow B$, then $\pi(B) = \bigcap_{n\in\mathbb{N}} \pi(B_n)$.
\end{lemma}

\begin{proof}
  % See the proof of Lemma 2.2 in the corrected paper.

  For each $\omega \in \Omega$ and each set $C \subseteq [0,+\infty) \times \Omega$, let
  \[S(C)(\omega) = \{s \leq t : (s,\omega) \in C\}.\]

  If $B \in \mathcal{K}_\delta(t)$, then $S(B)(\omega)$
  is compact. In fact, there exists a sequence $A_n \in \mathcal{K}(t)$ such that
  $A_n \searrow B$, therefore $S(A_n)(\omega)$ is compact and $S(A_n)(\omega) \searrow S(B)(\omega)$. In particular, $S(B_n)(\omega)$ is compact for each $n$.

  Now we divide the proof in two cases.

  One possibility is that
  $\bigcap_{n \in \mathbb{N}} S(B_n)(\omega) \neq \emptyset$; in this case, if $s \in \bigcap_{n \in \mathbb{N}} S(B_n)(\omega)$,
  then $(s,\omega) \in B_n$ for each $n$, and so $(s,\omega) \in B$. Therefore,
  $\omega \in \pi(B_n)$ for each $n$ and $\omega \in \pi(B)$.

  The other possibility is that $\bigcap_{n \in \mathbb{N}} S(B_n)(\omega) = \emptyset$.
  Since the sequence $S(B_n)(\omega)$ is a decreasing sequence of compact sets,
    $S(B_n)(\omega) = \emptyset$ for some $n$,
  for otherwise $\bigcap_{n \in \mathbb{N}} S(B_n)(\omega)$ would also be nonempty.
  Therefore $\omega \notin \pi(B_n)$ and $\omega \notin \pi(B)$.

  We conclude that $\omega \in \pi(B)$ if and only if $\omega \in \bigcap_{n \in \mathbb{N}} \pi(B_n)$, hence $\pi(B) = \bigcap_{n \in \mathbb{N}} \pi(B_n)$.
\end{proof}

\begin{lemma}\label{lem:K_delta_of_inter_K_delta}
  \leanok
  \lean{MeasureTheory.measurableSet_snd_of_mem_𝓚δ}
  \uses{def:count_inter_of_fin_union_RNN_times_measurable}
If $B \in \mathcal{K}_\delta (t)$, then $\pi(B) \in \mathcal{F}_t$.
\end{lemma}

\begin{proof}
  \uses{lem:iInf_snd_eq_snd_iInf}
  % See the proof of Lemma 2.2 in the corrected paper.
  By definition of $\mathcal{K}_\delta(t)$, $B = \bigcap_{n\in\mathbb{N}} B_n$ where $B_n \in \mathcal{K}(t)$. Therefore, by Lemma~\ref{lem:iInf_snd_eq_snd_iInf}, \[\pi(B) = \pi \left(\bigcap_{n\in\mathbb{N}} B_n\right) = \bigcap_{n\in\mathbb{N}} \pi(B_n) \in \mathcal{F}_t .\]
\end{proof}

\begin{lemma}\label{lem:measurable_of_t_approx}
  \leanok
  \lean{MeasureTheory.Approximation.measurableSet_snd}
  \uses{def:t_approx_set}
  If $A$ is $t$-approximable, then $\pi(A) \in \mathcal{F}_t$.
\end{lemma}

\begin{proof}
  \uses{lem:K_delta_of_inter_K_delta}
  % See the proof of Lemma 2.3 in the corrected paper.
  Choose $A_n \in \mathcal{K}_\delta(t)$ with $A_n \subseteq A$ and
  $P(\pi(A_n)) \to P^*(\pi(A))$. Let $B_n = A_1 \cup \cdots \cup A_n$ and
  let $B = \bigcup_{n \in \mathbb{N}} B_n$. Then
  $B_n \in \mathcal{K}_\delta(t)$, $B_n \nearrow B$, and $P(\pi(B_n)) \geq P(\pi(A_n)) \to
  P^*(\pi(A))$.
  Moreover, by Lemma~\ref{lem:K_delta_of_inter_K_delta}, $\pi(B_n) \in \mathcal{F}_t$.
  It follows that $\pi(B_n) \nearrow \pi(B)$, and so $\pi(B) \in \mathcal{F}_t$ and
  $$P(\pi(B)) = \lim_{n \to \infty} P(\pi(B_n)) = P^*(\pi(A)).$$

  For each $n$, there exists $C_n \in \mathcal{F}$ such that
  $\pi(A) \subseteq C_n$ and $P(C_n) \leq P^*(\pi(A)) + 1/n$. Setting
  $C = \bigcap_{n \in \mathbb{N}} C_n$, we have $\pi(A) \subseteq C$ and $P^*(\pi(A)) = P(C)$.
  Therefore
  $\pi(B) \subseteq \pi(A) \subseteq C$ and $P(\pi(B)) = P^*(\pi(A)) = P(C)$.
  This implies that $\pi(A) \setminus \pi(B)$ is a $P$-null set, and by
  the completeness assumption, $\pi(A) = (\pi(A) \setminus \pi(B)) \cup \pi(B) \in \mathcal{F}_t$.
\end{proof}

\begin{lemma}\label{lem:exists_B_of_t_approx}
  \leanok
  \lean{MeasureTheory.Approximation.tendsto_measure_diff_B'}
  \uses{def:count_inter_of_fin_union_RNN_times_measurable, def:t_approx_set}
  Suppose $A$ is $t$-approximable. Then, given $\epsilon > 0$, there exists
  $B \in \mathcal{K}_\delta (t)$ such that $P(\pi(A) \setminus \pi(B)) < \epsilon$.
\end{lemma}

\begin{proof}
  \uses{lem:K_delta_of_inter_K_delta}
  % See the proof of Lemma 2.3 in the corrected paper.
  Let $B_n$ and $B$ be as in the proof of Lemma~\ref{lem:measurable_of_t_approx}.
  Then,
  $$\lim_{n \to \infty} P(\pi(A) \setminus \pi(B_n)) = P(\pi(A) \setminus \pi(B)) = 0.$$
\end{proof}

% consider removing this lemma altogether, I think we do not even need it in the proof
\begin{lemma}\label{lem:aux1a}
  \leanok
If $A \subseteq \Omega$, there exists $C \in \mathcal{F}$ such that $A \subseteq C$ and $P^∗(A) = P(C)$.
\end{lemma}

\begin{proof}\leanok
  This is just \verb|MeasureTheory.exists_measurable_superset|.
\end{proof}

\begin{lemma}\label{lem:aux1b}
Let $(A_n)_{n\in\mathbb{N}},A\subseteq \Omega$.
Suppose $A_n \nearrow A$. Then $P^∗(A) = \lim_{n\rightarrow \infty} P^∗(A_n)$.
\end{lemma}

\begin{proof}\leanok
  This is just \verb|Monotone.measure_iUnion| (this is a version with the sup, if needed there is also the version with the limit).
\end{proof}

\begin{definition}[$\mathcal{L}$-sets]\label{def:L_sets}
  \leanok
  \lean{MeasureTheory.𝓛₀, MeasureTheory.𝓛₁, MeasureTheory.𝓛, MeasureTheory.𝓛σ, MeasureTheory.𝓛σδ}
  \uses{def:count_inter_of_fin_union_RNN_times_measurable}
  From hereafter the following sets are needed:

  \begin{itemize}
  \item $\mathcal{L}_0(X) := \left\lbrace A \times B\ :\ A \subseteq X ,\ A \text{ compact},\ B \in \mathcal{K}(t)\right\rbrace$
  \item $\mathcal{L}_1(X )$ the class of finite unions of sets in $\mathcal{L}_0(X )$
  \item $\mathcal{L} (X )$ the class of intersections of countable decreasing sequences in $\mathcal{L}_1(X )$
  \item $\mathcal{L}_\sigma(X )$ be the class of unions of countable increasing sequences of sets in $\mathcal{L} (X )$
  \item $\mathcal{L}_{\sigma\delta}(X )$ the class of intersections of countable decreasing sequences of sets in $\mathcal{L}_\sigma(X )$
  \end{itemize}
\end{definition}

\begin{lemma}\label{lem:exists_cpct_Hausdorff}
  \leanok
  \lean{MeasureTheory.exists_mem_𝓛σδ_of_measurableSet}
  \uses{def:L_sets}
If $A \in \mathcal{B}[0, t] \times \mathcal{F}_t$, there exists a compact Hausdorff space $X$ and $B \in \mathcal{L}_{\sigma\delta}(X )$ such  that $A = \rho^X (B)$.

Where $\rho^X:X\times ([0,t]\times\Omega)\rightarrow [0,t]\times\Omega$ is the projection.
\end{lemma}

\begin{proof}
  \uses{thm:monotone_class}
  % See Lemma 2.5 in the corrected paper.
  %TODO: this proof needs to be expanded in multiple lemmas, for now I just copy pasted it from the paper, but probably we will need to separately define the set M, prove as a lemma that it is a monotone class, etc. We will also need to find the monotone class theorem or prove it ourselves, in the latter case I think we will need to have a section dedicated to it.

  TODO: Reorganize this proof, possibly divide it in multiple lemmas.

  If $A \in \mathcal{K}(t)$, we take $X = [0,1]$, the unit interval with the
  usual topology and $B = X \times A$. Thus the collection $\mathcal{M}$ of
  subsets of $\mathcal{B}[0,t] \times \mathcal{F}_t$
  for which the lemma is satisfied contains $\mathcal{K}(t)$. We will
  show that $\mathcal{M}$ is a monotone class.


  Suppose $A_n \in \mathcal{M}$ with $A_n \downarrow A$. There exist compact Hausdorff
  spaces $X_n$ and sets $B_n \in \mathcal{L}_{\sigma\delta}(X_n)$ such that $A_n = \rho^{X_n}(B_n)$.
  Let $X = \prod_{n=1}^\infty X_n$ be furnished with the product topology. Let
  $\tau_n: X \times [0,t] \times \Omega \to X_n \times [0,t] \times \Omega$ be defined by $\tau_n(x,(s,\omega))
  = (x_n,(s,\omega))$ if $x = (x_1,x_2, \ldots)$. Let $C_n = \tau_n^{-1}(B_n)$
  and let $C = \bigcap_{n \in \mathbb{N}} C_n$. It is easy to check that $\mathcal{L}(X)$ is closed under
  the operations of finite unions and intersections, from which it follows
  that $C \in \mathcal{L}_{\sigma\delta}(X)$. If $(s,\omega) \in A$, then for each $n$ there exists $x_n \in X_n$ such that $(x_n,(s,\omega)) \in B_n$. Note that
  $((x_1,x_2, \ldots),(s,\omega)) \in C$ and therefore $(s,\omega) \in \rho^X(C)$.
  It is straightforward that $\rho^X(C) \subseteq A$, and we conclude
  $A \in \mathcal{M}$.

  Now suppose $A_n \in \mathcal{M}$ with $A_n \uparrow A$. Let $X_n$ and $B_n$ be as before.
  Let $X' = \bigcup_{n=1}^\infty (X_n \times \{n\})$ with the topology generated by
  $\{G \times \{n\}: G \text{ open in } X_n\}$. Let $X$ be the one point
  compactification of $X'$. We can write $B_n = \bigcap_{m \in \mathbb{N}} B_{nm}$ with
  $B_{nm} \in \mathcal{L}_\sigma(X_n)$. Let
  $$C_{nm} = \{((x,n),(s,\omega)) \in X \times [0,t] \times \Omega: x \in X_n, (x,(s,\omega)) \in B_{nm}\},$$
  $C_n = \bigcap_{m \in \mathbb{N}} C_{nm}$, and $C = \bigcup_{n \in \mathbb{N}} C_n$.
  Then $C_{nm} \in \mathcal{L}_\sigma(X)$ and so $C_n \in \mathcal{L}_{\sigma\delta}(X)$.

  If $((x,p),(s,\omega)) \in \bigcap_{m \in \mathbb{N}} \bigcup_{n \in \mathbb{N}} C_{nm}$, then
  for each $m$ there exists $n_m$ such that $((x,p),(s,\omega)) \in C_{n_mm}$.
  This is
  only possible if $n_m = p$ for each $m$. Thus $((x,p), (s, \omega)) \in \bigcap_{m \in \mathbb{N}} C_{pm} = C_p \subseteq C$.
  The other inclusion is easier and we thus obtain $C = \bigcap_{m \in \mathbb{N}}\bigcup_{n \in \mathbb{N}} C_{nm}$,
  which implies $C \in \mathcal{L}_{\sigma\delta}(X)$. We check that
  $A = \rho^X(C)$ along the same lines, and therefore $A \in \mathcal{M}$.


  If $\mathcal{I}^0(t)$ is the collection of sets of the form $[a,b) \times C$, where
  $a < b \leq t$ and $C \in \mathcal{F}_t$, and $\mathcal{I}(t)$ is the collection of finite
  unions of sets in $\mathcal{I}^0(t)$, then $\mathcal{I}(t)$ is an algebra of sets. We
  note that $\mathcal{I}(t)$ generates the $\sigma$-field $\mathcal{B}[0,t] \times \mathcal{F}_t$. A set
  in $\mathcal{I}^0(t)$ of the form $[a,b) \times C$ is the union of sets in $\mathcal{K}^0(t)$
  of the form $[a, b-(1/m)] \times C$, and it
  follows that every set in $\mathcal{I}(t)$ is the increasing union of sets
  in $\mathcal{K}(t)$. Since $\mathcal{M}$ is a monotone
  class containing $\mathcal{K}(t)$, then $\mathcal{M}$ contains $\mathcal{I}(t)$.
  By the monotone class theorem (Theorem~\ref{thm:monotone_class}), $\mathcal{M} = \mathcal{B}[0,t] \times \mathcal{F}_t$.
\end{proof}

\begin{lemma}\label{lem:t_approx_of_Borel_measurable}
  \leanok
  \lean{MeasureTheory.Approximation.of_mem_prod_borel}
  \uses{def:t_approx_set}
If $A \in \mathcal{B}[0, t] \times \mathcal{F}_t$, then $A$ is $t$-approximable.
\end{lemma}

\begin{proof}
  \uses{lem:exists_cpct_Hausdorff}
  % See the proof of Lemma 2.6 in the corrected paper.
  %TODO: expand this proof, this may need some auxiliary lemmas. For now I just copy pasted the proof from the paper

  TODO: Reorganize this proof, possibly divide it in multiple lemmas.

  We first prove that if $H \in \mathcal{L}(X)$, then $\rho^X(H) \in \mathcal{K}_\delta$. If $H \in \mathcal{L}_1(X)$,
  this is clear. Suppose that $H_n \downarrow H$ with each $H_n \in \mathcal{L}_1(X)$.
  If $(s,\omega) \in \bigcap_{n \in \mathbb{N}} \rho^X(H_n)$, there exist
  $x_n \in X$ such that $(x_n,(s,\omega)) \in H_n$. Then there exists a subsequence
  such that $x_{n_k} \to x_\infty$ by the compactness of $X$. Now $(x_{n_k},(s,\omega)) \in H_{n_k}
  \subseteq H_m$ for $n_k$ larger than $m$. For fixed $\omega$, $\{(x,s): (x,(s,\omega)) \in H_m\}$
  is compact, so $(x_\infty,(s,\omega)) \in H_m$ for all $m$. This implies
  $(x_\infty,(s,\omega)) \in H$. The other inclusion is easier and therefore $\bigcap_{n \in \mathbb{N}} \rho^X(H_n) = \rho^X(H)$.
  Since $\rho^X(H_n) \in \mathcal{K}_\delta(t)$, then $\rho^X(H) \in \mathcal{K}_\delta(t)$.
  We also observe that for fixed $\omega$, $\{(x,s):(x,(s,\omega)) \in H\}$
  is compact.

  Now suppose $A \in \mathcal{B}[0,t] \times \mathcal{F}_t$. Then by Lemma~\ref{lem:exists_cpct_Hausdorff}
  there exists a compact Hausdorff space $X$ and $B \in \mathcal{L}_{\sigma\delta}(X)$ such that $A = \rho^X(B)$. We can write
  $B = \bigcap_{n \in \mathbb{N}} B_n$ and $B_n = \bigcup_{m \in \mathbb{N}} B_{nm}$ with $B_n \downarrow B$, $B_{nm} \uparrow B_n$, and $B_{nm} \in \mathcal{L}(X)$.

  Let $a = P^*(\pi(A)) = P^*(\pi \circ \rho^X(B))$ and let $\epsilon > 0$.
  By Lemma~\ref{lem:aux1b},
  $$P^*(\pi \circ \rho^X(B \cap B_{1m})) \uparrow P^*(\pi \circ \rho^X(B \cap B_1))
  = P^*(\pi \circ \rho^X(B)) = a.$$
  Take $m$ large enough so that $P^*(\pi \circ \rho^X(B \cap B_{1m})) > a - \epsilon$,
  let $C_1 = B_{1m}$, and $D_1 = B \cap C_1$.

  We proceed by induction. Suppose we are given sets $C_1, \ldots, C_{n-1}$ and
  sets $D_1, \ldots, D_{n-1}$ with $D_{n-1} = B \cap \left(\bigcap_{i=1}^{n-1} C_i\right)$, $P^*(\pi
  \circ \rho^X(D_{n-1})) > a - \epsilon$, and each $C_i = B_{im_i}$ for
  some $m_i$. Since $D_{n-1} \subseteq B \subseteq B_n$, by Lemma~\ref{lem:aux1b}
  $$P^*(\pi \circ \rho^X(D_{n-1} \cap B_{nm}))
  \uparrow P^*(\pi \circ \rho^X(D_{n-1} \cap B_n))
  = P^*(\pi \circ \rho^X(D_{n-1})).$$
  We can take $m$ large enough so that
  $P^*(\pi \circ \rho^X(D_{n-1} \cap B_{nm})) > a - \epsilon$, let $C_n = B_{nm}$, and $D_n = D_{n-1} \cap C_n$.

  If we let $G_n = C_1 \cap \cdots \cap C_n$ and $G = \bigcap_{n \in \mathbb{N}} G_n = \bigcap_{n \in \mathbb{N}} C_n$, then
  each $G_n$ is in $\mathcal{L}(X)$, hence $G \in \mathcal{L}(X)$. Since $C_n \subseteq B_n$, then
  $G \subseteq \bigcap_{n \in \mathbb{N}} B_n = B$.
  Each $G_n \in \mathcal{L}(X)$ and so by the first paragraph of this proof, for each
  fixed $\omega$ and $n$, $\{(x,s): (x,(s,\omega)) \in G_n\}$
  is compact. Hence, by a proof very similar to that of Lemma~\ref{lem:iInf_snd_eq_snd_iInf},
  $\pi \circ \rho^X(G_n) \downarrow \pi \circ \rho^X(G)$.
  Using the first paragraph of this proof and Lemma~\ref{lem:iInf_snd_eq_snd_iInf},
  we see that $$P(\pi \circ \rho^X(G))
  = \lim_{n \to \infty} P(\pi \circ \rho^X(G_n)) \geq \lim_{n \to \infty} P^*(\pi \circ \rho^X(D_n)) \geq a - \epsilon.$$

  Using the first paragraph of this proof once again, we see that $A$ is $t$-approximable.
\end{proof}

\begin{theorem}\label{thm:debut_of_progr_meas_is_stop_time}
  \leanok
  \lean{MeasureTheory.Debut.isStoppingTime}
  \uses{def:progr_meas_set, def:debut_set}
If $E$ is a progressively measurable set, then $D_E$ is a stopping time.
\end{theorem}

\begin{proof}
  \uses{def:t_approx_set, lem:t_approx_of_Borel_measurable, lem:measurable_of_t_approx}
  % See the proof of Theorem 2.1 in the corrected paper.
  % TODO: It seems we need right continuity of the filtration, should we state it as an assumption?
  Let $E$ be a progressively measurable set and let $A_u = E \cap ([0,u] \times \Omega)$.
  By Lemma~\ref{lem:t_approx_of_Borel_measurable}, $A_u$ is $u$-approximable.
  By Lemma~\ref{lem:measurable_of_t_approx}, $\pi(A_u) \in \mathcal{F}_u$.
  Now fix $t$. If $\omega \in \{D_E \leq t\}$, we see that $\omega \in \pi(A_u)$ for all $u > t$.
  Conversely, if $\omega \in \pi(A_u)$ for all $u > t$, then $\omega \in \{D_E \leq t\}$.
  If $u_1 < u_2$, then $A_{u_1} \subseteq A_{u_2}$ and hence $\pi(A_{u_1}) \subseteq
  \pi(A_{u_2})$. Therefore
  $$\{D_E \leq t\} = \bigcap_{u > t} \pi(A_u) \in \bigcap_{u > t} \mathcal{F}_u = \mathcal{F}_t.$$
  Because $t$ was arbitrary, we conclude $D_E$ is a stopping time.
\end{proof}

\section{Hitting times}

% TODO: do we want to do the same distinction that the paper makes between the hitting time and the entry time?
% In the paper they are defined as follows:
% Entry time: inf{t \geq 0: X_t \in B}
% Hitting time: inf{t > 0: X_t \in B}
% What we call hitting time in Mathlib is actually a generalization of the entry time, but does not completely cover the hitting time. In case we would like to have both we may want to define another version of the hitting time with the strict inequality (essentially with Set.Ioc or Set.Ioo instead of Set.Icc).
% For the moment this does not seem necessary, and in any case the proof for the hitting time relies on the one for the entry time and takes a limit, so we can always add it later if needed.

\begin{theorem}\label{thm:hitting_is_stopping_time}
  \leanok
  \lean{MeasureTheory.hitting_isStoppingTime'}
  \uses{def:progr_meas_set, def:debut_set}
If $X$ is a progressively measurable process taking values in $\mathcal{S}$ and $B$ is a Borel-measurable subset of $\mathcal{S}$, then the hitting time of $X$ in $B$ is a stopping time.
\end{theorem}

\begin{proof}
  \uses{thm:debut_of_progr_meas_is_stop_time}
  % See the proof of Theorem 2.7 in the corrected paper.
  Since $B$ is a Borel subset of $\mathcal{S}$ and $X$ is progressively measurable,
  then $\mathbf{1}_B(X_t)$ is also progressively measurable. The hitting time is then the debut of the set
  $E = \{(s,\omega) : \mathbf{1}_B(X_s(\omega)) = 1\}$, and therefore is a stopping time by Theorem~\ref{thm:debut_of_progr_meas_is_stop_time}.
\end{proof}

\chapter{Doob-Meyer Theorem}
\label{chap:doob_meyer}


\section{Doob-Meyer}

This chapter follows \cite{Beiglböck_Schachermayer_Veliyev_2012}
which gives an elementary and short proof of the result.
\begin{definition}\label{def:Doob_Meyer_class}
$D$ is the class of all adapted processes $(S_t )_{0\leq t\leq T}$ such that $S_\tau$ is uniformly integrable w.r.t. any stopping time $\tau$.
\end{definition}

\begin{lemma}\label{lem:martingale_exists_dyadic_limit_left}
  Let $X=(X_t)_{t\in\mathcal{D}}$ be a martingale. There exists a negligible event $N$ such that, for every $t\geq 0$ the limit
  $$
  \lim_{\stackrel{s\rightarrow t^-}{s\in\mathcal{D}}}X_s(\omega)
  $$
  exists and is finite for every $\omega\in\Omega\setminus N$. Where $\mathcal{D}$ are the dyadics.
\end{lemma}
\begin{proof}
  See 8.2.1 of Pascucci.
\end{proof}

\begin{lemma}\label{lem:martingale_exists_dyadic_limit_right}
  Let $X=(X_t)_{t\in\mathcal{D}}$ be a martingale. There exists a negligible event $N$ such that, for every $t\geq 0$ the limit
  $$
  \lim_{\stackrel{s\rightarrow t^+}{s\in\mathcal{D}}}X_s(\omega)
  $$
  exists and is finite for every $\omega\in\Omega\setminus N$. Where $\mathcal{D}$ are the dyadics.
\end{lemma}
\begin{proof}
  See 8.2.1 of Pascucci.
\end{proof}

\begin{lemma}\label{lem:mg_is_cadlag}
  Let the filtered probability space satisfy the usual conditions.
  Then every martingale $X$ admits a modification that is still a martingale with cadlag trajectories.
\end{lemma}
\begin{proof}
  \uses{lem:martingale_exists_dyadic_limit_right,em:martingale_exists_dyadic_limit_left}
  See 8.2.3 of Pascucci.
\end{proof}

\begin{lemma}\label{lem:exists_cadlag_mod_of_nonneg_submg}
  Let the filtered probability space satisfy the usual conditions.
  Then every nonnegative submartingale $X$ admits a modification that is still a nonnegative submartingale with cadlag trajectories.
\end{lemma}
\begin{proof}
  \uses{lem:martingale_exists_dyadic_limit_right,em:martingale_exists_dyadic_limit_left}
  See 8.2.3 of Pascucci.
\end{proof}

\begin{lemma}\label{lem:exists_cadlag_mod_of_local_mg}
  Let the filtered probability space satisfy the usual conditions.
  Then every local martingale $X$ admits a modification that is still a local martingale with cadlag trajectories.
\end{lemma}
\begin{proof}
  \uses{lem:mg_is_cadlag}
\end{proof}

Firstly we will need Komlos' Lemma
%technically a more general version with Cesaro sums exists, but it is not needed for this case
%(see "J. Komlòs, A generalization of a problem of Steinhaus, Acta Math. Acad. Sci. Hungar. 18 (1967) 217–229").

\begin{lemma}\label{lem:komlos_aux}
  Let $H$ be a Hilbert space and $(f_n)_{n\in\mathbb{N}}$ a bounded sequence in $H$. Then there exist functions $g_n\in convex(f_n,f_{n+1},\cdots)$ such that $(g_n)_{n\in\mathbb{N}}$ converges in $H$.
\end{lemma}
\begin{proof}
  Let $r_n = \inf(\|g\|_2:g\in convex(f_n, f_{n+1},\ldots))$.
  Let $A=\sup_{n\geq1} r_n$. $A$ is finite by boundedness of $(f_n)_{n\in\mathbb{N}}$ and
  for each $n$ we  may pick some $g_n\in convex(f_n, f_{n+1},\ldots)$ such that $ \|g_n\|_2\leq A+1/n$ by $\inf$ and $\sup$ definitions.
  Let $\epsilon>0$.
  By construction $(r_n)_{n\in\mathbb{N}}$ is increasing. By properties of $\sup$ there exists $\bar{n}$ such that $r_{\bar{n}}\geq A-\epsilon$ and such that $\frac{1}{\bar{n}}\leq\epsilon$.
  Let $m\geq k\geq \bar{n}$. $(g_k+g_m)/2 \in convex(f_k,f_{k+1},\ldots)$. It follows since $(r_n)_{n\in\mathbb{N}}$ is increasing that
  $\|(g_k+g_m)/2\|_2\geq A-\epsilon$.
  Hence due to the ordering of $m,k,\bar{n}$
  $$ \|g_k-g_m\|_2^2=2 \|g_k\|_2^2+2\|g_m\|_2^2- \|g_k+g_m\|_2^2
  \leq 4(A+\frac{1}{\bar{n}})^2-4(A-\epsilon)^2\leq 16A\epsilon.$$ By completeness, $(g_n)_{n\geq1}$  converges in $\|.\|_2$.
\end{proof}

\begin{lemma}\label{lem:convex_of_converg_seq_is_converg}
  Let $X$ be a normed vector space (over $\mathbb{R}$).
  %For topological spaces we need that a convex combinations of elements of neighborhoods are still in the neighborhood (just like balls). Also for metric space we want that $d(ax,ay)\leq a d(x,y)$ (in lean dist_pair_smul).
  Let $(x_n)_{n\in\mathbb{N}}$ be a sequence in $X$ converging to $x$ w.r.t. the topology of $X$.
  Let $(N_n)_{n\in\mathbb{N}}$ be a sequence in $\mathbb{N}$ such that $n\leq N_n$ for every $n\in\mathbb{N}$ (maybe here we could have $N_n$ increasing WLOG).
  Let $(a_{n,m})_{n\in\mathbb{N},m\in\left\lbrace n,\cdots,N_n\right\rbrace}$ be a triangular array in $\mathbb{R}$ such that $0\leq a_{n,m}\leq 1$ and $\sum_{m=n}^{N_n}a_{n,m}=1$.
  Then $(\sum_{m=n}^{N_n}a_{n,m}x_m)_{n\in\mathbb{N}}$ converges to $x$ uniformly w.r.t. the triangular array.
\end{lemma}
\begin{proof}
  Let $\epsilon>0$.
  By convergence of $x_n$ we have $\exists \bar{n}$ such that $\forall n\geq\bar{n}$ $|x_n-x|\leq \epsilon$.
  By triangular inequality it follows that
  $$
  |\sum_{m=n}^{N_n}a_{n,m}x_m - x|\leq \sum_{m=n}^{N_n}a_{n,m}|x_m-x|\leq\epsilon.
  $$
\end{proof}

\begin{lemma}\label{lem:komlos_convex_aux}
  For $i,n\in\mathbb{N}$ set $f_{n}^{(i)}:=f_n \mathbb{1}_{(|f_n|\leq i)}$ such that $f_{n}^{(i)}\in L^2$.
  There exists the sequence of convex weights $\lambda_n^{n}, \ldots, \lambda_{N_n}^{n}$ such that the functions
  $ (\lambda_n^{n} f_n^{(i)} + \ldots+\lambda_{N_n}^{n} f_{N_n}^{(i)})_{n\in\mathbb{N}}$
  converge in $L^2$ for every $i\in\mathbb{N}$ uniformly.
\end{lemma}
\begin{proof}
  \uses{lem:komlos_aux, lem:convex_of_converg_seq_is_converg}
  Firstly by lemma \ref{lem:komlos_aux} over $(f_n^{(1)})_{n\in\mathbb{N}}$ there exist convex weights $\prescript{1}{}{\lambda}^n_n,\cdots,\prescript{1}{}{\lambda}^n_{N^1_n}$ such that
  $g^1_n=\sum_{m=n}^{N^1_n}\prescript{1}{}{\lambda}^n_mf_m^{(1)}$ converges to some $g^1$.
  Secondly apply the lemma to $(\tilde{g}^2_n=\sum_{m=n}^{N^1_n}\prescript{1}{}{\lambda}^n_mf^{(2)}_m)_{n\in\mathbb{N}}$, there exists convex weights $\tilde{\lambda}^n_n,\cdots,\tilde{\lambda}^n_{\tilde{N}_n}$ such that
  $g^2_n=\sum_{m=n}^{\tilde{N}_n}\tilde{\lambda}^n_m\tilde{g}_m^{(2)}=\sum_{m=n}^{N^2_n}\prescript{2}{}{\lambda}^n_mf_m^{(2)}$ converges to some $g^2$.
  Notice that $\sum_{m=n}^{N^2_n}\prescript{2}{}{\lambda}^n_mf_m^{(1)}=\sum_{m=n}^{\tilde{N}_n}\tilde{\lambda}^n_m\tilde{g}_m^{(1)}$ and thus this sequence by lemma \ref{lem:convex_of_converg_seq_is_converg} converges still to $g^1$.
  By iteration we may define $\prescript{i}{}{\lambda}^n_n,\cdots,\prescript{i}{}{\lambda}^n_{N^i_n}$ convex weights such that if used on $(f^j_n)_{n\in\mathbb{N}}$ they make the sequence convergent if $1\leq j\leq i$.
  At this point consider $\lambda^n_m=\prescript{n}{}{\lambda}^n_m$.
  Since $\forall m\geq i$ $\sum_{j=n}^{N^m_n}\prescript{m}{}{\lambda}^n_j f^{(i)}_j\rightarrow g^i$ and even better
  $\forall\epsilon>0$ $\exists\bar{n}$, $\forall n\geq\bar{n}$, $\forall m\geq i$ $|\sum_{j=n}^{N^m_n}\prescript{m}{}{\lambda}^n_j f^{(i)}_j - g^i|\leq\epsilon$
  (this works by lemma \ref{lem:convex_of_converg_seq_is_converg} uniformity of convergence w.r.t. triangular array) this concludes.
\end{proof}

\begin{lemma}[Komlòs Lemma]\label{lem:komlos}
  Let $( f_n)_{n\in\mathbb{N}}$ be a uniformly integrable sequence of functions on a probability space $(\Omega , \mathcal{F} , P)$.
  Then there exist functions $g_n \in convex( f_n, f_{n+1}, \cdots)$ such that $(g_n)_{n\in\mathbb{N}}$ converges in  $L^1 (\Omega )$.
\end{lemma}
\begin{proof}
  \uses{lem:komlos_convex_aux}
  For $i,n\in\mathbb{N}$ set $f_{n}^{(i)}:=f_n \mathbb{1}_{(|f_n|\leq i)}$ such that $f_{n}^{(i)}\in L^2$.
  Using \ref{lem:komlos_convex_aux} there exist for every $n$ convex weights $\lambda_n^{n}, \ldots, \lambda_{N_n}^{n}$ such that the functions
  $ \lambda_n^{n} f_n^{(i)} + \ldots+\lambda_{N_n}^{n} f_{N_n}^{(i)}$ converge in $L^2$ for every $i\in\mathbb{N}$.
  By uniform integrability, $\lim_{i\to \infty}\| f^{(i)}_n- f_n\|_1=0$, uniformly with respect to $n$.
  Hence, once again, uniformly with respect to $n$,
  $$ \textstyle\lim_{i\to\infty}\|  (\lambda_n^{n} f_n^{(i)} + \ldots+\lambda_{N_n}^{n} f_{N_n}^{(i)})-(\lambda_n^{n} f_n + \ldots+\lambda_{N_n}^{n} f_{N_n})\|_1= 0.$$
  Thus $(\lambda_n^{n} f_n + \ldots+\lambda_{N_n}^{n} f_{N_n})_{n\geq 1}$  is a Cauchy sequence in $L^1$.
\end{proof}

For uniqueness of Doob-Meyer Decomposition we will need theorem \ref{thm:IsLocalMartingale.eq_zero_of_finiteVariation}.

We now start the construction for the existence part, fix $T>0$, let $\mathcal{D}_n^T=\left\lbrace \frac{k}{2^n}T,\quad k=0,\cdots 2^n\right\rbrace$. Define $A_0=0$ and for $t\in\mathcal{D}_n^T$
\begin{align*}
A^n_t&=A^n_{t-T2^{-n}} + \mathbb{E}\left[ S_t-S_{t-T2^{-n}}|\mathcal{F}_{t-T2^{-n}}\right],\\
M^N_t&=S_t-A^n_t.
\end{align*}

\begin{lemma}\label{lem:Doob_Meyer_Finite_Predictable}
  $(A^n_t)_{t\in\mathcal{D}_n^T}$ is a predictable process.
\end{lemma}
\begin{proof}
  Trivial
\end{proof}

\begin{lemma}\label{lem:Doob_Meyer_Finite_Martingale}
  $(M^n_t)_{t\in\mathcal{D}_n^T}$ is a martingale.
\end{lemma}
\begin{proof}
  Trivial
\end{proof}

\begin{lemma}\label{lem:Predict_Part_Increasing}
  $(A^n_t)_{t\in\mathcal{D}_n^T}$ is an increasing process.
\end{lemma}
\begin{proof}
  $S$ is a submartingale.
\end{proof}

\begin{lemma}\label{lem:A_uniform_integrabl}
  \uses{lem:Doob_Meyer_Finite_Predictable,lem:Predict_Part_Increasing,lem:Doob_Meyer_Finite_Martingale}
  The sequence $(A^n_T)_{n\in\mathbb{N}}$ is uniformly integrable (bounded in $L^1$ norm).
\end{lemma}
\begin{proof}
  WLOG $S_T=0$ and $S_t\leq 0$ (else consider $S_t-\mathbb{E}\left[S_T\vert\mathcal{F}_{t}\right]$).

  We have that $0=S_T=M^n_T+A^n_T$. Thus
  \begin{equation}\label{equation_DM_e1}
  M^n_T=-A^n_T.
  \end{equation}
  Since $M^n$ is a martingale it follows by optimal sampling that for any $(\mathcal{F}_t)_{t\in\mathcal{D}_n}$ stopping time $\tau$
  \begin{equation}\label{equation_DM_e2}
  S_\tau=M^n_\tau+A^n_\tau = \mathbb{E}[M^n_T\vert\mathcal{F}_\tau]+A^n_\tau\stackrel{\eqref{equation_DM_e1}}{=} -\mathbb{E}[A^n_T\vert\mathcal{F}_\tau]+A^n_\tau.
  \end{equation}
  Let $c>0$. Define the last time when $A^n$ has always been inside $[0,c]$, by the Debùt Theorem \ref{thm:hitting_is_stopping_time} and the fact that $A^n$ is predictable the following is a stopping time
  $$
  \tau_n(c)=\inf\left(\frac{j-1}{2^n}T\vert\, A^n_{jT2^{-n}}>c\right)\wedge T.
  $$
  By construction $A^n_{\tau_n(c)}\leq c$. It follows that
  \begin{equation}\label{equation_DM_e3}
  S_{\tau_n(c)}\stackrel{\eqref{equation_DM_e2}}{=}-\mathbb{E}[A^n_T\vert\mathcal{F}_{\tau_n(c)}]+A^n_{\tau_n(c)}\leq -\mathbb{E}[A^n_T\vert\mathcal{F}_{\tau_n(c)}]+c.
  \end{equation}
  Since $(A^n_T>c)=(\tau_n(c)<T)$ we have
  \begin{align}\nonumber
  \int_{(A^n_T>c)}A^n_TdP&=\int_{(\tau_n(c)<T)}A^n_TdP\stackrel{\mathrm{Tower}}{=}\int_{(\tau_n(c)<T)}\mathbb{E}[A^n_T\vert\mathcal{F}_{\tau_n(c)}]dP\\
  &\stackrel{\eqref{equation_DM_e3}}{\leq} cP(\tau_n(c)<T)-\int_{\tau_n(c)<T}S_{\tau_n(c)}dP.\label{equation_DM_e4}
  \end{align}
  Now we notice that $(\tau_n(c)<T)\subseteq (\tau_n(c/2)<T)$, thus
  \begin{align}\nonumber
  \int_{\tau_n(c/2)<T}-S_{\tau_n(c/2)}dP&\stackrel{\eqref{equation_DM_e2}}{=}\int_{(\tau_n(c/2))<T}\mathbb{E}[A^n_T\vert\mathcal{F}_{\tau_n(c/2)}]-A^n_{\tau_n(c/2)}dP\\
  &\stackrel{\mathrm{Tower}}{=}\int_{(\tau_n(c/2)<T)}A^n_t-A^n_{\tau_n(c/2)}dP\geq \int_{(\tau_n(c)<T)}A^n_t-A^n_{\tau_n(c/2)}dP\nonumber\\
  \intertext{(over the event $(\tau_n(c)<T)$ $A^n_T\geq c$ and $A^n_{\tau_n(c/2)}\leq c/2$, thus $A^n_T-A^n_{\tau_n(c/2)}\geq c/2$)}
  &\geq \frac{c}{2}P(\tau_n(c)<T).\label{equation_DM_e5}
  \end{align}
  It follows
  $$
  \int_{(A^n_T>c)}A^n_TdP\stackrel{\eqref{equation_DM_e4}}{\leq}cP(\tau_n(c)<T)-\int_{\tau_n(c)<T}S_{\tau_n(c)}dP\stackrel{\eqref{equation_DM_e5}}{\leq}-2\int_{\tau_n(c/2)<T}S_{\tau_n(c/2)}dP-\int_{\tau_n(c)<T}S_{\tau_n(c)}dP.
  $$
  We may notice that
  $$
  P(\tau_n(c)<T)=P(A^n_T>c)\stackrel{Markov}{\leq}\frac{\mathbb{E}[A^n_T]}{c}=-\frac{\mathbb{E}[M^n_T]}{c}\stackrel{mg}{=}-\frac{\mathbb{E}[S_0]}{c}
  $$
  which goes to $0$ uniformly in $n$ as $c$ goes to infinity.
  This implies that $\int_{(A^n_T>c)}A^n_TdP$ is uniformly bounded in $n$ due to the fact that $S$ is of class $D$. And so also the $L^1$ norm is uniformly bounded.
\end{proof}

\begin{lemma}\label{lem:M_uniform_integrabl}
  The sequence $(M^n_T)_{n\in\mathbb{N}}$ is uniformly integrable (bounded in $L^1$ norm).
\end{lemma}
\begin{proof}
  \uses{lem:A_uniform_integrabl}
  $M^n_T=S_T-A^n_T$, also $S$ is of class $D$ and $A^n_T$ is uniformly integrable.
\end{proof}

\begin{lemma}\label{lem:incr_fun_lim_right_cont_limsup_ineq}
  If $f_n, f : [0, 1] \rightarrow \mathbb{R}$ are increasing functions such that $f$ is right continuous and
  $\lim_n f_n(t) = f (t)$ for $t \in\mathcal{D}^T$, then  $\limsup_n  f_n(t) \leq f (t)$ for all $t \in [0, T]$.
\end{lemma}
\begin{proof}
  Let $t\in[0,T]$ and $s\in\mathcal{D}^T$ such that $t<s$. We have
  $$
  \limsup_n f_n(t)\leq \limsup_n f_n(s)=f(s).
  $$
  Since the above is true uniformly in $s$ in particular since $f$ is right-continuous
  $$
  \limsup_n f_n(t)\leq\lim_{\stackrel{s\rightarrow t^+}{s\in\mathcal{D}^T}}f(s)=f(t).
  $$
\end{proof}

\begin{lemma}\label{lem:incr_fun_lim_right_cont_lim_eq}
  If $f_n, f : [0, 1] \rightarrow \mathbb{R}$ are increasing functions such that $f$ is right continuous and
  $\lim_n f_n(t) = f (t)$ for $t \in\mathcal{D^T}$, if $f$ is continuous in $t\in[0,T]$ then $\lim_n  f_n(t) = f (t)$.
\end{lemma}
\begin{proof}
  \uses{lem:incr_fun_lim_right_cont_limsup_ineq}
  By lemma \ref{lem:incr_fun_lim_right_cont_limsup_ineq} it is enough to show that $\liminf_n f_n(t)\geq f(t)$.
  Let $s\in\mathcal{D}^T$ such that $t>s$. We have
  $$
  \liminf_n f_n(t)\geq \liminf_n f_n(s)=f(s).
  $$
  Since the above is true uniformly in $s$ in particular since $f$ is continuous in $t$
  $$
  \liminf_n f_n(t)\geq\lim_{\stackrel{s\rightarrow t^-}{s\in\mathcal{D}^T}}f(s)=f(t).
$$
\end{proof}

Define $M^n_t$ on $[0,T]$ using $M^n_t=\mathbb{E}[M^n_T\vert\mathcal{F}_t]$.

\begin{lemma}\label{lem:M_n_cadlag_mg}
  $M^n_t$ admits a modification which is a cadlag martingale.
\end{lemma}
\begin{proof}
  \uses{lem:mg_is_cadlag}
  By theorem \ref{lem:mg_is_cadlag}
\end{proof}

From this point onwards $M^n_t$ will be redefined as the modification from lemma \ref{lem:M_n_cadlag_mg}.
\begin{lemma}\label{lem:M_cal_converges_L1}
  There are convex weights $\lambda^n_n,\cdots,\lambda^n_{N_n}$ such that
  $\mathcal{M}^n_T\stackrel{L^1}{\rightarrow}M$, where $\mathcal{M}^n:=\lambda^n_nM^n+\cdots +\lambda^n_{N_n}M^{N_n}.$
\end{lemma}
\begin{proof}
  \uses{lem:M_uniform_integrabl,lem:komlos}
  By lemma \ref{lem:M_uniform_integrabl} $(M^n_T)_{n\in\mathbb{N}}$ is uniformly bounded in $L^1$, thus by lemma \ref{lem:komlos} there are convex weights $\lambda^n_n,\cdots,\lambda^n_{N_n}$ such that
  $\mathcal{M}^n_T\stackrel{L^1}{\rightarrow}M$, where $\mathcal{M}^n:=\lambda^n_nM^n+\cdots +\lambda^n_{N_n}M^{N_n}.$
\end{proof}

\begin{lemma}\label{lem:M_cal_cadlag}
  $\mathcal{M}^n$ is cadlag.
\end{lemma}
\begin{proof}
  \uses{lem:M_n_cadlag_mg,lem:M_cal_converges_L1}
  By construction and \ref{lem:M_n_cadlag_mg}
\end{proof}

Let \begin{equation}\label{equation_DM_e6} M_t = \mathbb{E}[M\vert\mathcal{F}_t].\end{equation}

\begin{lemma}\label{lem:M_cadlag_mg}
  $M_t$ admits a martingale cadlag modification.
\end{lemma}
\begin{proof}
  \uses{lem:M_cal_converges_L1, lem:mg_is_cadlag}
  By construction $M_t$ is a martingale and thus by theorem \ref{lem:mg_is_cadlag} admits a cadlag martingale modification
  ($M_t$ is a version of $\mathbb{E}[M\vert\mathcal{F}_t]$ and thus passing to modification does not pose any problem).
\end{proof}

From this point onwards $M^n_t$ will be redefined as the modification from lemma \ref{lem:M_cadlag_mg}.
Define
\begin{itemize}
  \item Extend now $A^n$ as a left continuous process $A^n_s:=\sum_{t\in\mathcal{D}^T_n}A^n_t\mathbb{1}_{]t-2^{-n},t]}(s)$
  \item $\mathcal{A}^n=\lambda^n_nA^n+\cdots +\lambda^n_{N_n}A^{N_n}$
  \item $A_t=S_t-M_t$
\end{itemize}

\begin{lemma}\label{lem:M1_komlos}
  For every $t\in[0,T]$ we have $\mathcal{M}^n_t\stackrel{L^1}{\rightarrow}M_t$.
\end{lemma}
\begin{proof}
  \uses{lem:M_cal_converges_L1}
  We may notice that by Jensen's inequality, the tower lemma and lemma \ref{lem:M_cal_converges_L1}
  \begin{gather}\nonumber
    \mathbb{E}[|\mathcal{M}^n_t-M_t|]=\mathbb{E}[|\mathbb{E}[\mathcal{M}^n_T-M\vert\mathcal{F}_t]|]\leq \mathbb{E}[|\mathcal{M}^n_T-M|]\rightarrow0,\\
    \Rightarrow\mathcal{M}^n_t\stackrel{L^1}{\rightarrow} M_t,\quad \forall t\in[0,T].\label{equation_DM_e7}
  \end{gather}
\end{proof}

\begin{lemma}\label{lem:A_cal_conv_A_on_D_T}
  There exists a set $E\subseteq\Omega$, $P(E)=0$ and a subsequence $k_n$ such that $\lim_n\mathcal{A}^{k_n}_t(\omega)=A_t(\omega)$ for every $t\in\mathcal{D}^T,\omega\in\Omega\setminus E$.
\end{lemma}
\begin{proof}
  \uses{lem:M1_komlos}
  By Lemma \ref{lem:M1_komlos}
  $$
  \mathcal{A}^n_t=S_t-\mathcal{M}^n_t\stackrel{L^1}{\rightarrow}S_t-M_t=A_t,\quad\forall t\in\mathcal{D}^T.
  $$
  $\mathcal{D}^T$ is countable we can arrange the elements as $(t_n)_{n\in\mathbb{N}}$.
  Given $t_0\in\mathcal{D}^T$ there exists a subsequence $k^{0}_n$ for which $\mathcal{A}^{k^{0}_n}_{t_0}$ converges to $A_{t_0}$ over the set $\Omega\setminus E_{0}$ where $P(E_{0})=0$.
  Suppose we have a sequence $k^m_n$ for which $\mathcal{A}^{k^j_n}_{t_j}$ converges to $A_{t_j}$ over the set $\Omega\setminus E_{m}$ where $P(E_{m})=0$ for each $j=0,\cdots,m$.
  From this subsequence we may extract a new subsequence $k^{m+1}_n$ for which $\mathcal{A}^{k^{m+1}_n}_{t_{m+1}}$ converges to $A_{t_{m+1}}$ over the set $\Omega\setminus E_{m+1}$ where $P(E_{m+1})=0$.
  By construction over this subsequence the convergence for $t_0,\cdots,t_m$ still applies.
  With a diagonal argument we obtain the final result with $E=\bigcup_n E_n$.
\end{proof}

\begin{lemma}\label{lem:A_increasing}
  $(A_t)_{t\in[0,T]}$ is an increasing process.
\end{lemma}
\begin{proof}
  \uses{lem:A_cal_conv_A_on_D_T, lem:Predict_Part_Increasing, lem:M_cadlag_mg}
  Since $\mathcal{A}^n_t$ is increasing on $\mathcal{D}^T$ by lemma \ref{lem:A_cal_conv_A_on_D_T} also $A$ is almost surely increasing on $\mathcal{D}^T$.
  Since $S,M$ are cadlag also $A$ is cadlag (thus right-continuous). It follows that $A$ must be increasing on $[0,T]$.
\end{proof}

\begin{lemma}\label{lem:lim_Exp_A_n_tau_is_Exp_A_tau}
  Let $\tau$ be an $(\mathcal{F}_t)_{t\in[0,T]}$ stopping time. We have $\lim_n\mathbb{E}[A^n_\tau]=\mathbb{E}[A_\tau]$.
\end{lemma}
\begin{proof}
  \uses{lem:M_cadlag_mg, lem:M_n_cadlag_mg}
  Let $\sigma_n:=\inf\left(t\in\mathcal{D}^T_n\vert t>\tau\right)$. By construction of $A^n$ we have $A^n_\tau=A^n_{\sigma_n}$.
  Also $\sigma_n\searrow\tau$. Since $S$ is of class $D$ and cadlag we have
  \begin{align*}
    \mathbb{E}[A^n_\tau]&=\mathbb{E}[A^n_{\sigma_n}]=\mathbb{E}[S_{\sigma_n}]-\mathbb{E}[M^n_{\sigma_n}]=\mathbb{E}[S_{\sigma_n}]-\mathbb{E}[M^n_0]=\\
    &=\mathbb{E}[S_{\sigma_n}]-\mathbb{E}[S_0]\rightarrow \mathbb{E}[S_\tau]-\mathbb{E}[M_0]=\mathbb{E}[S_\tau]-\mathbb{E}[M_\tau]=\mathbb{E}[A_\tau].
  \end{align*}
\end{proof}

\begin{lemma}\label{lem:limsup_A_n_tau_is_A_tau_ae}
  Let $\tau$ be an $(\mathcal{F}_t)_{t\in[0,T]}$ stopping time. We have $\limsup_n \mathcal{A}_\tau^n = A_\tau$.
\end{lemma}
\begin{proof}
  \uses{lem:lim_Exp_A_n_tau_is_Exp_A_tau, lem:incr_fun_lim_right_cont_limsup_ineq, lem:Predict_Part_Increasing, lem:A_cal_conv_A_on_D_T}
  Firstly we notice that $\liminf_n \mathbb{E}[A_\tau^n]  \leq \limsup_n  \mathbb{E}  [\mathcal{A}_\tau^n  ]  \leq \mathbb{E}[\limsup_n  \mathcal{A}_\tau^n  ]  \leq \mathbb{E}[ A_\tau ]$,
  where the first inequality is justified by the definition of limsup and liminf and the fact that
  $$
  \sup_{k\geq n}\mathbb{E}[\mathcal{A}^k_\tau]\geq \sum_{m=k}^{N_k}\lambda^k_m\mathbb{E}[A^m_\tau]\geq \sum_{m=k}^{N_k}\lambda^k_m\inf_{j\geq n}\mathbb{E}[A^j_\tau]=\inf_{k\geq n}\mathbb{E}[A^k_\tau]
  $$
  the third inequality by \ref{lem:incr_fun_lim_right_cont_limsup_ineq}.
  Let's prove the second inequality: observe that
  $$
  \mathcal{A}^n_\tau= A_1+\mathcal{A}^n_\tau-A_1\leq A_1+(\mathcal{A}^n_\tau-A_1)_+,
  $$
  thus it follows that $\mathcal{A}^n_\tau - (\mathcal{A}^n_\tau-A_1)_+\leq A_1$; since $A_1$ is an integrable guardian the inverse Fatou Lemma may be applied to show together with limsup properties that
  \begin{align*}
    \limsup_n\mathbb{E}[\mathcal{A}^n_\tau]+0 &= \limsup_n\mathbb{E}[\mathcal{A}^n_\tau]+\liminf_n-\mathbb{E}[(\mathcal{A}^n_\tau-A_1)_+] \leq \limsup_n\mathbb{E}[\mathcal{A}^n_\tau-(\mathcal{A}^n_\tau-A_1)_+]\leq\\
    &\leq \mathbb{E}[\limsup_n\mathcal{A}^n_\tau-(\mathcal{A}^n_\tau-A_1)_+]\leq \mathbb{E}[\limsup_n\mathcal{A}^n_\tau]-\mathbb{E}[\liminf_n(\mathcal{A}^n_\tau-A_1)_+]\leq\mathbb{E}[\limsup_n\mathcal{A}^n_\tau],
    \end{align*}
  where the first equality is justified by the fact that $\mathcal{A}^n_\tau\leq\mathcal{A}^n_1\rightarrow A_1$ almost surely.
  Due to lemma \ref{lem:lim_Exp_A_n_tau_is_Exp_A_tau} and \ref{lem:incr_fun_lim_right_cont_limsup_ineq} the first sequence of inequalities is a sequence of equalities, thus
  we know that $A_\tau- \limsup_n \mathcal{A}_\tau^n $ is an a.s. nonnegative function with null expected value, and thus it must be almost everywhere null.
\end{proof}

\begin{theorem}\label{thm:Doob_Meyer}
  Let $S = (S_t )_{0\leq t\leq T}$ be a cadlag submartingale of class $D$.
  Then, $S$ can be written in a unique way in the form  $S = M + A$ where $M$ is a cadlag martingale and $A$ is a predictable increasing process starting at $0$.
\end{theorem}
\begin{proof}
  \uses{lem:A_increasing, lem:M_cadlag_mg, lem:limsup_A_n_tau_is_A_tau_ae, lem:incr_fun_lim_right_cont_lim_eq}
  By construction $M$ is a cadlag martingale and $A_0=0$ and by lemma \ref{lem:A_increasing} $A$ is increasing. It suffices to show that $A$ is predictable.
  $A^n,\mathcal{A}^n$ are left continuous and adapted, and thus they are predictable (measurable wrt the predictable sigma algebra (the one generated by left-cont adapted processes)).
  It is enough to show that $\omega-a.e.$, $\forall t\in[0,T]$, $\limsup_n\mathcal{A}^n_t(\omega)=A_t(\omega)$.

  By lemma \ref{lem:incr_fun_lim_right_cont_lim_eq} that is true for any continuity point of $A$. Since $A$ is increasing it can only have a finite amount of jumps larger than $1/k$ for any $k\in\mathbb{N}$.
  Consider now $\tau_{q,k}$ the family of stopping times equal to the $q$-th time that the process $A_t$ has a jump higher than $1/k$. This is a countable family.
  Given a time $t$ and a trajectory $\omega$ there are only two possibilities: either $A$ is continuous or not at time $t$ along $\omega$.
  If $A$ is continuous at time $t$ we have $\limsup_n\mathcal{A}^n_t(\omega)=A_t(\omega)$, if it jumps there exists $q(\omega),k(\omega)$ such that $t=\tau_{q(\omega),k(\omega)}(\omega)$.
  Due to lemma \ref{lem:limsup_A_n_tau_is_A_tau_ae} we know that $\limsup_n A^n_{\tau_{q,k}} = A_{\tau_{q,k}}$ for each $q,k$ almost surely. Thus, since it is an intersection of a countable amount of almost sure
  events $\forall\omega\in\Omega'$ with $P(\Omega')=1$, for each $q,k$ $\limsup_n A^n_{\tau_{q,k}}(\omega) = A_{\tau_{q,k}}(\omega)$ ($\omega$ does not depend upon $q,k$).
  Consequently, $\forall\omega\in\Omega'$ we have $\limsup_n\mathcal{A}^n_t(\omega)=\limsup_n\mathcal{A}^n_{\tau_{q(\omega),k(\omega)}}(\omega)=A_{\tau_{q(\omega),k(\omega)}}(\omega)=A_t(\omega)$
\end{proof}

\chapter{Stochastic integral}

The lecture notes at \href{https://dec41.user.srcf.net/h/III_L/stochastic_calculus_and_applications/}{this link} as well as chapter 18 of \cite{kallenberg2021} are good references for this chapter.
Some of the proofs are taken from \cite{pascucci2024}.

\section{Total variation and Lebesgue-Stieltjes integral}

TODO: in Mathlib, we can integrate with respect to the measure given by a right-continuous monotone function (\texttt{StieltjesFunction.measure}). This will be useful to integrate against the quadratic variation of a local martingale.
However, we will also want to integrate with respect to a signed measure given by a càdlàg function with finite variation.
We need to investigate what's already in Mathlib. See \texttt{Mathlib.Topology.EMetricSpace.BoundedVariation}.


\section{Doob's Lp inequality}

In this section, we prove Doob's Lp inequality.

\begin{lemma}\label{lem:convex_of_mg_is_submg}
  \uses{def:Martingale, def:Submartingale}
Let $X : T \rightarrow \Omega\rightarrow E$ a martingale with values in a normed space $E$.
Let $\phi : E \rightarrow \mathbb{R}$ convex such that
$\phi(X_t)\in L^1(\Omega)$ for every $t\in T$. Then $\phi(X)$ is a sub-martingale.
\end{lemma}

\begin{proof}
  % See 1.4.12 Pascucci
  By the conditional Jensen inequality (see \href{https://github.com/leanprover-community/mathlib4/pull/27953}{\#27953})
  $\phi(X_t) = \phi\left( \mathbb{E}[X_T\ |\ \mathcal{F}_t] \right)\leq \mathbb{E}[\phi(X_T)\ |\ \mathcal{F}_t]$.
\end{proof}


\begin{corollary}\label{cor:Martingale.submartingale_norm}
  \uses{def:Martingale, def:Submartingale}
  Let $X : T \rightarrow \Omega \rightarrow E$ a martingale with values in a normed space $E$.
  Then $\Vert X \Vert$ is a sub-martingale.
\end{corollary}

\begin{proof}
  \uses{lem:convex_of_mg_is_submg}
It is a consequence of Lemma~\ref{lem:convex_of_mg_is_submg} with $\phi = \Vert \cdot \Vert$.
\end{proof}


\begin{lemma}\label{lem:convex_of_submg_is_submg}
  \uses{def:Submartingale}
Let $X : T  \rightarrow \Omega \rightarrow \mathbb{R}^d$ a sub-martingale.
Let $\phi:\mathbb{R^d} \rightarrow \mathbb{R}$ convex increasing such that
$\phi(X_t)\in L^1(\Omega)$ for every $t\in T$. Then $\phi(X)$ is a sub-martingale.
\end{lemma}

\begin{proof}
  By Jensen and the fact that $\phi$ is increasing
  $\phi(X_t) \leq \phi\left( \mathbb{E}[X_T\ |\ \mathcal{F}_t] \right)\leq \mathbb{E}[\phi(X_T)\ |\ \mathcal{F}_t]$.
\end{proof}


\begin{lemma}[Doob's maximal inequality for $\mathbb{N}$]\label{lem:maximal_ineq}
  \uses{def:Submartingale}
  \mathlibok
  \lean{MeasureTheory.maximal_ineq}
Let $X : \mathbb{N} \rightarrow \Omega \rightarrow \mathbb{R}$ be a non-negative sub-martingale.
Then for every $n \in \mathbb{N}$ and $\lambda > 0$,
\begin{align*}
  \mathbb{P}\left(\sup_{i \le n}X_i\geq\lambda \right)
  \le \frac{\mathbb{E}\left[X_n \mathbb{I}_{\sup_{i \le n}X_i \ge \lambda}\right]}{\lambda}
  \le \frac{\mathbb{E}[X_n]}{\lambda}
  \: .
\end{align*}
\end{lemma}

\begin{proof}\leanok

\end{proof}


\begin{lemma}[Doob's maximal Inequality for countable]\label{lem:doob_countable}
  \uses{def:Submartingale}
  Let $X : I \rightarrow \Omega \rightarrow \mathbb{R}$ be a non-negative sub-martingale with $I$ countable.
  Then for every $M \in I,\lambda > 0$ and $p>1$ we have
  \begin{align*}
    P\left( \sup_{i\in I, i\leq M}X_i\geq\lambda \right)
    \le \frac{\mathbb{E}\left[X_M \mathbb{I}_{\sup_{i \le M}X_i \ge \lambda}\right]}{\lambda}
    \le \frac{\mathbb{E}[X_M]}{\lambda}
    \: .
  \end{align*}
\end{lemma}

\begin{proof}
  \uses{lem:maximal_ineq}
For any finite subset $J \subset I$ with $M \in J$, we have by Lemma~\ref{lem:maximal_ineq}
\begin{align*}
  P\left( \sup_{i\in J, i \le M}X_i\geq\lambda \right)
  \le \frac{\mathbb{E}\left[X_{M} \mathbb{I}_{\sup_{i \in J, i \le M}X_i \ge \lambda}\right]}{\lambda}
  \: .
\end{align*}
Then we build a countable increasing sequence of finite sets $J_n$ with $\sup_{i\in I, i\leq M}X_i = \sup_n\sup_{i\in J_n, i \le M}X_i$ and conclude by monotone convergence.
%See 8.1.1 Pascucci.
\end{proof}


\begin{lemma}[Doob Lp Inequality for countable]\label{lem:doob_Lp_countable}
  Let $X : I \rightarrow \Omega \rightarrow \mathbb{R}$ be a non-negative sub-martingale. Let $I$ be countable.
  For every $M\in I,\lambda > 0$ and $p>1$ we have
  \begin{align*}
    \mathbb{E}\left[ \sup_{i\in I, i \leq M}X_i^p \right]
    \leq \left(\frac{p}{p-1}\right)^p\mathbb{E}[X_M^p]
    \: .
  \end{align*}
  That is, for $\Vert \cdot \Vert_p$ the $L^p$ norm,
  $\left\Vert \sup_{i \le M}  X_i  \right\Vert_p
    \leq \frac{p}{p-1} \left\Vert X_M \right\Vert_p
    \: .$
\end{lemma}

\begin{proof}
  \uses{lem:doob_countable}
\begin{align*}
  \mathbb{E}\left[ \sup_{i \le M}X_i^p \right]
  = p \int_0^\infty \mathbb{P}\left( \sup_{i \le M}X_i \geq \lambda \right) \lambda^{p-1} d\lambda
\end{align*}
By Theorem~\ref{lem:doob_countable} and then Fubini's theorem, we have then
\begin{align*}
  \mathbb{E}\left[ \sup_{i \le M}X_i^p \right]
  &\le p \int_0^\infty \mathbb{E}\left[X_M \mathbb{I}_{\sup_{i \le M}X_i \ge \lambda}\right] \lambda^{p-2} d\lambda
  \\
  &= p \mathbb{E}\left[X_M \int_0^{\sup_{i \le M}X_i} \lambda^{p-2} d\lambda\right]
  \\
  &= \frac{p}{p - 1} \mathbb{E}\left[X_M (\sup_{i \le M}X_i)^{p-1}\right]
  \: .
\end{align*}
Then by Hölder's inequality,
\begin{align*}
  \mathbb{E}\left[ \sup_{i \le M}X_i^p \right]
  &\le \frac{p}{p - 1} \left(\mathbb{E}[X_M^p]\right)^{1/p} \left(\mathbb{E}\left[\sup_{i \le M}X_i^p \right]\right)^{(p-1)/p}
  \: .
\end{align*}
We then divide the two sides by $\left(\mathbb{E}\left[\sup_{i \le M}X_i^p \right]\right)^{(p-1)/p}$ and raise to the power $p$ to conclude.
%  8.1.1 Pascucci.
\end{proof}


\begin{theorem}[Doob Inequality]\label{thm:doob_ineq}
  Let $X:\mathbb{R}\times\Omega\rightarrow \mathbb{R}$ be a right-continuous non-negative sub-martingale.
  For every $T, \lambda>0$ and $p>1$ we have
  \begin{align*}
    P\left( \sup_{t\in[0,T]}X_t \geq \lambda \right)
    \leq \frac{\mathbb{E}[X_T \mathbb{I}_{\sup_{i \le T}X_i \ge \lambda}]}{\lambda}
    \leq \frac{\mathbb{E}[X_T]}{\lambda}
    \: .
  \end{align*}
\end{theorem}

\begin{proof}
  \uses{lem:doob_countable}
  % TODO: put a sketch of the proof here
  8.1.2 Pascucci.
\end{proof}


\begin{corollary}[Doob Inequality for normed spaces]\label{cor:doob_ineq_norm}
  Let $X:\mathbb{R}\times\Omega\rightarrow E$ be a right-continuous martingale with values in a normed space $E$.
  For every $T$ and $\lambda>0$ we have
  $$
  P\left( \sup_{t\in[0,T]} \lVert X_t \rVert \geq \lambda \right)
  \leq \frac{\mathbb{E}[\lVert X_T \rVert]}{\lambda}.
  $$
\end{corollary}

\begin{proof}
  \uses{cor:Martingale.submartingale_norm, thm:doob_ineq}
  By Corollary~\ref{cor:Martingale.submartingale_norm}, $\lVert X \rVert$ is a sub-martingale.
  Then apply Theorem~\ref{thm:doob_ineq}.
\end{proof}


\begin{theorem}[Doob's Lp inequality in $\mathbb{R}$]\label{thm:doob_lp}
  Let $X:\mathbb{R} \rightarrow \Omega \rightarrow \mathbb{R}$ be a right-continuous non-negative sub-martingale.
  For every $T, \lambda>0$ and $p>1$ we have
  \begin{align*}
    \mathbb{E}\left[ \sup_{t\in[0,T]}X_t^p \right]
    \leq \left(\frac{p}{p-1}\right)^p\mathbb{E}[X_T^p]
    \: .
  \end{align*}
  That is, for $\Vert \cdot \Vert_p$ the $L^p$ norm,
  $\left\Vert \sup_{t\in[0,T]}  X_t  \right\Vert_p
    \leq \frac{p}{p-1} \left\Vert X_T \right\Vert_p
    \: .$
\end{theorem}

\begin{proof}
  \uses{thm:doob_ineq}
% TODO: put a sketch of the proof here
8.1.2 Pascucci.
\end{proof}


\begin{corollary}[Doob's Lp inequality for normed spaces]\label{cor:doob_lp_norm}
  Let $X : \mathbb{R} \rightarrow \Omega\rightarrow E$ be a right-continuous martingale with values in a normed space $E$.
  For every $T, \lambda>0$ and $p>1$ we have
  \begin{align*}
    \mathbb{E}\left[ \sup_{t\in[0,T]} \lVert X_t \rVert ^ p \right]
    \leq \left(\frac{p}{p-1}\right)^p\mathbb{E}[\lVert X_T \rVert ^p]
    \: .
  \end{align*}
  That is, for $\Vert \cdot \Vert_p$ the $L^p$ norm,
  $\left\Vert \sup_{t\in[0,T]}  X_t  \right\Vert_p
    \leq \frac{p}{p-1} \left\Vert X_T \right\Vert_p
    \: .$
\end{corollary}

\begin{proof}
  \uses{lem:convex_of_mg_is_submg, thm:doob_lp}
  By Lemma~\ref{lem:convex_of_mg_is_submg}, $\lVert X \rVert$ is a sub-martingale.
  Then apply Theorem~\ref{thm:doob_lp}.
\end{proof}


\begin{lemma}[Stopped Martingale]\label{lem:stop_of_mg_is_mg}
  \uses{def:Martingale, def:stoppedProcess}
  Let $X:\mathbb{R}\times\Omega\rightarrow \mathbb{R}$ be a cadlag martingale and $\tau_0$ a stopping time.
  Then the stopped process $(X_{t\wedge\tau_0})_{t\geq 0}$ is a martingale.
\end{lemma}

\begin{proof}

\end{proof}

\begin{lemma}[Doob Inequality for stopping times]\label{lem:doob_ineq_stop}
  Let $X:\mathbb{R}\times\Omega\rightarrow \mathbb{R}$ be a right-continuous non-negative sub-martingale.
  For every $\lambda>0$ and $p>1$ and $\tau$ stopping time a.s. bounded by $T>0$, we have
  $$
  P\left( \sup_{t\in[0,\tau]}X_t\geq\lambda \right)\leq \frac{\mathbb{E}[X_\tau]}{\lambda}.
  $$
\end{lemma}
\begin{proof}
  \uses{thm:doob_ineq, lem:stop_of_mg_is_mg}
  Almost already in mathlib MeasureTheory.Submartingale.stoppedProcess.
\end{proof}

\begin{corollary}[Doob Inequality for stopping times in normed spaces]\label{cor:doob_ineq_stop}
  Let $X:\mathbb{R}\times\Omega\rightarrow E$ be a right-continuous martingale with values in a normed space $E$.
  For every $\lambda>0$ and $p>1$ and $\tau$ stopping time a.s. bounded by $T>0$, we have
  $$
  P\left( \sup_{t\in[0,\tau]}\lVert X_t \rVert \geq\lambda \right)\leq \frac{\mathbb{E}[\lVert X_\tau \rVert]}{\lambda}.
  $$
\end{corollary}
\begin{proof}
  \uses{lem:convex_of_mg_is_submg, lem:doob_ineq_stop}
  By Lemma~\ref{lem:convex_of_mg_is_submg}, $\lVert X \rVert$ is a sub-martingale.
  Then apply Theorem~\ref{lem:doob_ineq_stop}.
\end{proof}

\begin{lemma}[Doob's Lp Inequality for stopping times]\label{lem:doob_ineq_stop_exp_val}
  Let $X:\mathbb{R}\times\Omega\rightarrow \mathbb{R}$ be a right-continuous non-negative sub-martingale.
  For every $\lambda>0$ and $p>1$ and $\tau$ stopping time a.s. bounded by $T>0$, we have
  $$
  \mathbb{E}\left[ \sup_{t\in[0,\tau]}X_t^p \right]\leq \left(\frac{p}{p-1}\right)^p\mathbb{E}[X_\tau^p].
  $$
\end{lemma}
\begin{proof}
  \uses{lem:doob_ineq_stop, lem:stop_of_mg_is_mg}
  8.1.3 Pascucci.
\end{proof}

\begin{corollary}[Doob's Lp Inequality for stopping times in normed spaces]\label{cor:doob_ineq_stop_exp_val}
  Let $X:\mathbb{R}\times\Omega\rightarrow E$ be a right-continuous martingale with values in a normed space $E$.
  For every $\lambda>0$ and $p>1$ and $\tau$ stopping time a.s. bounded by $T>0$, we have
  $$
  \mathbb{E}\left[ \sup_{t\in[0,\tau]}\lVert X_t \rVert^p \right]\leq \left(\frac{p}{p-1}\right)^p\mathbb{E}[\lVert X_\tau \rVert^p].
  $$
\end{corollary}
\begin{proof}
  \uses{lem:convex_of_mg_is_submg, lem:doob_ineq_stop_exp_val}
  By Lemma~\ref{lem:convex_of_mg_is_submg}, $\lVert X \rVert$ is a sub-martingale.
  Then apply Theorem~\ref{lem:doob_ineq_stop_exp_val}.
\end{proof}

\section{Square integrable martingales}

In this section, $E$ denotes a complete normed space.

\begin{definition}[Square integrable martingales]\label{def:squareIntegrableMartingales}
  \uses{def:Martingale}
Let $\mathcal{M}^2$ be the set of square integrable continuous martingales with respect to a filtration $\mathcal{F}$ indexed by $\mathbb{R}_+$,
\begin{align*}
  \mathcal{M}^2
  = \{ M : \mathbb{R}_+ \to \Omega \to \mathbb{R} \mid M \text{ continuous martingale with } \sup_{t}\mathbb{E}[M_t^2] < \infty \}
  \: .
\end{align*}
\end{definition}


\begin{theorem}\label{thm:hilbertSpace_squareIntegrableMartingales}
  \uses{def:squareIntegrableMartingales}
The space $\mathcal{M}^2$ is a Hilbert space with the inner product defined by
\begin{align*}
  \langle M, N \rangle = \mathbb{E}[M_\infty N_\infty]
  \: .
\end{align*}
\end{theorem}

\begin{proof}
  \uses{cor:doob_lp_norm}

\end{proof}


\section{Local martingales}

TODO: filtrations should be assumed right-continuous and complete whenever needed.


\begin{definition}[Quadratic variation]\label{def:quadraticVariation}
  \uses{def:IsLocalMartingale}
For any continuous local martingale $M$, there exists a continuous process $[M]$ with $[M]_0 = 0$ such that $M^2 - [M]$ is a local martingale. That process is a.s. unique and is called the \emph{quadratic variation} of $M$.
\end{definition}


\begin{definition}[Covariation]\label{def:covariation}
  \uses{def:IsLocalMartingale}
For any continuous local martingales $M$ and $N$, there exists a continuous process $[M,N]$ with $[M,N]_0 = 0$ such that $MN - [M,N]$ is a local martingale. That process is a.s. unique and is called the \emph{covariation} of $M$ and $N$.

It can be defined by $[M, N]_t = \frac{1}{4}\left([M+N]_t - [M-N]_t \right)$~.
\end{definition}


\begin{lemma}\label{lem:covariation_eq_inner}
  \uses{def:covariation, def:squareIntegrableMartingales}
Let $M$ and $N$ be continuous square integrable martingales. Then
\begin{align*}
  \mathbb{E}\left[[M,N]_\infty\right] = \langle M - M_0, N - N_0 \rangle_{\mathcal{M}^2}
  \: .
\end{align*}
\end{lemma}

\begin{proof}

\end{proof}


\begin{lemma}\label{lem:quadraticVariation_brownian}
  \uses{def:brownian, def:quadraticVariation}
Let $B$ be a standard Brownian motion. Then the quadratic variation of $B$ is given by $[B]_t = t$~.
\end{lemma}

\begin{proof}

\end{proof}


\begin{definition}[Continuous semi-martingale]\label{def:continuousSemiMartingale}
  \uses{def:IsLocalMartingale}
A continuous semi-martingale is a process that can be decomposed into a local martingale and a finite variation process.
More formally, a process $X : \mathbb{R}_+ \to \Omega \to E$ is a continuous semi-martingale if there exists a continuous local martingale $M$ and a continuous adapted process $A$ with locally finite variation and $A_0 = 0$ such that
\begin{align*}
  X_t = M_t + A_t
\end{align*}
for all $t \ge 0$.
The decomposition is a.s. unique.
\end{definition}


\section{Stochastic integral}


\begin{definition}[Simple process]\label{def:simpleProcess}
Let $0 \le t_0 \le t_1 \le \ldots \le t_n$ in $\mathbb{R}_+$.
Let $(\eta_k)_{0 \le k \le n-1}$ be $\mathcal{F}_{t_k}$-measurable random variables.
Then the simple process for that sequence is the process $V : \mathbb{R}_+ \to \Omega \to E$ defined by
\begin{align*}
  V_t = \sum_{k=0}^{n-1} \eta_k \mathbb{1}_{(t_k, t_{k+1}]}(t)
  \: .
\end{align*}
Let $\mathcal{E}$ be the set of simple processes.
\end{definition}


\begin{definition}[Elementary stochastic integral]\label{def:elemStochIntegral}
  \uses{def:simpleProcess}
Let $V \in \mathcal{E}$ be a simple process and let $X$ be a stochastic process.
The \emph{elementary stochastic integral} process $V \cdot X : \mathbb{R}_+ \to \Omega \to E$ is defined by
\begin{align*}
  (V \cdot X)_t
  &= \sum_{k=0}^{n-1} \eta_k (X^t_{t_{k+1}} - X^t_{t_k})
  \: .
\end{align*}
\end{definition}


\begin{lemma}\label{lem:sq_norm_elemStochIntegral}
  \uses{def:elemStochIntegral}
For $V \in \mathcal{E}$ and $M \in \mathcal{M}^2$, then $V \cdot M \in \mathcal{M}^2$ and
\begin{align*}
  \Vert V \cdot M \Vert_{\mathcal{M}^2}^2
  &= \mathbb{E}\left[ \int_0^{\infty} V_t^2 \: d[M]_t \right]
  \: .
\end{align*}
\end{lemma}

\begin{proof}

\end{proof}


\subsection{Itô isometry}

\begin{definition}\label{def:L2M}
  \uses{def:squareIntegrableMartingales}
Let $M \in \mathcal{M}^2$ be a continuous square integrable martingale. We define
\begin{align*}
  L^2(M) = L^2(\Omega \times \mathbb{R}_+, \mathcal{P}, \mathbb{P} \times d[M])
\end{align*}
in which $\mathcal{P}$ is the predictable $\sigma$-algebra and $d[M]$ is the measure induced by the quadratic variation of $M$.
The norm on that Hilbert space is $\Vert X \Vert^2 = \mathbb{E}\left[ \int_0^{\infty} X_t^2 \: d[M]_t \right]$~.
\end{definition}

TODO the sources don't use the same assumptions: predictable vs progressive (\texttt{MeasureTheory.ProgMeasurable}). Progressive would be more general.


\begin{lemma}\label{lem:dense_simpleProcess}
  \uses{def:L2M, def:simpleProcess}
Let $M \in \mathcal{M}^2$. Then the set of simple processes is dense in $L^2(M)$.
\end{lemma}

\begin{proof}

\end{proof}


\begin{definition}[Itô isometry]\label{def:itoIsometry}
  \uses{lem:dense_simpleProcess, lem:sq_norm_elemStochIntegral, thm:hilbertSpace_squareIntegrableMartingales}
Let $M \in \mathcal{M}^2$. Then the elementary stochastic integral map $\mathcal{E} \to \mathcal{M}^2$ defined by $V \mapsto V \cdot M$ extends to an isometry $L^2(M) \to \mathcal{M}^2$.
\end{definition}


\begin{lemma}\label{lem:inner_itoIsometry}
  \uses{def:itoIsometry}
$\langle X \cdot M, Y \cdot M \rangle_{\mathcal{M}^2} = (XY) \cdot \langle M, N \rangle_{\mathcal{M}^2}$.
\end{lemma}

\begin{proof}

\end{proof}


\subsection{Local martingales}

\begin{definition}[$L^2_{loc}(M)$]\label{def:L2locM}
  \uses{def:L2M}
Let $M$ be a continuous local martingale.
We define $L^2_{loc}(M)$ as the space of predictable processes $X$ such that for all $t \ge 0$, $\mathbb{E}\left[ \int_0^t X_s^2 \: d[M]_s \right] < \infty$.
\end{definition}


\begin{definition}[Stochastic integral for continuous local martingales]\label{def:locStochIntegral}
  \uses{def:L2locM, def:itoIsometry}
Let $M$ be a continuous local martingale and let $X \in L^2_{loc}(M)$.
We define the local stochastic integral $X \cdot M$ as the unique continuous local martingale with $(X \cdot M)_0 = 0$ such that for any continuous local martingale $N$, almost surely,
\begin{align*}
  [X \cdot M, N] = X \cdot [M, N]
  \: .
\end{align*}
\end{definition}


\subsection{Semi-martingales}

\begin{definition}\label{def:stochIntegral}
  \uses{def:continuousSemiMartingale, def:locStochIntegral}
For a continuous semi-martingale $X = M + A$ and $V \in L^2_{semi}(X)$ (to be defined) we define the stochastic integral as
\begin{align*}
  V \cdot X = V \cdot M + V \cdot A
  \: ,
\end{align*}
in which $V \cdot M$ is the local stochastic integral defined in \ref{def:locStochIntegral} and $V \cdot A$ is the Lebesgue-Stieltjes integral with respect to the locally finite variation process $A$.
\end{definition}


For $X = M + A$ and $Y = N + B$, we define the covariation as
\begin{align*}
  [X, Y] = [M, N]
  \: .
\end{align*}

\section{Itô formula}


\begin{theorem}[Integration by parts]\label{thm:integration_by_parts}
  \uses{def:continuousSemiMartingale, def:stochIntegral}
Let $X$ and $Y$ be two continuous semi-martingales. Then we have almost surely
\begin{align*}
  X_t Y_t - X_0 Y_0
  = (X \cdot Y)_t + (Y \cdot X)_t + [X,Y]_t
  \: .
\end{align*}
\end{theorem}

\begin{proof}

\end{proof}


\begin{theorem}[Itô's formula]\label{thm:Ito_formula}
  \uses{def:continuousSemiMartingale}
Let $X^1, \ldots, X^d$ be continuous semi-martingales and let $f : \mathbb{R}^d \to \mathbb{R}$ be a twice continuously differentiable function.
Then, writing $X = (X^1, \ldots, X^d)$, the process $f(X)$ is a semi-martingale and we have
\begin{align*}
  f(X_t)
  &= f(X_0)
  + \sum_{i=1}^d \int_0^t \frac{\partial f}{\partial x_i}(X_s) \: dX^i_s
  + \frac{1}{2} \sum_{i,j=1}^d \int_0^t \frac{\partial^2 f}{\partial x_i \partial x_j}(X_s) \: d[X^i, X^j]_s
  \: .
\end{align*}
\end{theorem}

\begin{proof}
  \uses{thm:integration_by_parts}

\end{proof}


\bibliographystyle{amsalpha}
\bibliography{bib}
